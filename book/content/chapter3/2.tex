
本示例中,将使用Dockerfile创建Docker容器,并在持续集成(CI)工作流程中使用它。每次更改内容时,工作流程都会运行该操作。

\mySubsubsection{3.2.1}{Getting ready}

创建一个名为DockerActionRecipe的新存储库。将其设为公开,并用README文件初始化(见图3.1):

\myGraphic{0.6}{content/chapter3/images/1.png}{图3.1 --- 为Docker容器操作创建一个新库}

在本地克隆该库,并在VS Code或GitHub Codespaces中打开它。

\mySubsubsection{3.2.2}{How to do it…}

\begin{enumerate}
\item 
在库的根目录下创建一个名为Dockerfile的新文件,并向文件中添加以下内容:

\begin{shell}
# Container image that runs your code
FROM alpine:latest
CMD echo "Hello World"
\end{shell}

这将基于最新的Alpine镜像创建一个镜像,并添加一个将“Hello World”输出到控制台的层。

\item 
使用以下命令在本地运行Docker容器:

\begin{shell}
$ docker run $(docker build -q .)
\end{shell}

它将创建一个镜像(docker build)并运行它(docker run),应该能够在控制台上看到Hello World。

\item 
为了更灵活,创建一个名为entrypoint.sh的新文件,并添加以下内容:

\begin{shell}
#!/bin/sh -l
echo "Hello World"
\end{shell}

\item 
现在,调整Dockerfile,使其执行脚本而不是直接写入控制台。将脚本文件复制到容器的根目录,然后将其作为入口点:

\begin{shell}
FROM alpine:3.10
COPY entrypoint.sh /entrypoint.sh
RUN chmod +x entrypoint.sh
ENTRYPOINT ["/entrypoint.sh"]
\end{shell}

\item 
请注意,这里添加了 \verb|chmod +x entrypoint.sh| 命令来使脚本可执行。否则,本地运行容器时会失败,并显示一条消息:\verb|exec: /entrypoint.sh": permission denied|(权限被拒绝)。在所有类Unix的系统中,可以在本地运行 \verb|chmod +x entrypoint.sh|,并在提交到git时附加该属性。在Windows上,可以使用git来设置文件权限:

\begin{shell}
$ git add entrypoint.sh
$ git update-index --chmod=+x entrypoint.sh
\end{shell}
  
再次运行Docker容器,应该会再次看到Hello World——这次是从脚本文件输出:

\begin{shell}
$ docker run $(docker build -q .)
\end{shell}

\item 
在这个操作中,我们想要使用一个输入参数,这就是要对脚本参数化的原因。将单词World替换为传递给Docker的参数(\$@代表所有参数):

\begin{shell}
#!/bin/sh -l
echo "Hello $@"
\end{shell}

尝试再次在本地运行容器,并传入一些单词。容器将进行输出,如下所示:

\begin{shell}
$ docker run $(docker build -q .) foo bar
> Hello foo bar
\end{shell}

\item 
接下来,向库中添加一个名为action.yml的新文件,并添加以下输入:

\begin{shell}
name: 'Docker Action Recipe'
description: 'Greet someone'
inputs:
  who-to-greet:
    description: 'Who to greet'
    required: true
    default: 'World'
  runs:
    using: 'docker'
    image: 'Dockerfile'
    args:
      - ${{ inputs.who-to-greet }}
\end{shell}

\item 
此时,操作已经准备好了。为了测试,需要添加一个名为.github/workflows/ci.yml的本地工作流文件,将在每次推送时运行。它会下载库,并使用自定义输入参数执行相应的操作:

\begin{shell}
name: Action CI

on: [push]

jobs:
  ci:
    runs-on: ubuntu-latest
    steps:
      - uses: actions/checkout@v4.1.1
      - name: Run my own container action
        uses: ./
        with:
          who-to-greet: '@wulfland'
\end{shell}

\begin{myTip}{引用本地操作}
请注意,我们是通过本地路径 ./ 引用操作的——这就是为什么必须首先使用检出操作。工作流程将使用与工作流程运行时相同的版本,也可以通过指定 <owner>/DockerActionRecipe@main 来正常引用操作——就像从另一个库引用它一样。
\end{myTip}

\item 
提交并推送所有更改。推送触发器将自动运行工作流,可以检查操作的输出。它应该看起来像图3.2所示:

\myGraphic{0.6}{content/chapter3/images/2.png}{图3.2 --- 工作流操作的输出}

通过Docker守护程序和传递给操作的参数,来检查操作的输出。
\end{enumerate}

\mySubsubsection{3.2.3}{How it works…}

有三种不同类型的操作:

\begin{itemize}
\item 
Docker容器操作

\item 
JavaScript操作

\item 
复合操作
\end{itemize}

Docker容器操作仅运行在Linux上,而JavaScript操作和复合操作可以在没有平台限制。所有操作都由一个名为action.yml(或action.yaml)的文件定义,该文件包含定义操作的元数据。这个文件名不能变,所以一个操作必须位于自己的库或文件夹中。action.yml文件中的runs部分定义了操作的类型。

Docker容器操作包含它们所有的依赖项在容器中,因此一致。允许使用其他语言开发操作——唯一的限制是必须运行在Linux上。由于检索或构建镜像和启动容器所需的时间,Docker容器操作比JavaScript操作慢。

Docker容器操作可以引用容器注册表中的镜像,例如:Docker Hub或GitHub Packages,或者可以在运行时构建相应的Dockerfile。这时,必须在action.yml文件中将Dockerfile指定为镜像名称。

\mySubsubsection{3.2.4}{There’s more…}

容器操作非常强大,可以使用任何语言进行编写。可以向工作流返回输出参数,向工作流日志写入消息,在拉取请求中注释文件,并编写作业摘要。在下一个简短的示例中,我们将添加输出参数并写入作业摘要。