
这个示例中,将从模板创建一个基本的 TypeScript 操作,构建和发布,并在工作流中使用。

\mySubsubsection{3.4.1}{Getting ready}

确保安装了较新版本的 Node.js(\url{https://nodejs.org/en/download})。请按照以下步骤操作:

\begin{enumerate}
\item 
前往 \url{https://github.com/actions/typescript-action},然后点击右上角的 \textbf{Use this template | Create a new repository}(使用此模板|创建新库)(如图 3.6 所示):

\myGraphic{0.6}{content/chapter3/images/6.png}{图3.6 --- 使用 typescript-action 模板创建新库}

\item 
选择您的 GitHub 帐户作为所有者,将其命名为 TypeScriptActionRecipe,将其可见性保留为Public(公开),然后单击\textbf{Create repository}(创建库)。

\item 
本地克隆该库,并在 VS Code 中打开该文件夹。
\end{enumerate}

\mySubsubsection{3.4.2}{How to do it…}

\begin{enumerate}
\item 
打开终端,并转到库的根目录。安装所有必要的依赖项:

\begin{shell}
$ npm install
\end{shell}

该库包含一些单元测试,运行它们以检查一切是否正常:

\begin{shell}
$ npm test
\end{shell}

\item 
打开 action.yml 文件,并更新 name(名称)、description(描述)和 author(作者)的元数据信息:

\begin{shell}
name: 'TypeScript Action Recipe'
description: 'Waites for some milliseconds, writes an awesome job summary to the workflow output and returns the current date and time.'
author: 'Michael Kaufmann'
\end{shell}

暂时忽略品牌信息(branding) --- 但请注意,该操作具有定义的输入参数(milliseconds)和输出参数(time):

\begin{shell}
# Define your inputs here.
inputs:
  milliseconds:
    description: 'Your input description here'
    required: true
    default: '1000'

# Define your outputs here.
outputs:
  time:
    description: 'Your output description here'
\end{shell}

\item 
打开 src/main.ts 文件并找到 run 函数:

\begin{shell}
export async function run(): Promise<void> {
\end{shell}

这里使用了工具包(@actions/core;参见 \url{https://github.com/actions/toolkit})来读取输入参数:

\begin{shell}
const ms: string = core.getInput('milliseconds')
\end{shell}

还使用工具包来写入调试消息。这类似于我们在第 2 章中使用的 \verb|echo "::debug::{debug message}"| 工作流命令:

\begin{shell}
core.debug(`Waiting ${ms} milliseconds ...`)
\end{shell}

对于设置输出参数也是如此。这相当于写入 GITHUB\_OUTPUT 环境文件:

\begin{shell}
echo "answer=42" >> $GITHUB_OUTPUT
\end{shell}

使用工具包的代码为:

\begin{shell}
core.setOutput('time', new Date().toTimeString())
\end{shell}

\item 
下一步是在 run 函数中的 core.setOutput 下面编写作业摘要。从 core.summary 开始,并注意可以使用自动完成功能,来了解所有可用的功能和语法。添加一个 <h2> 标题:

\begin{shell}
// Write an advanced job summary
core.summary
  .addHeading('Advanced Job Summary', 'h2')
\end{shell}

core.summary 对象具有流式接口(fluent interface),所以可以直接将新方法链接到前一个方法的输出中。接下来,添加一张图片并将其大小设置为 64x64:

\begin{shell}
.addImage(
  'https://octodex.github.com/images/droidtocat.png',
  'Droidtocat',
    {
    width: '64',
    height: '64'
    }
  )
\end{shell}

添加一个带有一些数据的表格:

\begin{tcolorbox}[ breakable,colback = bashcodebg, colframe= black!50!white]
\scriptsize{
.addTable([ \\
\hspace*{1em}[ \\
\hspace*{2em}\{ data: 'File', header: true \}, \\
\hspace*{2em}\{ data: 'Result', header: true \} \\
\hspace*{1em}], \\
\hspace*{1em}['foo.js', 'Pass \emoji{✅}'], \\
\hspace*{1em}['bar.js', 'Fail \emoji{❌}'], \\
\hspace*{1em}['test.js', 'Pass \emoji{✅}'] \\
])
}
\end{tcolorbox}

以及一个简单的链接:

\begin{shell}
.addLink('My custom link', 'https://writeabout.net'
\end{shell}

需要使用 write 函数来完成摘要,以将缓冲区写入环境文件:

\begin{shell}
.write()
\end{shell}

可以在此处查看代码:\url{https://github.com/wulfland/TypeScriptActionRecipe}。

\item 
打包 TypeScript 以便分发。这是一个重要步骤,每次修改 .ts 文件时都必须执行!

\begin{shell}
$ npm run bundle
\end{shell}

\item 
提交更改。这将自动触发一些工作流程——行时,可以使用这些时间来查看它们在做什么。.github/workflows/linter.yml 在对 main 的所有推送和拉取请求上运行,其会对您的代码库进行检查:

\begin{shell}
- name: Lint Code Base
  id: super-linter
  uses: super-linter/super-linter/slim@v5
  env:
    DEFAULT_BRANCH: main
    GITHUB_TOKEN: ${{ secrets.GITHUB_TOKEN }}
    TYPESCRIPT_DEFAULT_STYLE: prettier
    VALIDATE_JSCPD: false
\end{shell}

如果这失败了,可能忘记在提交更改之前运行 npm run bundle,还可以在本地运行 prettier 以确保已遵守了 检查 标准:

\begin{shell}
$ npx prettier . --check
$ npx prettier . --write
\end{shell}

.github/workflows/codeql-analysis.yml 工作流也将在每次推送和对 main 的拉取请求上运行——但也会每周运行一次,将扫描代码以查找安全漏洞。

Check dist(.github/workflows/check-dist.yml)是一个简单的工作流,将运行 npm run bundle 并将输出与您的 dist 文件夹进行比较。如果忘记使用小型简单脚本来打包您的 TypeScript,它将失败:

\begin{shell}
  - name: Compare Expected and Actual Directories
    id: diff
    run: |
      if [ "$(git diff --ignore-space-at-eol --text dist/ | wc -l)" -gt "0" ]; then
        echo "Detected uncommitted changes after build. See status below:"
        git diff --ignore-space-at-eol --text dist/
        exit 1
      fi
\end{shell}

后一个是在 CI 构建(.github/workflows/ci.yml)。它有两个作业:一个安装所有依赖项并运行单元测试,而另一个执行您的操作并使用输出,就像第 2 章中所做的那样:

\begin{shell}
- name: Test Local Action
  id: test-action
  uses: ./
  with:
    milliseconds: 1000

- name: Print Output
  id: output
  run: echo "${{ steps.test-action.outputs.time }}"
\end{shell}

test-typescript 作业将失败,因为没有调整单元测试,但第二个作业应该成功。此时,可以检查工作摘要,其应该看起来像图 3.6 中所示的那样:

\myGraphic{0.6}{content/chapter3/images/7.png}{图3.7 --- 使用工具包创建的工作摘要}

同时,在工作流程日志中检查输出参数的值(见图 3.8):

\myGraphic{0.6}{content/chapter3/images/8.png}{图3.8 --- 工作流程日志中 TypeScript 操作的输出}

\begin{myTip}{修复单元测试}
我没有在这本书中包含调整单元测试的示例,这本书是关于 GitHub Actions 而非 TypeScript。但想要修复测试,可以查看:\url{https://github.com/wulfland/TypeScriptActionRecipe/blob/main/__tests__/main.test.ts}。
\end{myTip}

\item 
回到 main.ts 的末尾,如果遇到错误,操作将会失败。其通过使用 core.setFailed ,而非返回非零值来实现:

\begin{shell}
} catch (error) {
  // Fail the workflow run if an error occurs
  if (error instanceof Error) core.setFailed(error.message)
}
\end{shell}

\item 
创建一个名为 fail-ci-build 的新分支并切换:

\begin{shell}
$ git switch -c fail-ci-build
\end{shell}

在包含 core.setFailed 的行之前,向 catch 块中添加以下代码块:

\begin{shell}
core.error('Something bad happened', {
    title: 'Bad Error',
    file: '.github/workflows/ci.yml',
    startLine: 59,
    startColumn: 11,
    endColumn: 23
  })
\end{shell}

在 .github/workflows/ci.yml 中,修改 milliseconds 参数为无法转换为整数的值:

\begin{shell}
- name: Test Local Action
  id: test-action
  uses: ./
  with:
    milliseconds: xxx
\end{shell}

打包 TypeScript,提交更改,并创建一个拉取请求:

\begin{shell}
$ npm run bundle
$ git add .
$ git commit -m "Fail CI build"
$ git push –set-upstream origin fail-ci-build
$ gh pr create --fill
\end{shell}
  
\item 
检查创建的拉取请求,并注意日志中来自 core.setFailed 和 core.error 的输出(见图 3.9):

\myGraphic{0.6}{content/chapter3/images/9.png}{图3.9 --- 工作流程日志中失败操作的输出}

同时,请注意注释在工作流程文件中显示(见图 3.10):

\myGraphic{0.6}{content/chapter3/images/10.png}{图3.10 --- 来自工具包的注释}

围绕 TypeScript 和工具包的工具使得开发 GitHub Actions 变得更加便捷,并且在编写高质量的动作时提供了巨大的帮助
\end{enumerate}

\mySubsubsection{3.4.3}{How it works…}

TypeScript 操作与容器操作没有太大区别 --- 它们只是在 Node.js 环境中运行,而非 Docker 容器中。TypeScript 只是在 JavaScript 之上的一层,如果运行 npm run bundle,会转译成 JavaScript(进入 /dist 文件夹)。如果你是 TypeScript 新手,所有这些工具可能看起来令人不知所措——但这些工具也使得入门变得相当容易。VS Code 中的自动完成和 IntelliSense、自动检查 和格式化、带有模拟的单元测试——有很多工具有助于编写良好的代码。

\mySubsubsection{3.4.4}{There’s more…}

我们只是接触了工具包(\url{https://github.com/actions/toolkit}),并了解了其一些功能。它还可以在使用 GitHub REST 或 GraphQL API、OICD 令牌等功能时给予我们帮助。如果想要在 GitHub Actions 中做更多事情,了解一点 TypeScript 很有必要,从而能更好借助工具包的力量。
