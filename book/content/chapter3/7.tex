GitHub Actions 的强大之处在于其社区 --- 分享即是关爱。因此,GitHub Marketplace 在赋能基于社区的工作流方面扮演着至关重要的角色。在本教程中,您将为其中一个操作添加品牌信息和其他元数据,并将其分享到市场上。

\mySubsubsection{3.7.1}{Getting ready}

我将使用我们之前创建的 Docker 容器操作来完成本教程 --- 但您也可以使用 TypeScript 操作或复合操作。这无关紧要。只要该操作位于其自己的公共仓库中,就可以正常工作。

\mySubsubsection{3.7.2}{How to do it…}

\begin{enumerate}
\item 
在浏览器中导航到您仓库的根目录。GitHub 会检测到您的仓库包含一个 action.yml 文件,并会在蓝色横幅中建议您发布一个版本(见图 3.12):

\myGraphic{0.4}{content/chapter3/images/12.png}{图3.12 --- 起草一个发布版本以将操作发布到市场}

这与进入 Releases 并点击 Draft a new release(/releases/new)是相同的。

\item 
在对话框中,GitHub 会显示一些警告,以帮助您改进市场列表(见图 3.13):

\myGraphic{0.4}{content/chapter3/images/13.png}{图3.13 --- 发布操作到市场的指南}

点击可用图标和颜色的链接,各选一个。我将选择铃铛(bell)和紫色(purple)。

\item 
打开您的操作。在VS代码中添加品牌和作者信息:

\begin{shell}
name: 'Docker Action Recipe'
description: 'Greet someone'
branding:
  icon: bell
  color: purple
author: 'Michael Kaufmann'
\end{shell}

\item 
一个好的 README.md 文件对市场列表非常重要。添加输入、输出以及使用示例的部分。可以参考 \url{https://github.com/wulfland/DockerActionRecipe} 获取建议。

\item 
提交并推动您的更改。

\item 
返回浏览器并刷新新版本发布窗口。检查应该现在没有任何警告,并且看起来像图 3.14 所示的内容:

\myGraphic{0.4}{content/chapter3/images/14.png}{图3.14 --- 检查成功,因为该操作具有唯一的名称、品牌标识、描述以及一个 README 文件。}

\item 
单击选择标签,输入v1.0,然后单击创建新标签(请参见图3.15):

\myGraphic{0.4}{content/chapter3/images/15.png}{图3.15 --- 创建或选择一个标签以创建发布版本}

\item 
将 v1.0 添加为发布的标题。请注意,您可以自动生成该版本的发布说明。然而,只有在使用拉取请求(pull requests)时,生成的结果才会非常好。您也可以手动添加描述。

\item 
点击 Publish release(发布版本)。在发布的版本中,您将看到 Marketplace 和 Latest 标签(见图 3.16):

\myGraphic{0.4}{content/chapter3/images/16.png}{图3.16 --- 存储库中发行版的内容}

\item 
单击市场标签。这将带您进入您的市场上市。

\item 
请注意,在您的 action.yml 文件中 branding 部分指定的颜色和图标会显示在操作名称旁边。您仓库的 README.md 文件将构成市场列表的主要内容。在右侧,您会看到一个 Delist 按钮(用于将操作从市场中移除)、一个版本选择器以及一些重要链接。第一个链接会带您返回到您的仓库(有关列表区域的详细信息,请参见图 3.17):

\myGraphic{0.4}{content/chapter3/images/17.png}{图3.17 --- 市场上架}

\item 
返回您的仓库,重复相同的步骤,创建一个名为 v1.1 的新版本。请注意,这次对话框中会有一个标志,显示这是最新的发布版本(见图 3.18)。如果您忘记了这一点,GitHub 将不会将该版本标记为最新版本 --- 这与您选择的版本号标签无关:

\myGraphic{0.4}{content/chapter3/images/18.png}{图3.18 --- 在创建或编辑版本时,您必须手动将某个版本设置为最新版本。}

\item 
最后,创建一个新的工作流文件或在浏览器中编辑现有的工作流。在右侧的市场窗口中,输入您操作的名称。您的操作应该会立即被找到。请注意,GitHub 会根据您的发布版本自动生成安装说明(见图 3.19):

\myGraphic{0.4}{content/chapter3/images/19.png}{图3.17 --- 工作流编辑器中的列表}

\item 
如果您不想将其保存在市场上,则将其贴上您的动作,如图3.17所示。
\end{enumerate}

\mySubsubsection{3.7.3}{How it works…}

GitHub Marketplace 是基于 GitHub 的发布版本构建的(详见 \url{https://docs.github.com/en/repositories/releasing-projects-on-github}),而发布版本又是基于 Git 标签构建的。

标签可以是任意文本 --- 但建议使用语义化版本控制(Semantic Versioning)来为版本赋予意义。

语义化版本控制是一种用于指定软件版本号的正式约定。它由具有不同含义的多个部分组成。语义化版本号的示例包括 1.0.0 或 2.5.99-beta。其格式如下:

\begin{shell}
<major>.<minor>.<patch>-<pre>
\end{shell}

让我们仔细看看:

\begin{itemize}
\item 
主版本号:一个数字标识符,如果版本不向后兼容且包含重大更改时增加。更新到新的主版本必须谨慎处理!主版本号为零表示初始开发阶段。

\item 
次版本号:一个数字标识符,如果有新功能添加但保持向后兼容时增加。可以更新到新的次版本而不破坏现有功能,如果需要新功能的话。

\item 
修订版本号:一个数字标识符,当发布向后兼容的错误修复时增加。新的修订版本应该总是安装。

\item 
预发布版本:一个文本标识符,通过连字符附加。该标识符只能使用 ASCII 字母数字字符和连字符([0-9A-Za-z-])。文本越长,预发布版本越小(意味着 -alpha < -beta < -rc)。预发布版本总是小于正常版本(1.0.0-alpha < 1.0.0)。
\end{itemize}


在 GitHub 发布版本中,最佳实践是使用 v 作为语义化版本的前缀(例如,v1.0、v1.0.1 等)。

有关语义化版本的完整规范,请参见 \url{https://semver.org/}。

\mySubsubsection{3.7.4}{There’s more…}

通过标签进行版本控制存在某种风险,因为任何对仓库有写权限的人都可以修改标签。因此,作为 GitHub 操作的维护者,建议您除了分支保护规则外,还使用标签保护规则(详见 \url{https://docs.github.com/en/repositories/managing-your-repositorys-settings-and-features/managing-repository-settings/configuring-tag-protection-rules})。例如,为 v* 设置标签保护规则后,将防止任何没有管理员权限的人修改以 v 开头的标签。

如果您希望自动化生成语义化版本和自动生成高质量的发布说明,可以使用约定式提交(Conventional Commits)(详见 \url{https://www.conventionalcommits.org})。约定式提交为每次提交添加一个前缀,表明它是一个功能(feature)、修复(fix),或者是否包含破坏性更改(breaking change)。您可以将其与 GitVersion(详见 \url{https://gitversion.net/docs/})结合使用,以自动为您的发布生成语义化版本。您将在第 7 章《使用 GitHub Actions 发布您的软件》中了解更多相关内容。