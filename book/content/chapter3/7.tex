GitHub Actions 的强大之处在于社区——分享即是关爱。这就是为什么 GitHub 市场在赋能基于社区的流程中发挥着重要作用。在这个示例中,将为我们自定义的操作添加品牌和其他元数据,并在市场中分享。

\mySubsubsection{3.7.1}{Getting ready}

我将使用之前创建的 Docker 容器操作,来完成这个指南——也可以使用 TypeScript 操作或复合操作,这无关紧要。只要操作位于自己的公共库中,它就可以工作。

\mySubsubsection{3.7.2}{How to do it…}

\begin{enumerate}
\item 
在浏览器中导航到库的根目录。GitHub 将检测到库包含一个 action.yml 文件,并且会在蓝色横幅中提议发布一个版本(见图 3.12):

\myGraphic{0.6}{content/chapter3/images/12.png}{图3.12 --- 起草一个版本以将操作发布到市场}

这与你转到\textbf{Releases}(发布)并点击\textbf{Draft a new release}(起草新版本)(/releases/new)相同。

\item 
在对话框中,GitHub 将显示一些警告,以帮助改进市场列表(见图 3.13):

\myGraphic{0.6}{content/chapter3/images/13.png}{图3.13 --- 发布操作到市场的指导}

点击可用图标和颜色的链接,各选一个。我选择铃铛(bell)和紫色(purple)。

\item 
在 VS Code 中打开你的 action.yml 文件,并添加品牌和作者信息:

\begin{shell}
name: 'Docker Action Recipe'
description: 'Greet someone'
branding:
  icon: bell
  color: purple
author: 'Michael Kaufmann'
\end{shell}

\item 
对于市场列表来说,一个好的 README.md 文件很重要。添加一个用于输入、输出和使用示例的部分。可以参考 \url{https://github.com/wulfland/DockerActionRecipe} 获取建议。

\item 
提交并推送更改。

\item 
返回你的浏览器并刷新新版本窗口。现在检查应该显示没有警告,并且看起来像图 3.14 所示的样子:

\myGraphic{0.6}{content/chapter3/images/14.png}{图3.14 --- 检查成功,有唯一的名称、品牌、描述和一个 README 文件}

\item 
点击\textbf{Choose a tag}(选择标签),输入 v1.0,然后点击\textbf{Create new tag}(创建新标签)(见图 3.15):

\myGraphic{0.6}{content/chapter3/images/15.png}{图3.15 --- 创建或选择一个标签来创建一个版本}

\item 
将 v1.0 添加为版本标题。请注意,可以为版本自动生成发布说明,只有在处理拉取请求时,结果才会非常好,也可以手动添加描述。

\item 
点击\textbf{Publish release}(发布版本),将看到市场(Marketplace)和最新(Latest)标签(见图 3.16):

\myGraphic{0.6}{content/chapter3/images/16.png}{图3.16 --- 库中版本信息}

\item 
点击市场标签,这将转到市场列表。

\item 
注意在操作名称旁边,有一个图标,颜色是在 action.yml 文件中的品牌部分指定的。库的 README.md 文件将涵盖列表的最大部分。在右侧,有一个按钮可以取消列出该操作(从市场中移除),一个版本选择器,以及重要链接。第一个链接将转回到你的库(见图 3.17 列表区域):

\myGraphic{0.6}{content/chapter3/images/17.png}{图3.17 --- 市场列表}

\item 
返回到库并通过重复相同的步骤创建一个名为 v1.1 的新版本。这次对话框有一个标记作为最新版本(见图 3.18)。如果忘记这一点,GitHub 将不会标记该版本为最新——无论选择的标签版本号是什么:

\myGraphic{0.6}{content/chapter3/images/18.png}{图3.18 --- 在创建或编辑版本时,必须手动设置一个版本为最新}

\item 
最后,在浏览器中创建一个新的工作流文件或编辑现有的工作流。在右侧,在市场窗口中输入操作的名称。操作应该立即被找到。请注意 GitHub 从版本自动创建的安装说明(见图 3.19):

\myGraphic{0.6}{content/chapter3/images/19.png}{图3.17 --- 工作流编辑器中的列表}

\item 
如果不想在市场中保留,请取消列出操作,如图 3.17 所示。
\end{enumerate}

\mySubsubsection{3.7.3}{How it works…}

GitHub 市场是建立在 GitHub 版本之上(\url{https://docs.github.com/en/repositories/releasing-projects-on-github}),而 GitHub 版本又是建立在 git 标签之上的。标签可以是任何文本——但鼓励使用语义化版本控制来为版本赋予意义。

语义化版本控制是为软件指定版本号的正式约定,由具有不同含义的不同部分组成。语义化版本号的示例是 1.0.0 或 1.5.99-beta。格式如下:

\begin{shell}
<major>.<minor>.<patch>-<pre>
\end{shell}

来仔细看看:

\begin{itemize}
\item 
主版本号:一个数字标识符,如果版本不向后兼容并且有破坏性更改,则该标识符会增加。必须谨慎处理对新主版本的更新!主版本号为零表示初始开发。

\item 
次版本号:一个数字标识符,如果添加了新功能但版本与先前版本向后兼容,并且如果你需要新功能,可以更新而不会破坏任何东西,则该标识符会增加。

\item 
修订版本号:一个数字标识符,如果你发布了向后兼容的 bug 修复,则该标识符会增加。应该始终安装新的补丁。

\item 
预发布版本:使用连字符附加的文本标识符。标识符必须仅使用 ASCII 字母数字字符和连字符([0-9A-Za-z-])。文本越长,预发布版本越小(意味着 -alpha < -beta < -rc)。预发布版本始终小于正常版本(1.0.0-alpha < 1.0.0)。
\end{itemize}


在 GitHub 发布中,最佳实践是在语义化版本前缀加上 v(例如,v1.0、v1.0.1 等)。

有关语义化版本的完整规范,请参见 \url{https://semver.org/}。

\mySubsubsection{3.7.4}{There’s more…}

使用标签进行版本控制会带来一些风险,对库具有写权限的人都可以修改标签,作为 GitHub 操作的维护者,鼓励在分支保护规则之外使用标签保护规则( \url{https://docs.github.com/en/repositories/managing-your-repositorys-settings-and-features/managing-repository-settings/configuring-tag-protection-rules})。例如,对 v* 的标签保护规则,将防止没有库管理员权限的人修改以 v 开头的标签。

如果想要自动化发布创建语义化版本和自动创建良好的发布说明的过程,可以使用约定式提交(Conventional Commits)(\url{https://www.conventionalcommits.org})。约定式提交在每个提交前添加一个前缀,指示它是功能还是修复,以及它是否破坏性的。可以将此与 GitVersion(详见 \url{https://gitversion.net/docs/})结合使用,为发布自动创建语义化版本。