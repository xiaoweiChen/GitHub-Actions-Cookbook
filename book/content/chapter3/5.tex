
除了 Docker 容器操作和 JavaScript/TypeScript 操作之外,第三种类型的操作是复合操作。复合操作是其他操作的包装器。这个示例中,将创建一个简单的复合操作,并在工作流程中进行使用——一次使用 bash 脚本,一次使用 GitHub 脚本。

\mySubsubsection{3.5.1}{Getting ready}

创建一个名为 CompositeActionRecipe 的新库,将其设为公开,并用 README 文件初始化。在本地克隆该库,并在 VS Code 中打开,或在 GitHub Codespaces 中打开。

\mySubsubsection{3.5.2}{How to do it…}

\begin{enumerate}
\item 
在库的根目录下添加一个名为 action.yml 的新文件。添加一个名称和描述:

\begin{shell}
name: 'Composite Action Recipe'
description: 'Greets the user and returns 42.'
\end{shell}

\item 
添加一个名为 who-to-greet 的输入和一个名为 answer 的输出。注意,需要步骤 ID 来访问输出。我们将在下一步中添加:

\begin{shell}
inputs:
  who-to-greet:
    description: 'Who to greet'
    required: true
    default: 'World'
outputs:
  answer:
    description: "Answer to life, the universe, and everything"
    value: ${{ steps.deep-thought.outputs.answer }}
\end{shell}

\item 
接下来,添加 runs 部分,将 using 设置为 composite,并添加一个执行 bash 脚本的步骤。在复合操作中,必须指定 shell,且不能依赖于默认的 shell:

\begin{shell}
runs:
  using: "composite"
  steps:
    - name: Awesome bash script action
      id: deep-thought
      shell: bash
      run: |
        echo "Hello '${{ inputs.who-to-greet }}'."
        echo "answer=42" >> $GITHUB_OUTPUT
        echo "So long, and thanks for all the fish."
\end{shell}

使用输入向工作流日志写入问候消息,并且以与在容器操作中相同的方式将输出参数设置为 42。唯一的区别是,直接在工作流程运行器上执行脚本,而非在容器中。

\item 
添加一个新的工作流程文件名为 .github/workflows/ci.yml,并配置它在 push 和 pull\_request 上运行:

\begin{shell}
name: CI Workflow
on: [push, pull_request]
\end{shell}

添加一个作业,它检出存储库并带有输入参数运行复合操作。给步骤一个 ID 以便访问输出,像这样:

\begin{shell}
jobs:
  ci-job:
    runs-on: ubuntu-latest
    steps:
      - uses: actions/checkout@v4.1.1
      - name: Run my own composite action
        id: my-action
        uses: ./
        with:
          who-to-greet: '@wulfland'
\end{shell}

然后,将输出写入控制台:

\begin{shell}
      - name: Output the answer
        run: echo "The answer is ${{ steps.my-action.outputs.answer }}"
\end{shell}

这些概念应该已经非常熟悉了。

\item 
提交并推送你的更改。这将触发工作流;可以检查工作流日志,看看操作如何执行的。
\end{enumerate}

\mySubsubsection{3.5.3}{How it works…}

复合操作是步骤和其他操作的包装器,可以使用它们来捆绑多个运行命令或操作——无论自定义的,还是来自市场——并且可以为组织中的用户提供其他操作的默认值。

复合操作在 action.yml 文件的 runs 部分中有步骤——就像正常工作流中会有的那样。可以使用 inputs 上下文来访问输入参数。输出参数可以使用步骤上下文中的 outputs 上下文来访问。

\mySubsubsection{3.5.4}{There’s more…}

复合操作直接在工作流运行器上执行,每个运行器都安装了 Node.js,所以也可以在复合操作中运行 JavaScript 和 TypeScript。我们将使用一个名为 github-script 的操作来利用工具包在复合操作中的力量,也可以直接在工作流中使用该操作。