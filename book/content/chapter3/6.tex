在这个示例中,我们将使用 github-script 操作在复合操作展示工具包的强大功能。

\mySubsubsection{3.6.1}{How to do it…}

\begin{enumerate}
\item 
从 action.yml 文件中删除带有 bash 脚本的 run 步骤,并替换为 github-script 操作:

\begin{shell}
- name: Awesome github script action
  uses: actions/github-script@v6
  with:
    script: |
\end{shell}

\item 
使用工具包读取输入参数,向日志写入问候语,并设置输出参数:

\begin{shell}
var whoToGreet = core.getInput('who-to-greet')
core.notice(`Hello ${whoToGreet}`)
core.setOutput('answer', 42)
\end{shell}

\item 
现在,我们想要添加一些功能。如果工作流是由一个issue事件触发的,则想要在该issue添加一个评论。检查触发工作流程的事件实际上是issue,并且使用了 github.rest.issues 来创建一个评论:

\begin{tcolorbox}[ breakable,colback = bashcodebg, colframe= black!50!white]
\scriptsize{
if (context.eventName === 'issues') \{ \\
\hspace*{1em}github.rest.issues.createComment(\{ \\
\hspace*{2em}issue\_number: context.issue.number, \\
\hspace*{2em}owner: context.repo.owner, \\
\hspace*{2em}repo: context.repo.repo, \\
\hspace*{2em}body: '\emoji{👋} Thanks for reporting!' \\
\hspace*{1em}\}) \\
\}
}
\end{tcolorbox}

\item 
在 CI 工作流中,添加一个触发器,当打开一个新问题时执行工作流程:

\begin{shell}
on:
  issues:
    types: [opened]
\end{shell}

\item 
提交并推送更改。

\item 
在库的\textbf{Issues | New issue}(问题|新问题)(issues/new)下创建一个新issue,给它一个标题,然后保存它。工作流将运行,并在该issue上添加一个评论,如图 3.11 所示:

\myGraphic{0.4}{content/chapter3/images/11.png}{图3.11 --- 使用 github-script 操作对新issue添加评论}

github-script 操作是使用工具包的力量快速原型设计事物的方式,而且不需要创建单独操作。
\end{enumerate}

\mySubsubsection{3.6.2}{How it works…}

github-script 操作使得在工作流程或复合操作中快速编写使用 GitHub API 和工作流程运行上下文的脚本变得容易。它使用一个名为 script 的输入,其中包含异步函数调用的主体。该操作将提供以下参数:

\begin{itemize}
\item 
github: 一个预认证的 octokit/rest.js 客户端,带有分页插件

\item 
context: 一个包含工作流运行上下文的对象

\item 
core: 对 @actions/core 包的引用。

\item 
glob: 对 @actions/glob 包的引用。

\item 
io: 对 @actions/io 包的引用。

\item 
exec: 对 @actions/exec 包的引用。

\item 
fetch: 对 node-fetch 包的引用。

\item 
require: 一个围绕正常 Node.js require 的代理包装器,以启用相对于当前工作目录的相对路径,并要求 npm 包安装在当前工作目录中
\end{itemize}

由于脚本只是一个函数体,这些值将已经定义,因此不需要导入,并且可以直接像示例中使用 core、context 和 github 那样使用。

要了解更多关于 github-script 的信息,请访问 \url{https://github.com/actions/github-script}。

\mySubsubsection{3.6.3}{There’s more…}

在 YAML 文件中的编辑和调试体验并不好,可以像这样在操作中使用脚本文件:

\begin{shell}
with:
  script: |
    const script = require('./path/to/script.js')
    await script({github, context, core})
\end{shell}

这种方式,可以将 github-script 操作的简单性与更好的 JavaScript 编写体验结合起来。

github-script 操作是快速尝试某些事物,并创建集成概念验证的好方法。通过复合操作,可以逐渐将其放入易于共享的构建块中,也是打包可重用功能的简单方法。随着解决方案的发展,可能需要考虑将其移动到 TypeScript 或 Docker 容器操作中,以便更好地维护。











