在这个示例中,我们将向操作添加一个输出参数,该参数可以在后续步骤中使用,并且将向工作流作业摘要写入相应内容。

\mySubsubsection{3.3.1}{Getting ready}

您必须完成之前的示例才能继续这个示例。

\mySubsubsection{3.3.2}{How to do it…}

\begin{enumerate}
\item 
打开action.yml文件,并在inputs部分的下方(runs部分之前)添加以下代码:

\begin{shell}
outputs:
  answer:
    description: 'The answer to everything (always 42)'
\end{shell}

这定义了一个ID为answer的输出。

\item 
接下来,打开entrypoint.sh,并在文件末尾添加以下内容:

\begin{shell}
echo "answer=42" >> $GITHUB_OUTPUT
\end{shell}

这将把answer的输出值设置为42。

\item 
现在,在entrypoint.sh末尾添加以下内容,以向步骤摘要写入一些Markdown和HTML内容:

\begin{shell}
echo "### Hello $@! :rocket:" >> $GITHUB_STEP_SUMMARY
echo "<h3> The answer from Deep Thought is 42 :robot:</h3>" >>
$GITHUB_STEP_SUMMARY
\end{shell}

\item 
在提交更改之前,我们必须调整工作流文件 .github/workflows/ci.yml,以便它能够使用输出参数。为执行我们操作的步骤添加一个 ID,命名为 my-action,如下所示:

\begin{shell}
- name: Run my own container action
  id: my-action
  uses: ./
  with:
    who-to-greet: '@wulfland'
\end{shell}

添加另一个步骤,将结果输出到工作流日志:

\begin{shell}
- name: Output the answer
  run: echo "The answer is ${{ steps.my-action.outputs.answer }}"
\end{shell}

\item 
当Docker容器操作返回意外结果时,为了使CI构建失败,必须添加一个新的步骤,只有当结果不是预期时才执行。返回非零值(例如,exit 1)将向工作流程指示该步骤已失败。可以使用文件注释来指示错误的位置(就像我们在第2章中所做的那样):

\begin{shell}
- name: Test the container
  if: ${{ steps.my-action.outputs.answer != 42 }}
  run: |
    echo "::error file=entrypoint.sh,line=4,title=Error in container::The answer was not expected"
  exit 1
\end{shell}

\item 
提交并推送所有更改。构建将自动运行并应该成功。检查工作流日志,并确保输出参数已正确传递到下一步(见图3.3):

\myGraphic{0.6}{content/chapter3/images/3.png}{图3.3 --- Docker容器操作的输出}

同时,在工作流摘要页面上查看作业摘要,它已经渲染了Markdown/HTML(见图3.4):

\myGraphic{0.6}{content/chapter3/images/4.png}{图3.4 --- 工作流摘要页面上的作业摘要}

\item 
最后,我们希望确保当返回意外值时,我们的CI构建会失败。打开entrypoint.sh并将42更改为其他值(例如,7)。

切换到另一个分支,提交并推送,然后创建一个新的拉取请求:

\begin{shell}
$ git switch -c fail-ci-build
$ git commit -m "Fail CI build"
$ git push -u origin fail-ci-build
$ gh pr create --fill
\end{shell}

该拉取请求的检查将失败,并且它将注释相应的文件(见图3.5):

\myGraphic{0.6}{content/chapter3/images/5.png}{图3.5 --- 如果输出意外,CI构建拉取请求将失败}

请注意GitHub如何标记更改并,注释确切的行。

\end{enumerate}

\mySubsubsection{3.3.3}{How it works…}

让我们了解代码是如何工作的。

\mySamllsectionNoContent{环境文件}

将输出值传递给后续步骤和作业是通过将“名称-值”对管道传递到环境文件来实现的--- 即 \$GITHUB\_OUTPUT:

\begin{shell}
echo "{name}={value}" >> "$GITHUB_OUTPUT"
\end{shell}

>{}> 操作符将名称-值对追加到文件的末尾。文件的路径和名称存储在 \$GITHUB\_OUTPUT 环境变量中。可以通过步骤上下文(steps context)中步骤的 outputs 属性来访问该输出:

\begin{shell}
"${{ steps.<step-id>.outputs.<name> }}"
\end{shell}

输出是Unicode字符串,大小不能超过1 MB。一次工作流运行中所有输出的总大小,不能超过50 MB。

环境文件的另一个用例是,为作业中的后续步骤设置环境变量。相应环境文件的路径存储在 \$GITHUB\_ENV 中,只需将一个“名称-值”对追加到文件末尾,如下所示:

\begin{shell}
echo "{environment_variable_name}={value}" >> "$GITHUB_ENV"
\end{shell}

请注意,名称区分大小写!以下是一个完整的示例,展示了如何在一步中设置环境变量,并在后续步骤中使用 env 上下文进行访问:

\begin{shell}
steps:
  - name: Set the value
    id: step_one
    run: |
      echo "action_state=yellow" >> "$GITHUB_ENV"

  - run: |
      echo "${{ env.action_state }}" # This will output 'yellow'
\end{shell}

\mySamllsectionNoContent{作业摘要}

可以为工作流中的每个作业设置自定义 Markdown,渲染后的 Markdown 将在工作流程运行的摘要页面上显示。作业摘要可用于显示内容(例如:测试或代码覆盖率结果),以便查看工作流程运行结果时,无需进入日志或外部系统。作业摘要支持 GitHub 风格的 Markdown,但由于 Markdown 是 HTML,因此也可以将 HTML 输出到作业摘要文件中。这个示例中,我的 GitHub 用户名 @wulfland 是一个链接到我个人资料并带有预览的链接,并且所有 GitHub 表情符号都受支持。

通过将 Markdown 追加到以下文件,可以将步骤结果添加到作业摘要中:

\begin{shell}
echo "{markdown content}" >> $GITHUB_STEP_SUMMARY
\end{shell}

各个步骤是隔离的,并且限制为 1 MiB(1.04858 MB),以便单个步骤的畸形 Markdown 不会破坏后续步骤的 Markdown 渲染。只有 20 个步骤可以写入摘要,之后的步骤的输出都不可见。

在下一个示例中,我们将使用工具包向作业摘要写入更复杂的内容。

有关环境文件和作业摘要的完整参考,请参阅 \url{ https://docs.github.com/en/actions/using-workflows/workflow-commands-for-githubactions?tool=bash#environment-files}。

\mySamllsectionNoContent{表达式和条件执行}

在第 1 章和第 2 章中,使用了表达式(\$\{\{ ... \}\})从上下文对象中输出值。本示例中,我们使用了一个带有步骤的 if 属性的表达式来有条件地执行:

\begin{shell}
- name: Test the container
  if: ${{ steps.my-action.outputs.answer != 42 }}
\end{shell}

此步骤仅在步骤上下文中,需要有 ID 为 my-action 操作的输出答案不等于 42 时才会执行。

if 属性也存在于作业中,以有条件地执行这些作业。

当您在 if 属性中使用表达式时,可以省略 \$\{\{ \}\} 表达式语法,GitHub Actions 会自动将 if 条件评估为表达式。

对于条件执行,表达式必须返回 true 或 false。要编写表达式并将上下文与静态值进行比较,可以使用表 3.1 中提供的运算符:

% Please add the following required packages to your document preamble:
% \usepackage{longtable}
% Note: It may be necessary to compile the document several times to get a multi-page table to line up properly
\begin{longtable}[c]{|>{\centering\arraybackslash}p{3cm}|>{\centering\arraybackslash}p{12cm}|}
\hline
\textbf{运算符}              & \textbf{描述}                   \\ \hline
\endfirsthead
%
\endhead
%
( )                            & 逻辑分组                       \\ \hline
{[} {]}                        & 索引                                  \\ \hline
.                              & 属性解引用                  \\ \hline
!                              & 非                                    \\ \hline
\textless , \textless{}=       & 小于,小于或等于      \\ \hline
\textgreater , \textgreater{}= & 大于,大于或等于 \\ \hline
==                             & 等于                                  \\ \hline
!=                             & 不等于                              \\ \hline
\&\&                           & 与                                    \\ \hline
||                             & 或                                     \\ \hline
\end{longtable}

\begin{center}
表 3.1 --- 表达式运算符
\end{center}

GitHub 提供了一组内置函数,可以在表达式中使用,帮助搜索字符串、格式化输出或处理数组。请参阅表 3.2 以获取可用函数的列表:

% Please add the following required packages to your document preamble:
% \usepackage{longtable}
% Note: It may be necessary to compile the document several times to get a multi-page table to line up properly
\begin{longtable}[c]{|p{4cm}|p{11cm}|}
\hline
\textbf{函数} &
  \textbf{描述} \\ \hline
\endfirsthead
%
\endhead
%
contains(search, item) &
  \begin{tabular}[c]{@{}l@{}}如果 search 包含 item,则返回 true。 \\ \\ 例如:\\ contains('Hello world', 'llo') 返回 true\\ contains(github.event.issue.labels.*.name, 'bug') \\如果与事件相关的议题包含标签 bug,则返回 true。\end{tabular} \\ \hline
startsWith(search, iten) &
  当 search 以 item 开头时,返回 true。 \\ \hline
endsWith(search, item) &
  当 search 以 item 结尾时,返回 true。 \\ \hline
format(string, v0, v1, ...) &
  \begin{tabular}[c]{@{}l@{}}替换字符串中的值。 \\ \\ 例如:\\ format('Hello \{0\} \{1\} \{2\}', 'Mona', 'the', \\'Octocat') returns 'Hello Mona the Octocat'.\end{tabular} \\ \hline
join(array, optS) &
  数组中的所有值将连接成一个字符串。如果提供了可选的分隔符 optS,其将插入到连接的值之间。如果没有提供分隔符,则默认使用逗号作为分隔符。 \\ \hline
toJSON(value) &
  返回值的格式化 JSON 表示。 \\ \hline
fromJSON(value) &
  为 value 返回一个 JSON 对象或 JSON 数据类型。 \\ \hline
hashFiles(path) &
  为匹配路径模式的一组文件返回唯一的哈希值。\\ \hline
\end{longtable}

\begin{center}
表3.2 --- GitHub 中表达式的内置函数
\end{center}

\mySubsubsection{3.3.4}{There’s more…}

还有一些特殊函数用于检查当前作业的状态。下面的示例中,最后一步只有在作业的先前步骤失败时才会执行——返回一个非零值:

\begin{shell}
steps:
  - run: exit 1
  - name: The job has failed
    if: ${{ failure() }}
\end{shell}

有关用于检查作业状态的可用函数列表,请参阅表 3.3:

% Please add the following required packages to your document preamble:
% \usepackage{longtable}
% Note: It may be necessary to compile the document several times to get a multi-page table to line up properly
\begin{longtable}[c]{|>{\centering\arraybackslash}p{3cm}|>{\centering\arraybackslash}p{12cm}|}
\hline
\textbf{函数} & \textbf{描述}                                   \\ \hline
\endfirsthead
%
\endhead
%
success() & 如果之前的步骤都没有失败或取消,则返回 true。                               \\ \hline
always()  & 即使之前的某个步骤取消,也返回 true,并确保该步骤必定执行。 \\ \hline
cancelled()       & 仅当工作流取消时返回 true。        \\ \hline
failure()         & 如果作业的前一个步骤失败,则返回 true。 \\ \hline
\end{longtable}

这些函数还可以帮助有条件地执行某些步骤并执行(例如:清理操作)。要了解更多关于条件执行的表达式,请参阅:\url{ https://docs.github.com/en/actions/learn-github-actions/expressions and https:// ocs.github.com/en/actions/using-jobs/using-conditions-to-controljob-execution}。




