
本示例中,我们将设置 Azure 中的 Kubernetes 集群,并在 Azure 中配置 OIDC,以便在不使用存储秘钥的情况下部署到集群。

\mySubsubsection{7.3.1}{Getting ready}

确保至少有一个具有对软件包的读访问权限的 PAT 令牌。

如果熟悉 Azure 并且本地安装了 Azure CLI(\url{https://docs.microsoft.com/cli/azure/install-azure-cli?view=azure-cli-latest}),可以直接从那里开始操作。如果是 Azure 新手或者没有安装 CLI,只需使用Azure Cloud Shell(\url{https://shell.azure.com})即可。

将 PAT 令牌设置为环境变量:

\begin{shell}
$ export GHCR_PAT=<YOUR_PAT_TOKEN>
\end{shell}

该令牌将由 Kubernetes 用于从 GitHub Package Registry 读取。打开脚本 setup-azure.sh 并调整文件顶部的 location 变量,以选择想要的 Azure 区域。可以使用 az account list-locations -o table 获取区域列表。

提交并推送更改,然后运行脚本:

\begin{shell}
$ git clone https://github.com/{OWNER}/release-recipe.git
$ cd release-recipe
$ chmod +x setup-azure.sh
$ ./setup-azure.sh
\end{shell}

这将创建一个 Azure Kubernetes Service ,并将其与 GitHub Container registry 连接。在脚本运行时,可以配置 OIDC 以从工作流访问。

\mySubsubsection{7.3.2}{How to do it…}

\begin{enumerate}
\item 
使用 Cloud Shell 或本地终端,并创建一个新的应用程序注册:

\begin{shell}
$ az ad app create --display-name release-recipe
\end{shell}

\item 
然后,使用注册输出中的应用 ID 创建一个服务主体:

\begin{shell}
$ az ad sp create --id <appId>
\end{shell}

\item 
然后,打开 Azure 门户,在 Microsoft Entra 中,在\textbf{App registrations}(应用程序注册)下找到 release-recipe。在\textbf{Certificates \& secrets | Federated credentials | Add credentials}(证书和机密|联合凭证|添加凭证)下添加 OIDC 信任,填写表单,将组织设置为 GitHub 用户名,输入库名,并将实体类型选择为\textbf{Environment}(环境)(见图 7.1):

\myGraphic{0.6}{content/chapter7/images/1.png}{图7.1 --- 在Microsoft Entra中连接GitHub账户}

为这些凭证命名并点击\textbf{Add}(添加)。记下 release-recipe 应用程序的\textbf{Application (client) ID}(应用程序(客户端)ID)和\textbf{Directory (tenant) ID}(目录(租户)ID)(见图 7.2)。稍后会用到这些信息:

\myGraphic{0.6}{content/chapter7/images/2.png}{图7.2 --- 应用注册的客户端和租户 ID}

\item 
然后,在订阅中为服务主体分配一个角色,并在门户中打开订阅。在\textbf{Access control (IAM) | Role assignment | Add | Add role assignment}(访问控制 (IAM) | 角色分配 | 添加 | 添加角色分配)下,按照向导操作。选择角色——例如,\textbf{Contributor}(参与者)——再点击\textbf{Next}(下一步)。选择“ User, group, or service principal”,然后选择之前创建的服务主体。

\end{enumerate}

\mySubsubsection{7.3.3}{How it works…}

可以使用 OIDC(而不是使用存储为机密的凭据,来连接到云提供商,如 Azure、AWS、GCP 或 HashiCorp)。OIDC 将使用短期有效的令牌进行身份验证,而非凭据。云提供商也需要在其端支持 OIDC。

使用 OIDC 时,无需将云凭据存储在 GitHub 中,可以对工作流可以访问的资源拥有更细粒度的控制,并且拥有轮换的、短期有效的令牌,这些令牌在工作流运行后将过期。图 7.3 展示了 OIDC 的工作原理概述:

\myGraphic{0.6}{content/chapter7/images/3.png}{图7.3 --- OIDC 与云提供商的集成}

步骤如下:

\begin{itemize}
\item 
在云提供商和 GitHub 之间创建一个 OIDC 信任关系。将信任限制到一个组织和库,并进一步限制对环境、分支或拉取请求的访问。

\item 
GitHub OIDC 提供程序在工作流运行期间,自动生成一个 JSON Web Token。该令牌包含多个声明,用于为特定工作流作业建立安全且可验证的身份。

\item 
云提供商验证声明,并提供一个仅在作业生命周期内有效的短期访问令牌。

\item 
使用访问令牌访问该身份有权访问的资源。
\end{itemize}

可以使用该身份直接访问资源,或者使用它从安全库(例如 Azure Key Vault 或 HashiCorp Vault)中获取凭据。通过这种方式,可以使用库安全地连接到不支持 OIDC 和自动密钥轮换的服务。

在 GitHub 上,可以在 \url{https://docs.github.com/en/actions/deployment/security-hardening-your-deployments} 找到有关为 AWS、Azure 和 GCP 配置 OIDC 的说明。
