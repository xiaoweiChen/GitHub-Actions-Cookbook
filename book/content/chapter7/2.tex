在本示例中,我们将容器化一个简单的 Web 应用程序,并将其推送到容器注册表。

\mySubsubsection{7.2.1}{Getting ready}

打开库(\url{https://github.com/wulfland/release-recipe}),然后点击\textbf{Use
this template}(使用此模板)创建新的副本(直接链接:\url{https://github.com/new?template_name=release-recipe&template_owner=wulfland})。在个人账户中创建一个新的公共库,将其命名为 release-recipe,并克隆该库。

打开 package.json 并调整作者和库 URL。

在 dependencies 下,将 package recipe 的所有者和版本调整为来自第 6 章的软件包:

\begin{shell}
"dependencies": {
  "@wulfland/package-recipe": "^2.0.5",
  "express": "^4.18.2"
}
\end{shell}

在文件 .npmrc 中替换所有者:

\begin{shell}
@wulfland:registry=https://npm.pkg.github.com
\end{shell}

在终端中,运行以下命令:

\begin{shell}
$ npm login --registry https://npm.pkg.github.com
\end{shell}

输入 GitHub 用户名和具有访问软件包权限的 PAT 令牌。运行以下命令:

\begin{shell}
$ npm install
$ npm start

> release-recipe@1.0.0 start
> node src/index.js

Server running at http://localhost:3000
\end{shell}

现在,应该有一个在端口 3000 上运行的 Node.js 应用程序。打开浏览器,导航到 http://localhost:3000,并验证它是否显示 Hello World!。通过退出进程(CTRL+C)停止服务器。

\mySubsubsection{7.2.2}{How to do it…}

\begin{enumerate}
\item 
在库根目录创建一个新文件,Dockerfile。从 node 镜像继承,并选择 21-bullseye 版本。创建一个文件夹,将库内容复制到其中,并将其设置为工作文件夹:

\begin{shell}
FROM node:21-bullseye
RUN mkdir -p /app
COPY . /app
WORKDIR /app
\end{shell}

\item 
注意,需要在构建 Docker 镜像之前运行 npm install,以避免在容器中存储凭据。在容器中重建 npm 包:

\begin{shell}
RUN npm rebuild
\end{shell}

\item 
暴露 express 网站的 3000 端口,并将 npm start 作为容器的启动命令运行:

\begin{shell}
EXPOSE 3000
CMD [ "npm", "start"]
\end{shell}

\item 
接下来,本地构建容器镜像并运行:

\begin{shell}
$ npm install
$ docker build -t hello-world-recipe .
$ docker run -it -p 3000:3000 hello-world-recipe
\end{shell}

\item 
再次验证网站是否在本地机器的 3000 端口上运行。

\item 
创建一个新的工作流文件 .github/workflows/publish.yml。在拉取请求和推送到 main 分支时运行:

\begin{shell}
name: Publish Docker Image

on:
  push:
    branches: [ main ]
  pull_request:
    branches: [ main ]
\end{shell}

\item 
将注册表名称和镜像名称设置为环境变量:

\begin{shell}
env:
  REGISTRY: 'ghcr.io'
  IMAGE_NAME: '${{ github.repository }}'
\end{shell}

\item 
添加一个作业,为 GITHUB\_TOKEN 授予写入软件包和读取内容的权限:

\begin{shell}
jobs:
  build-and-push-image:
    runs-on: ubuntu-latest

    permissions:
      packages: write
      contents: read
\end{shell}

\item 
添加以下步骤。检出库并登录到 Docker 注册表:

\begin{shell}
- name: Checkout repository
  uses: actions/checkout@v4

- name: Log in to the Container registry
  uses: docker/login-action@v3
  with:
    registry: ${{ env.REGISTRY }}
    username: ${{ github.actor }}
    password: ${{ secrets.GITHUB_TOKEN }}
\end{shell}

\item 
从注册表中提取镜像的元数据,使用长 SHA 作为容器的标签。为该步骤指定一个 ID,以便稍后访问其输出:

\begin{shell}
- name: Extract metadata (tags, labels) for Docker
  id: meta
  uses: docker/metadata-action@v5
  with:
    images: ${{ env.REGISTRY }}/${{ env.IMAGE_NAME }}
    tags: |
      type=sha,format=long
\end{shell}

\item 
设置 Node.js 以使用正确的版本和注册表。然后,构建和测试代码。必须将 NODE\_AUTH\_TOKEN 环境变量设置为 GITHUB\_TOKEN,以验证到软件包注册表,并从第 6 章的软件包配方中接收 npm 软件包:

\begin{shell}
- uses: actions/setup-node@v4
  with:
    node-version: 21.x
    registry-url: https://npm.pkg.github.com/

- name: Build and test
  env:
    NODE_AUTH_TOKEN: ${{ secrets.GITHUB_TOKEN }}
  run: |
    npm install
    npm run test
\end{shell}

\item 
现在,我们准备好构建并推送 Docker 镜像。使用 meta 步骤的输出来设置标签和标签:

\begin{shell}
- name: Build and push Docker image
  uses: docker/build-push-action@v5
  with:
    context: .
    push: ${{ github.event_name != 'pull_request' }}
    tags: ${{ steps.meta.outputs.tags }}
    labels: ${{ steps.meta.outputs.labels }}
\end{shell}

\item 
提交并推送更改。工作流运行后,将在库的\textbf{Code}(代码)选项卡右侧的\textbf{Packages}(软件包)下找到 Docker 镜像。

\end{enumerate}

\mySubsubsection{7.2.3}{How it works…}

我使用Express(\url{https://expressjs.com/})作为简单的 Web 框架来运行网站。该网站显示软件包中的内容;将在后续示例中利用这一点来自动保持依赖项的更新。代码很容易理解:

\begin{shell}
const express = require('express');
const greet = require('@wulfland/package-recipe/src/index')
const app = express();
const port = 3000;

app.get('/', (req, res) => {
  res.send(greet());
});

app.listen(port, () => {
  console.log(`Server running at http://localhost:${port}`);
});
\end{shell}

这次,我省略了所有的代码检查和测试,已经在第 6 章中介绍过了。并且需要将此应用程序容器化,以便稍后将其部署到云端。

要将容器部署到云端,必须将其存储在容器注册表中。这里使用 GitHub packages,但使用特定于云的注册表方式相同。唯一区别是,需要配置一个 PAT 令牌,不能使用 GITHUB\_TOKEN。

\mySubsubsection{7.2.4}{There’s more…}

Docker meta-action(\url{https://github.com/docker/metadata-action})可以从 Git 引用和 GitHub 事件中提取元数据。就像第 6 章中的 GitVersion 一样,可以用于自动化您的容器版本控制,还支持语义化版本控制:

\begin{shell}
- name: Docker meta
  id: meta
  uses: docker/metadata-action@v5
  with:
    images: |
      ${{ env.REGISTRY }}/${{ env.IMAGE_NAME }}
    tags: |
      type=ref,event=branch
      type=ref,event=pr
      type=semver,pattern={{version}}
      type=semver,pattern={{major}}.{{minor}}
\end{shell}

我们的示例中,只使用 Git SHA 以便能够部署每个提交,但可以根据工作流轻松扩展版本控制。
