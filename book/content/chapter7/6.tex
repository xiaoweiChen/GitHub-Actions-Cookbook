
现在我们已经有了一个从包仓库到发布仓库的端到端工作流程,并从这里进入生产环境,我想向您展示如何结合使用Dependabot和GitHub Actions来自动化您的依赖项更新过程。

\mySubsubsection{7.6.1}{Getting ready}

在仓库中,导航到“设置”|“代码安全性和分析”,确保已启用“依赖关系图”(见图7.4):

\myGraphic{0.4}{content/chapter7/images/5.png}{图7.5  ---  启用依赖关系图和可选的Dependabot警报}

这将分析您的仓库并检测您可以在“洞察”|“依赖关系图”下检查的所有依赖项。您还可以启用Dependabot警报。在这种情况下,当您的某个依赖项存在已知漏洞时,Dependabot会通知您。Dependabot安全更新更进一步,Dependabot会为您生成一个包含更新到非漏洞版本的拉取请求。为了减少拉取请求的数量,您可以组合更新(此功能仍处于测试阶段)。

\mySubsubsection{7.6.2}{How to do it…}

\begin{enumerate}
\item 
创建一个新的名为PAT的dependabot密钥,并将其设置为具有包读取权限的PAT令牌的值:

\begin{shell}
$ gh secret set PAT --app dependabot
\end{shell}

\item 
创建一个新文件:.github/dependabot.yml。它总是以version:2开始:

\begin{shell}
version: 2
\end{shell}

\item 
使用PAT配置一个新的npm-registry,指向\url{https://npm.pkg.github.com}

\begin{shell}
registries:
  npm-pkg:
    type: npm-registry
    url: https://npm.pkg.github.com
    token: ${{ secrets.PAT }}
    replaces-base: true
\end{shell}

\item 
版本更新在updates下配置。添加生态系统npm,并指向您创建的注册表名称:

\begin{shell}
updates:
  - package-ecosystem: "npm"
    directory: "/"
    schedule:
      interval: "weekly"
    registries:
      - npm-pkg
\end{shell}

这将检查npm包的每周更新,包括私有注册表中的包。

\item 
可选地,为GitHub操作添加更新:

\begin{shell}
- package-ecosystem: "github-actions"
  directory: "/"
  schedule:
    interval: "weekly"
\end{shell}

这将检查GitHub操作是否有更新版本。

\item 
您还可以将Docker添加为生态系统:

\begin{shell}
- package-ecosystem: "docker"
  directory: "/"
  schedule:
    interval: "weekly"
\end{shell}

这也适用于Kubernetes清单文件,Dependabot将检查清单文件中镜像标签的更新。然而,在我们的情况下,我们使用环境变量直接部署新版本。

\item 
提交并推送文件。然后,转到Insights | Dependency graph | Dependabot。为每个配置的生态系统都有一个条目,您可以通过点击右侧的链接来检查日志文件(参见图7.6):

\myGraphic{0.4}{content/chapter7/images/6.png}{图7.6  ---  检查Dependabot版本更新的日志}

检查日志中是否有错误。

\item 
现在,转到包-配方仓库,并使用新的补丁版本创建一个新版本。一旦新包版本发布,转到Insights | Dependency graph| Dependabot,并点击package.json行中的链接。点击检查更新以强制Dependabot现在检查更新(参见图7.7):

\myGraphic{0.4}{content/chapter7/images/7.png}{图7.7  ---  让Dependabot现在检查更新}

\item 
Dependabot将创建一个新的拉取请求,以更新到新版本(参见图7.8):

\myGraphic{0.4}{content/chapter7/images/8.png}{图7.8  ---  Dependabot版本更新创建的拉取请求}

合并拉取请求,或者通过在拉取请求上评论 @dependabot squash and merge 让 Dependabot 来完成合并。

\item 
由于我们信任包的所有者,我们可以自动化这最后一步,并在所有检查都成功后,为特定版本合并每一个拉取请求。创建一个新文件:.github/workflows/dependencies.yml。工作流程将在 pull\_request\_target 上运行,并且它需要写入拉取请求的权限:

\begin{shell}
name: Dependabot auto-merge

on: [ pull_request_target ]

permissions:
  pull-requests: write
  contents: write
\end{shell}

\item 
只有在拉取请求的作者是 dependabot 时才运行作业:

\begin{shell}
jobs:
  dependabot:
    runs-on: ubuntu-latest
    if: ${{ github.actor == 'dependabot[bot]' }}
    steps:
\end{shell}
  
\item 
第一步是获取 Dependabot 元数据:

\begin{shell}
- name: Dependabot metadata
  id: metadata
  uses: dependabot/fetch-metadata@v1
  with:
    github-token: "${{ secrets.GITHUB_TOKEN }}"
\end{shell}

\item 
接下来,添加一个基于元数据的条件运行的步骤:如果依赖项名称包含您的包(将 {OWNER} 替换为您的用户名)并且版本更新是补丁版本。如果所有检查都成功,则使用 gh merge -{}-auto 合并拉取请求:

\begin{shell}
- name: Enable auto-merge for all patch versions
  if: ${{contains(steps.metadata.outputs.dependency-names, '@{OWNER}/package-recipe') && steps.metadata.outputs.update-type == 'version-update:semver-patch'}}
  run: gh pr merge --auto --merge "$PR_URL"
  env:
    PR_URL: ${{github.event.pull_request.html_url}}
    GITHUB_TOKEN: ${{secrets.GITHUB_TOKEN}}
\end{shell}

请注意,这个工作流程不会触发监听主分支推送事件的 publish.yml 工作流程。这是因为合并操作是使用 GITHUB\_TOKEN 执行的。您可以使用 PAT(个人访问令牌)令牌,就像最后一步中那样,或者您可以将发布逻辑移动到一个可重用的工作流程中,并直接从该工作流程中调用它。

\item 
现在,回到 package-recipe 仓库,并使用新的补丁版本创建一个新的发布。一旦新包版本发布,前往 Insights | Dependency graph | Dependabot,并点击 package.json 行中的链接。点击“检查更新”以强制 Dependabot 现在检查更新(见图7.7)。Dependabot 将创建一个拉取请求,触发工作流程,并且拉取请求将自动合并。
\end{enumerate}

\mySubsubsection{7.6.3}{How it works…}

Dependabot 可以帮助您以更少的努力保持依赖项的最新状态。您可以使用它来自动化更新过程,并跟上所有依赖项的最新发布。

有许多生态系统支持这一功能:

\begin{itemize}
\item 
Bundler

\item 
Cargo

\item 
Composer

\item 
开发容器(包括 GitHub Codespaces)

\item 
Docker

\item 
Hex

\item 
Elm-packages

\item 
Git 子模块

\item 
GitHub Actions

\item 
Go 模块

\item 
Maven 和 Gradle

\item 
npm

\item 
NuGet

\item 
pip、pipenv 和 pip-compile

\item 
pnpm

\item 
poetry

\item 
pub

\item 
Swift

\item 
Terraform

\item 
yarn
\end{itemize}

为了获得完整的列表,请参阅以下链接:\url{: https://docs.github.com/en/code-security/dependabot/dependabot-version-updates/about-dependabot-version-updates}。

Dependabot 将为每个依赖项的每个版本更新创建拉取请求。您可以在拉取请求上使用 @dependabot 命令与 Dependabot 交互,并告诉它某些事情,例如忽略版本或重新提交更改。

dependabot.yml 文件有很多选项。您可以指定允许的更新类型、自定义提交信息、将更新分组、忽略某些依赖项,并添加审查者、标签或分配人员。有关配置选项的完整列表,请参阅以下链接:\url{https://docs.github.com/en/code-security/dependabot/dependabotversion-updates/configuration-options-for-the-dependabot.yml-file}。

结合工作流程和 dependabot/fetch 元数据操作,这是一个非常强大的工具,可以跨多个存储库和团队自动化您的供应链。如果您查看第 6 章和第 7 章,我们会使用 Conventional Commits 与 GitVersion 来根据传统的提交信息完全自动化语义版本控制。然后我们可以利用 Dependabot 来完全自动化下游依赖项的更新。

\mySubsubsection{7.6.4}{There’s more…}

在我们的示例中,我们直接构建容器并将部署推送到 Kubernetes。然而,您也可以使用 Dependabot 来更新 Kubernetes 清单文件,通过为包含引用 Docker 镜像标签的清单的每个目录在您的 dependabot.yml 文件的 Docker 包生态系统元素中添加条目。Kubernetes 清单可以是正常的 Kubernetes 部署文件,也支持 Helm 图表。有关为 Kubernetes 配置您的 dependabot.yml 文件的更多信息,请参阅以下链接:\url{https://docs.github.com/en/code-security/dependabot/dependabot-version-updates/configurationoptions-for-the-dependabot.yml-file#docker}。