
现在,我们已经实现了从包库到发布库,再到生产环境的端到端工作流,我展示了如何使用 dependabot 结合 GitHub Actions 来自动化更新依赖项的过程。

\mySubsubsection{7.6.1}{Getting ready}

导航到\textbf{Settings | Code security and analysis}(设置 | 代码安全与分析),并确保\textbf{Dependency graph}(依赖项图)已启用(见图 7.4):

\myGraphic{0.6}{content/chapter7/images/5.png}{图7.5 --- 启用依赖项图和可选的 dependabot 警报}

这将分析库并检测所有依赖项,可以在\textbf{Insights | Dependency graph}(洞察|依赖项图)下检查这些依赖项,还可以启用 Dependabot 警报。当依赖项存在已知漏洞时,dependabot 会进行通知。Dependabot 安全更新更进一步,dependabot 会生成一个拉取请求,其中包含一个更新到非易受攻击版本的版本。为了减少拉取请求的数量,可以将更新分组在一起(此功能仍处于测试阶段)。

\mySubsubsection{7.6.2}{How to do it…}

\begin{enumerate}
\item 
创建一个名为 PAT 的新 dependabot 密钥,并将其设置为对软件包具有读取访问权限的 PAT 令牌:

\begin{shell}
$ gh secret set PAT --app dependabot
\end{shell}

\item 
创建一个新文件:.github/dependabot.yml。总是以 version:2 开头:

\begin{shell}
version: 2
\end{shell}

\item 
使用 PAT 配置一个新的 npm-registry,指向\url{https://npm.pkg.github.com}

\begin{shell}
registries:
  npm-pkg:
    type: npm-registry
    url: https://npm.pkg.github.com
    token: ${{ secrets.PAT }}
    replaces-base: true
\end{shell}

\item 
版本更新在 updates 下配置。添加生态系统 npm ,并指向创建的注册表的名称:

\begin{shell}
updates:
  - package-ecosystem: "npm"
    directory: "/"
    schedule:
      interval: "weekly"
    registries:
      - npm-pkg
\end{shell}

这将检查 npm 包的每周更新,包括私有注册表中的包。

\item 
可选地,添加 GitHub actions 的更新:

\begin{shell}
- package-ecosystem: "github-actions"
  directory: "/"
  schedule:
    interval: "weekly"
\end{shell}

这将检查 GitHub actions 的更新版本。

\item 
还可以将 Docker 添加为生态系统:

\begin{shell}
- package-ecosystem: "docker"
  directory: "/"
  schedule:
    interval: "weekly"
\end{shell}

这同样适用于 Kubernetes 清单文件,dependabot 将检查清单文件中镜像标签的更新。但在我们的示例中,需要使用环境变量直接部署新版本。

\item 
提交并推送该文件。然后,转到\textbf{Insights | Dependency graph | Dependabot}(洞察|依赖项图|Dependabot)。每个已配置的生态系统都有一个条目,可以通过点击右侧的链接来检查日志文件(见图 7.6):

\myGraphic{0.6}{content/chapter7/images/6.png}{图7.6 --- 检查 dependabot 版本更新的日志}

请检查日志中是否有错误。

\item 
现在,转到 package-recipe 库,并创建一个带有新补丁版本的新发布。当新包版本发布,转到\textbf{Insights | Dependency graph | Dependabot}(洞察|依赖项图|Dependabot)并点击 package.json 行中的链接。点击\textbf{Check for updates}(检查更新)以强制 dependabot 现在检查更新(见图 7.7):

\myGraphic{0.6}{content/chapter7/images/7.png}{图7.7 --- 让 dependabot 现在检查更新}

\item 
dependabot 将创建一个新的拉取请求,将版本更新到最新版本(见图 7.8):

\myGraphic{0.6}{content/chapter7/images/8.png}{图7.8 --- 由 dependabot 版本更新创建的拉取请求}

合并拉取请求或通过在拉取请求上评论 @dependabot squash and merge 让 dependabot 来完成。

\item 
由于我们信任包的所有者,可以自动化这一最后步骤,并在所有检查成功后仅合并特定版本的每个拉取请求。创建一个新文件:.github/workflows/dependencies.yml。该工作流将在 pull\_request\_target 上运行,并且需要对拉取请求进行写入权限:

\begin{shell}
name: Dependabot auto-merge

on: [ pull_request_target ]

permissions:
  pull-requests: write
  contents: write
\end{shell}

\item 
仅当拉取请求的作者为 dependabot 时才运行作业:

\begin{shell}
jobs:
  dependabot:
    runs-on: ubuntu-latest
    if: ${{ github.actor == 'dependabot[bot]' }}
    steps:
\end{shell}
  
\item 
第一步是获取 dependabot 元数据:

\begin{shell}
- name: Dependabot metadata
  id: metadata
  uses: dependabot/fetch-metadata@v1
  with:
    github-token: "${{ secrets.GITHUB_TOKEN }}"
\end{shell}

\item 
接下来,添加一个基于以下元数据条件的步骤:如果依赖项名称包含您的包(将 \{OWNER\} 替换为相应的用户名)并且版本更新是补丁版本。如果所有检查都成功,使用 gh merge -{}-auto 合并拉取请求:

\begin{shell}
- name: Enable auto-merge for all patch versions
  if: ${{contains(steps.metadata.outputs.dependency-names, '@{OWNER}/package-recipe') && steps.metadata.outputs.update-type == 'version-update:semver-patch'}}
  run: gh pr merge --auto --merge "$PR_URL"
  env:
    PR_URL: ${{github.event.pull_request.html_url}}
    GITHUB_TOKEN: ${{secrets.GITHUB_TOKEN}}
\end{shell}

请注意,此工作流不会触发监听 main 分支 push 触发的 publish.yml 工作流,因为合并是使用 GITHUB\_TOKEN 进行的。可以使用与上一步相同的个人访问令牌(PAT),或者可以将发布逻辑移动到一个可重用的工作流中,并直接从此工作流中进行调用。

\item 
现在,回到 package-recipe 库并创建一个具有新补丁版本的新发布。当新包版本发布,请转到\textbf{Insights | Dependency graph | Dependabot}(洞察|依赖项图|Dependabot),然后单击 package.json 行中的链接。点击\textbf{Check for updates}(检查更新)以强制 dependabot 现在检查更新(见图 7.7)。Dependabot 将创建一个拉取请求,触发工作流,并且该拉取请求将自动合并。
\end{enumerate}

\mySubsubsection{7.6.3}{How it works…}

Dependabot 可以以更少的精力使依赖项保持最新。可以使用它来自动化更新过程,并跟踪所有依赖项的最新发布。

它支持许多生态系统:

\begin{itemize}
\item 
Bundler

\item 
Cargo

\item 
Composer

\item 
开发容器(包括 GitHub Codespaces)

\item 
Docker

\item 
Hex

\item 
Elm-packages

\item 
Git 子模块

\item 
GitHub Actions

\item 
Go 模块

\item 
Maven 和 Gradle

\item 
npm

\item 
NuGet

\item 
pip、pipenv 和 pip-compile

\item 
pnpm

\item 
poetry

\item 
pub

\item 
Swift

\item 
Terraform

\item 
yarn
\end{itemize}

为了获得完整的列表,请参阅以下链接:\url{https://docs.github.com/en/code-security/dependabot/dependabot-version-updates/about-dependabot-version-updates}。

Dependabot将为每个依赖项的每个版本更新创建拉取请求,可以在拉取请求中使用@Dependabot命令与dependabot交互,并告诉它某些事情,例如:忽略版本或重新应用更改。

dependabot.yml文件有许多选项,可以指定允许哪些类型的更新,自定义提交消息,将更新分组在一起,忽略某些依赖项,以及添加审阅者、标签或分配人。有关配置选项的完整列表,请参阅以下链接:\url{https://docs.github.com/en/code-security/dependabot/dependabotversion-updates/configuration-options-for-the-dependabot.yml-file}。

结合工作流和dependabot/fetch metadata操作,这是一个非常强大的工具,可以跨许多库和团队自动化您的供应链。回看第6章和第7章,我们使用Conventional Commits和GitVersion根据约定的提交消息完全自动化语义化版本控制,可以利用dependabot完全自动化下游依赖项的更新。

\mySubsubsection{7.6.4}{There’s more…}

我们的示例中,直接构建容器并将部署推送到Kubernetes,也可以通过在dependabot.yml文件的Docker package-ecosystem元素中为包含引用Docker镜像标签的清单的每个目录添加一个条目,使用dependabot更新Kubernetes清单文件。Kubernetes清单可以是普通的Kubernetes部署文件,也支持Helm图表。有关为Kubernetes配置dependabot.yml文件的更多信息,请参阅以下链接:\url{https://docs.github.com/en/code-security/dependabot/dependabot-version-updates/configurationoptions-for-the-dependabot.yml-file#docker}。