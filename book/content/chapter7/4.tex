我们已经在第 5 章中使用了环境,所以这部分是重复的,所以这个示例会很简短。环境管理核心发布,接下来的示例将展示,如何使用它们在 Kubernetes 中显示我们服务的 URL。

\mySubsubsection{7.4.1}{Getting ready}

确保手头有来自前几章的\textbf{Application (client) ID}(应用程序(客户端)ID)、\textbf{Directory (tenant) ID}(目录(租户)ID) 和\textbf{Subscription ID}(订阅 ID)。可以通过以下方式获取订阅 ID:

\begin{shell}
$ az account show
\end{shell}

\mySubsubsection{7.4.2}{How to do it…}

\begin{enumerate}
\item 
在库设置中,转到\textbf{Environments}(环境),点击\textbf{New environment}(新建环境),并创建一个新环境:Production。

\item 
添加 main 作为部署分支。

\item 
添加一个名为 AZURE\_CLIENT\_ID 的新\textbf{Environment secret}(环境密钥),并将其设置为\textbf{Application (client) ID}(应用程序(客户端)ID)。

\item 
添加一个名为 AZURE\_TENANT\_ID 的新\textbf{Environment secret}(环境密钥),并将其设置为\textbf{Directory (tenant) ID}(目录(租户)ID)。

\item 
添加一个名为 AZURE\_SUBSCRIPTION\_ID 的新\textbf{Environment secret}(环境密钥),并将其设置为\textbf{Subscription ID}(订阅 ID)。

\item 
添加一个名为 AZURE\_CLUSTER\_NAME 的新\textbf{Environment secret}(环境密钥),并将其设置为集群名称(如果没有修改 setup-azure.sh 脚本,则为 AKSCluster)。

\item 
添加一个名为 AZURE\_RESOURCE\_GROUP 的新\textbf{Environment secret}(环境密钥),并将其设置为资源组名称(如果没有修改 setup-azure.sh 脚本,则为 AKSCluster)。
\end{enumerate}

我们将在下一个示例中使用这些环境密钥,以安全地将应用程序部署到云端的 Kubernetes。

\mySubsubsection{7.4.3}{How it works…}

环境为工作流中的作业添加了一层抽象,并且可以通过规则进行保护。有关审批检查的更多详情,请参阅第 5 章。环境也可以被 OIDC 实体信任,这就是我们将在下一个示例中使用的功能。