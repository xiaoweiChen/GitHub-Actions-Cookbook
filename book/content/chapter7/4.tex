我们已经在第5章《使用GitHub Actions在GitHub中自动化任务》中使用过环境,所以这是一个重复的内容。因此,我会保持这个指南相对简短。环境管理核心发布,我们将在接下来的指南中使用它们来显示我们在Kubernetes中的服务的URL。

\mySubsubsection{7.4.1}{Getting ready}

确保您手头有来自前面章节的应用(客户端)ID、目录(租户)ID和订阅ID。可以通过以下命令获取订阅ID:

\begin{shell}
$ az account show
\end{shell}

\mySubsubsection{7.4.2}{How to do it…}

\begin{enumerate}
\item 
在您的存储库设置中,转到“环境”,点击“新环境”,并创建一个名为“生产”的新环境。

\item 
将main作为部署分支添加。

\item 
添加一个名为AZURE\_CLIENT\_ID的新环境密钥,并将其设置为应用(客户端)ID。

\item 
添加一个名为AZURE\_TENANT\_ID的新环境密钥,并将其设置为目录(租户)ID。

\item 
添加一个名为AZURE\_SUBSCRIPTION\_ID的新环境密钥,并将其设置为订阅ID。

\item 
添加一个名为AZURE\_CLUSTER\_NAME的新环境密钥,并将其设置为集群的名称(如果您没有修改setup-azure.sh脚本,则为AKSCluster)。

\item 
添加一个名为AZURE\_RESOURCE\_GROUP的新环境密钥,并将其设置为资源组的名称(如果您没有修改setup-azure.sh脚本,则为AKSCluster)。
\end{enumerate}

我们将在下一个指南中使用这些环境密钥,以安全地部署到云中的Kubernetes。

\mySubsubsection{7.4.3}{How it works…}

环境在工作流程中的作业上添加了一层抽象,并且可以通过规则进行保护。有关批准检查的更多详细信息,请参阅第5章《使用GitHub Actions在GitHub中自动化任务》。环境还可以被OIDC实体信任,这就是我们将在下一个指南中使用的内容。