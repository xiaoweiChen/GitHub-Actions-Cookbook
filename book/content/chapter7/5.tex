
现在,是时候将我们的应用程序发布到 AKS 中的生产环境了。

\mySubsubsection{7.5.1}{Getting ready}

打开文件 .github/workflows/publish.yml。

\mySubsubsection{7.5.2}{How to do it…}

\begin{enumerate}
\item 
在工作流的顶部,添加两个更多的环境变量:

\begin{shell}
env:
  REGISTRY: 'ghcr.io'
  IMAGE_NAME: '${{ github.repository }}'
  APP_NAME: 'release-recipe-app'
  SERVICE_NAME: 'release-recipe-service'
\end{shell}

为前一个示例中的 build-and-push-image 作业添加一个输出,以便新作业能够访问镜像名称:

\begin{shell}
outputs:
  image_tag: ${{ fromJSON(steps.meta.outputs.json).tags[0] }}
\end{shell}

\item 
向工作流添加一个名为 production 的第二个作业,该作业仅在推送到 main 时运行,并与生产环境相关联。将环境的 URL 设置为稍后添加的一个步骤的输出。该作业需要权限 id-token: write 和 content: read 以便 OIDC 能够工作:

\begin{shell}
production:
  if: github.ref == 'refs/heads/main' && github.event_name == 'push'
  needs: build-and-push-image
  runs-on: ubuntu-latest
  permissions:
    id-token: write
    contents: read
  environment:
    name: Production
    url: ${{ steps.get-service-url.outputs.SERVICE_URL}}
\end{shell}

\item 
添加步骤以签出库并使用 OIDC 登录 Azure:

\begin{shell}
- name: Checkout
  uses: actions/checkout@v4

- name: 'Az CLI login'
  uses: azure/login@v1
  with:
    client-id: ${{ secrets.AZURE_CLIENT_ID }}
    tenant-id: ${{ secrets.AZURE_TENANT_ID }}
    subscription-id: ${{ secrets.AZURE_SUBSCRIPTION_ID }}
\end{shell}

\item 
设置 Kubernetes 部署的上下文:

\begin{shell}
- name: 'Az CLI set AKS context'
  uses: azure/aks-set-context@v3
  with:
    cluster-name: ${{ secrets.AZURE_CLUSTER_NAME }}
    resource-group: ${{ secrets.AZURE_RESOURCE_GROUP }}
\end{shell}

\item 
检查文件 service.yml,其部署了一个负载均衡器,将应用程序的 3000 端口映射在 80 端口上。此外,检查 deployment.yml,其中包含应用程序的定义。要替换文件中的环境变量,可使用 envsubst。然后,将结果传递给 kubectl 并应用清单文件:

\begin{shell}
- name: Deploy
  env:
    IMAGE: ${{ needs.build-and-push-image.outputs.image_tag }}
  run: |-
    envsubst < service.yml | kubectl apply -f -
    envsubst < deployment.yml | kubectl apply -f -
\end{shell}

\item 
使用 kubectl describe service 获取 Kubernetes 中的服务 URL,并将其设置为步骤输出,以便在环境 URL 中使用:

\begin{shell}
- name: 'Get Service URL'
  id: get-service-url
  run: |
    IP=$(kubectl describe service $SERVICE_NAME| grep "LoadBalancer Ingress: " | awk '{print $3}')
  echo "SERVICE_URL=http://$IP" >> $GITHUB_OUTPUT
\end{shell}

\item 
最后,我们想要检查部署是否成功。如果应用程序有一个 /health 端点,需要查询该端点,但由于应用程序非常简单,只依赖返回的状态码即可:

\begin{shell}
- name: 'Run smoke test'
  env:
    SERVICE_URL: ${{ steps.get-service-url.outputs.SERVICE_URL}}
  run: |
   status=`curl -s --head $SERVICE_URL | head -1 | cut -f 2 -d' '`
  if [ "$status" != "200" ]
  then
    echo "Wrong HTTP Status. Actual: '$status'"
    exit 1
  fi
\end{shell}

这段代码的作用是使用 curl -{}-head 查询网站的头部信息,-s 选项会抑制其他输出。然后,使用 head -1 获取第一行。该行看起来像 HTTP/1.1 200 OK。我们使用空格分割字符串并取第二个元素(状态码),如果状态码不是 200(OK),则会引发异常。

\item 
提交并推送更改到 main 分支。这将触发工作流;这将把容器的新版本推送到注册表,并从那里在 AKS 中发布它。按照服务的 URL(见图 7.4),并验证网站是否正确显示:

\myGraphic{0.6}{content/chapter7/images/4.png}{图7.4 --- 部署到动态环境}

\item 
将显示来自容器的 Hello World 应用程序。
\end{enumerate}

\mySubsubsection{7.5.3}{How it works…}

Azure 登录操作将使用 OIDC 身份验证 Azure,可以将此身份授予对 Azure 资源的细粒度访问权限。由于我们限制了库中对生产环境的访问,因此只有此作业可以使用应用程序来验证 Azure,也可以使用分支、标签或拉取请求进行实现。

然后使用 aks-set-context 操作来配置使用 kubectl 访问 AKS,可将实际应用程序部署到 Kubernetes。

\mySubsubsection{7.5.4}{There’s more…}

Kubernetes 可以非常快地变得非常复杂。实际上,可向集群添加 DNS 和 SSL,并使用命名空间来管理并行运行的多容器。这是一种很棒的方式,可以将每个拉取请求部署到动态环境,但这些超出了本书的范围。

如果想使用其他云提供商(而不是 Azure)来发布容器,可在这里找到有关如何部署到 AWS Elastic Container Service (ECS) 的动手指南:\url{https://github.com/wulfland/AccelerateDevOps/blob/main/ch9_release/Deploy_to_AWS_ECS.md},或者在这里找到另一个有关如何部署到 Google Kubernetes Engine (GKE) 的指南:\url{https://github.com/wulfland/AccelerateDevOps/blob/main/ch9_release/Deploy_to_GKE.md}。