
现在,是时候将我们的应用程序发布到AKS的生产环境了。

\mySubsubsection{7.5.1}{Getting ready}

打开文件 .github/workflows/publish.yml。

\mySubsubsection{7.5.2}{How to do it…}

\begin{enumerate}
\item 
在 workflow 的顶部添加两个环境变量:

\begin{shell}
env:
  REGISTRY: 'ghcr.io'
  IMAGE_NAME: '${{ github.repository }}'
  APP_NAME: 'release-recipe-app'
  SERVICE_NAME: 'release-recipe-service'
\end{shell}

将上一个指南中的 build-and-push-image 作业的输出添加到新作业中,以便新作业能够访问镜像名称:

\begin{shell}
outputs:
  image_tag: ${{ fromJSON(steps.meta.outputs.json).tags[0] }}
\end{shell}

\item 
在 workflow 中添加一个名为 production 的第二个作业,该作业仅在推送到 main 分支时运行,并与生产环境相关联。将环境的 URL 设置为我们稍后将添加的步骤的输出。该作业将需要 id-token: write 和 contents: read 的权限才能使 OIDC 工作:

\begin{shell}
production:
  if: github.ref == 'refs/heads/main' && github.event_name == 'push'
  needs: build-and-push-image
  runs-on: ubuntu-latest
  permissions:
    id-token: write
    contents: read
  environment:
    name: Production
    url: ${{ steps.get-service-url.outputs.SERVICE_URL}}
\end{shell}

\item 
添加步骤以检出仓库并使用 OIDC 登录 Azure:

\begin{shell}
- name: Checkout
  uses: actions/checkout@v4

- name: 'Az CLI login'
  uses: azure/login@v1
  with:
    client-id: ${{ secrets.AZURE_CLIENT_ID }}
    tenant-id: ${{ secrets.AZURE_TENANT_ID }}
    subscription-id: ${{ secrets.AZURE_SUBSCRIPTION_ID }}
\end{shell}

\item 
设置 Kubernetes 部署的上下文:

\begin{shell}
- name: 'Az CLI set AKS context'
  uses: azure/aks-set-context@v3
  with:
    cluster-name: ${{ secrets.AZURE_CLUSTER_NAME }}
    resource-group: ${{ secrets.AZURE_RESOURCE_GROUP }}
\end{shell}

\item 
检查文件 service.yml。它部署了一个 LoadBalancer,将我们应用程序的 3000 端口显示在 80 端口上。此外,检查 deployment.yml,其中包含我们应用程序的定义。为了替换文件中的环境变量,我们使用 envsubst。然后我们将结果管道传输到 kubectl 并应用清单文件:

\begin{shell}
- name: Deploy
  env:
    IMAGE: ${{ needs.build-and-push-image.outputs.image_tag }}
  run: |-
    envsubst < service.yml | kubectl apply -f -
    envsubst < deployment.yml | kubectl apply -f -
\end{shell}

\item 
使用 kubectl describe service 获取 Kubernetes 中服务的 URL,并将其设置为步骤输出,以便在生产环境 URL 中使用:

\begin{shell}
- name: 'Get Service URL'
  id: get-service-url
  run: |
    IP=$(kubectl describe service $SERVICE_NAME| grep "LoadBalancer Ingress: " | awk '{print $3}')
  echo "SERVICE_URL=http://$IP" >> $GITHUB_OUTPUT
\end{shell}

\item 
最后,我们想要检查部署是否成功。如果应用程序有一个 /health 端点,那么我们将查询它,但因为我们的应用程序非常简单,我们将仅依赖于返回的状态码:

\begin{shell}
- name: 'Run smoke test'
  env:
    SERVICE_URL: ${{ steps.get-service-url.outputs.SERVICE_URL}}
  run: |
   status=`curl -s --head $SERVICE_URL | head -1 | cut -f 2 -d' '`
  if [ "$status" != "200" ]
  then
    echo "Wrong HTTP Status. Actual: '$status'"
    exit 1
  fi
\end{shell}

这段代码的作用是使用curl --head查询网站的头信息。-s选项抑制了其他输出。然后使用head -1获取第一行。该行看起来像HTTP/1.1 200 OK。我们通过空格切割字符串并取第二个元素(状态码)。如果状态码不是200(OK),则会抛出异常。

\item 
将您的更改提交并推送到主分支。这将触发工作流程;这将把新版本的容器推送到注册表,并从那里在AKS中发布。跟随服务的URL(见图7.4)并验证网站是否正确显示:

\myGraphic{0.4}{content/chapter7/images/4.png}{图7.4  ---  部署到动态环境}

\item 
它将显示您容器中的Hello World应用程序。
\end{enumerate}

\mySubsubsection{7.5.3}{How it works…}

Azure登录操作将使用OIDC身份验证到Azure。然后可以授予这个身份对Azure资源进行细粒度访问。由于我们限制了仓库中对生产环境的访问,只有这个作业可以使用应用程序对Azure进行身份验证。您也可以使用分支、标签或拉取请求来实现这一点。

然后我们使用aks-set-context操作配置使用kubectl对AKS的访问。这将允许我们部署实际应用程序到Kubernetes。

\mySubsubsection{7.5.4}{There’s more…}

Kubernetes可能很快就会变得非常复杂。在现实世界中,您将向集群添加DNS和SSL,并使用命名空间来管理并行运行多个容器。这是一种将每个拉取请求部署到动态环境的不错方法。然而,这超出了本书的范围。

如果您想使用除Azure以外的其他云提供商来发布容器,您可以在这里找到一个部署到AWS弹性容器服务(ECS)的指南:\url{https://github.com/wulfland/AccelerateDevOps/blob/main/ch9_release/Deploy_to_AWS_ECS.md},或者另一个部署到Google Kubernetes Engine(GKE)的指南:\url{https://github.com/wulfland/AccelerateDevOps/blob/main/ch9_release/Deploy_to_GKE.md}。