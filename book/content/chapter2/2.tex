
Visual Studio Code (VS Code) 是世界上最受欢迎和广泛使用的代码编辑器之一。由于其灵活性、广泛的扩展生态系统和强大的社区支持,它在开发者社区中获得了显著的人气。

VS Code 与 GitHub 高度集成,提供了诸如 Git 集成、使用 GitHub 账户同步设置、直接访问仓库,以及使用 GitHub 提供的扩展在编辑器内创建、编辑和管理 GitHub Action 工作流程等功能。这种紧密的集成,简化了工作流程的创建过程,还简化了在 GitHub Action 工作流程上的协作。

这个示例中,我们将安装 GitHub Actions 的 VS Code 扩展,并展示可以用它做什么。

\mySubsubsection{2.2.1}{Getting ready}

开始之前,请检查电子邮件地址和名称是否正确设置在git中:

\begin{shell}
$ git config --global user.email
$ git config --global user.name
\end{shell}

请记住,我们在公共库中工作。如果想要保持电子邮件地址为隐私,请使用来自 \url{https://github.com/settings/emails} 的邮件地址(见图 2.1):

\myGraphic{0.6}{content/chapter2/images/1.png}{图2.1 --- 公共库中保持电子邮件地址为隐私}

这个电子邮件地址由你的 GitHub 用户 ID 和名称组成,位于 users.noreply.github.com 域中:

\begin{shell}
$ git config --global user.email 5276337+wulfland@users.noreply.github.com
\end{shell}

GitHub 将自动将提交与账户关联,而不会暴露电子邮件地址。

然后,从第 1 章中本地克隆库,以在仓库的\textbf{Code}(代码)部分的\textbf{Code | Local}(代码 | 本地)下找到相应的 URL(见图 2.2):

\myGraphic{0.6}{content/chapter2/images/2.png}{图2.2 --- 本地克隆库}

这里使用 SSH 进行身份验证,这样就可以在 1Password 中管理 SSH 密钥——但也可以使用 HTTPS 和一个 PAT 令牌。可以在这里找到关于本地克隆库的更多信息:\url{https://docs.github.com/en/repositories/creating-and-managing-repositories/cloning-a-repository}。

\mySubsubsection{2.2.2}{How to do it…}

\begin{enumerate}
\item 
打开 VS Code,通过组合键 [Shift]+[Command]+[X] 或 [Ctrl]+[Shift]+[X] 或只需单击左侧栏中的扩展图标来打开扩展窗口(见图 2.3)。搜索 github actions 并安装带有验证徽章的 GitHub 的操作(\url{https://marketplace.visualstudio.com/items?itemName=GitHub.vscodegithub-actions})。如有必要,重启 VS Code,并使用 GitHub 账户登录:

\myGraphic{0.6}{content/chapter2/images/3.png}{图2.3 --- 为 VS Code 安装 GitHub Actions 扩展}

该扩展提供了以下功能:

\begin{itemize}
\item 
管理工作流和监控工作流运行

\item 
手动触发工作流

\item 
工作流和表达式的语法高亮

\item 
集成文档

\item 
验证和代码补全

\item 
智能验证
\end{itemize}

智能验证尤其有帮助,支持对引用的操作和可重用工作流程进行代码补全,将解析引用操作的参数、输入和输出,并提供验证、代码补全和内联文档功能。

\item 
打开本地克隆库。可以在本地克隆的仓库文件夹的命令行中输入 code .(. 代表当前文件夹)。这将打开一个新的 VS Code 实例并打开当前文件夹。或者在 VS Code 中使用\textbf{File | Open Folder}(文件 | 打开文件夹),并选择克隆库的所在文件夹。

\item 
单击左侧的 GitHub Actions 图标(见图 2.4),并检查\textbf{Current Branch}(当前分支)窗口。可以看到当前分支中的所有工作流运行,每次运行都有由工作流名称和哈希值组成 ID。还可以展开工作流运行状态,以查看作业和步骤。可以直接在 VS Code 中打开作业日志,或者可以在浏览器中打开步骤日志:

\myGraphic{0.6}{content/chapter2/images/4.png}{图2.4 --- 检查当前分支中的工作流运行}

\item 
WORKFLOWS 窗口显示了主分支中仓库的所有工作流程,可以将工作流程固定在列表顶部,打开工作流程文件进行编辑,或者如果工作流具有 workflow\_dispatch 触发器,则可以运行工作流的触发器(见图 2.5):

\myGraphic{0.6}{content/chapter2/images/5.png}{图2.5 --- 工作流窗口允许触发和打开工作流}

打开工作流文件。

\item 
接下来,在工作流程文件中尝试使用自动补全功能。删除第 15 行中 vars 后面的值。请注意,扩展程序知晓所有的变量和秘钥(见图 2.6)

\myGraphic{0.6}{content/chapter2/images/6.png}{图2.6 --- 自动补全所有已知的秘钥和变量}

还具有完整的 workflow 语法,可以在工作流文件中按组合键 [Control]+[Space] 来获取有效元素的列表。

VS Code 会检测到你工作流文件中的问题,并为其提供快速修复(见图 2.7):

\myGraphic{0.6}{content/chapter2/images/7.png}{图2.7 --- VS Code 为问题提供快速修复}

\item 
在SETTINGS窗口中,可以找到库中所有的环境、密钥和变量。但不能创建新的环境,可以为环境或库级别添加密钥和变量,以及编辑或删除它们(见图 2.8):

\myGraphic{0.6}{content/chapter2/images/8.png}{图2.8 --- 在 VS Code 中管理环境、密钥和变量}

\end{enumerate}

\mySubsubsection{2.2.3}{How it works…}

对于 GitHub Actions 扩展,VS Code 是编写和执行工作流的完美编辑器,可以在一个地方完成所有操作,也可以离线工作,并且具有高级语法高亮和针对工作流和表达式的自动补全功能。这就是为什么我们将在本书的其余部分中,使用VS Code来编写工作流。

\mySubsubsection{2.2.4}{There’s more…}

VS Code 不仅仅可以本地安装——也可以直接在浏览器中使用。在 GitHub 库中,只需按下点号 [.] 键,即可直接在浏览器中用 VS Code 打开当前库,或按下 [Shift]+[>] 在新标签页中打开。要直接在 VS Code 中打开仓库,也可以导航到https://github.dev/<owner>/<repository>。

在 GitHub.dev 中,可以像在本地一样处理文件和 Git:首先,提交更改,然后推送到 GitHub。可以安装扩展,并且可以使用 GitHub 账户同步 VS Code 设置。

然而,如果需要终端或安装一些框架,必须在本机进行操作或使用 GitHub Codespaces (\url{https://github.com/features/codespaces})。Codespaces 提供了在 Microsoft Azure 上运行的全功能远程开发环境。每月有 120 小时和 15 GB 的免费存储空间(GitHub Pro 计划则有 180 分钟和 20 GB);之后,将按分钟和 GB 付费。每 GB 每月成本为 0.07 美元,机器的计算费用每小时在 0.18 美元(双核机器)到 2.88 美元(32 核机器)之间。对于本书来说,我选择了本地版本,以避免用完免费时长的限制,但如果各位读者从未尝试过 Codespaces,我鼓励各位进行尝试,这是为每个项目配备专业开发环境的绝佳方式。