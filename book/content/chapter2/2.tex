
从Greenfield存储库开始,最好在主分支上创建工作流。但是,如果您必须在开发人员正在工作的活动存储库中创建工作流程,并且您不想妨碍他们的工作流,那么可以在分支中编写工作流并将其合并回主分支。

但是,某些触发因素可能无法正常工作。如果您想使用WorkFlow\_DISPATCH触发器手动运行工作流,则必须将第一个动作与触发器合并回MAIN或使用API触发工作流程。之后,您可以在分支中创建工作流程,并在通过UI触发工作流程时选择分支。

如果您的工作流程需要Webhook触发器,例如push,pull\_request或plul\_request\_target,则可能有必要在存储库的叉子中创建工作流,这取决于您打算使用触发器进行的工作。这样, 您可以测试和调试工作流程而不会干扰开发人员工作,并且完成后,您可以将其合并回原始存储库。

\mySubsubsection{2.2.1}{Getting ready}

如果您在上一个食谱中使用工作流程后仍然进行了本地更改,请务必撤消所有更改以具有清洁版本的存储库。您可以执行以下命令来做到这一点:

\begin{shell}
$ git reset --hard HEAD
\end{shell}

您也可以在VS代码的GIT窗口中执行此操作(请参见图2.9):

\myGraphic{0.4}{content/chapter2/images/9.png}{图2.9 --- 丢弃您的本地更改}

\mySubsubsection{2.2.2}{How to do it…}

\begin{enumerate}
\item 
在VS代码中,单击左下角的MAIN,选择+在命令调色板中创建新分支,输入新workfl ow作为新分支的名称,然后点击[Enter](见图2.10):

\myGraphic{0.4}{content/chapter2/images/10.png}{图2.10 --- 在VS代码中创建一个新分支}

另外,您也可以使用以下命令:

\begin{shell}
$ git switch -c new-workflow
\end{shell}

\item 
在Explorer窗口中的VS代码中创建一个新的WorkFlow文件。找到并标记.github/WorkFl ow文件夹,然后单击新文件⋯图标。输入Developinbranch。 YML作为文件名,然后单击Enter(请参见图2.11):

\myGraphic{0.4}{content/chapter2/images/11.png}{图2.11 --- 在VS代码中创建一个新的工作流文件}

请注意,VS代码会自动检测到这是一个工作流文件。使用pull\_request和workflow\_dis patch触发器创建一个简单的工作流,该触发器将某些上下文值输出到控制台,如列表2.1:

\filename{清单2.1 - 在分支中创建的工作流程}

\begin{shell}
# workflow to show how to develop workflows in branches
name: Develop in a branch

on: [pull_request, workflow_dispatch]
jobs:
  job1:
    runs-on: ubuntu-latest
    steps:
      - run: |
          echo "Workflow triggered in branch '${{ github.ref }}'."
          Echo "Workflow triggered by event '${{ github.event_name }}'."
          Echo "Workflow triggered by actor '${{ github.actor}}''."
\end{shell}

\item 
添加新文件(阶段更改),输入提交消息并提交更改(请参见图2.12):

\myGraphic{0.4}{content/chapter2/images/12.png}{图2.12 --- 在VS Code中提交新文件}

如果愿意,也可以使用命令行:

\begin{shell}
$ git add .
$ git commit -m "Added a workflow file in local branch"
\end{shell}

\item 
在VS代码中,您可以通过单击发布分支直接推动更改。从命令行,您可以使用以下内容:

\begin{shell}
$ git push -u origin new-workflow
\end{shell}

\item 
接下来,我们将为我们的新分支机构创建拉动请求。当我们使用pull\_request触发器时,这将自动运行我们的新工作流程。转到您的浏览器中的存储库,然后导航以拉出请求。git将检测到您已经推出了一个新分支,并将为您提供创建拉动请求的选项(比较和拉请请求,请参见图2.13):

\myGraphic{0.4}{content/chapter2/images/13.png}{图2.13 --- 在浏览器中创建一个新的pull请求}

只需要保留默认的标题(你之前添加的提交信息),然后点击Create pull request(见图2.14):

\myGraphic{0.4}{content/chapter2/images/14.png}{图2.14 --- 创建带有标题和描述的拉取请求}

您还可以使用GitHub CLI创建拉动请求:

\begin{shell}
$ gh pr create --fill
\end{shell}

\begin{myTip}{GitHub CLI}
在整本书中,我们将使用github cli(\url{https://cli.github.com/})。它适用于所有平台和许多包装管理器(Homebrew,Winget,RPM等)。有关更多安装说明,请参见\url{https://github.com/cli/cli/cli#installation}。安装后,您必须使用gh auth login登录(请参阅\url{https://cli.github.com/manual/gh_auth_login})进行身份验证。
\end{myTip}

\item 
打开您的拉请请求,并注意它已自动执行工作流程(请参见图2.15):

\myGraphic{0.4}{content/chapter2/images/15.png}{图2.15 --- 由于pull\_request触发器,工作流将自动运行}

您还可以在“操作”选项卡中查看工作流,但是您无法手动运行工作流程。 即使使用WorkFlow\_DisPatch触发器,运行工作流的按钮也只有在将工作流与该触发器合并为MAIN时可用。之后,您还可以在分支上手动运行它。

但是,您可以在VS代码中运行工作流程。在必要时使用右上角的刷新图标,打开GitHub操作扩展,并刷新工作流窗口。 现在,您应该看到新的工作流程,并且可以使用箭头按钮手动触发它(请参见图2.16):

\myGraphic{0.4}{content/chapter2/images/16.png}{图2.16 --- 在VS Code中手动运行工作流}
\end{enumerate}

\mySubsubsection{2.2.3}{How it works…}

在某些触发因素方面存在一些限制,但是通常,在单独的分支中开发工作流并使用拉力请求进行更改的协作非常有效。

\mySubsubsection{2.2.4}{There’s more…}

为了更进一步,我们将在工作流程中添加一个衬里,以便在存储库中的所有工作流程中发现错误,安全问题以及缺少最佳实践。