
基于现有结果文件的问题匹配者也可以通过使用WorkFlow命令将单个警告或错误事件编写日志中来实现。在此食谱中,我们将在工作流程中添加一些输出并注释工作流文件。

\mySubsubsection{2.4.1}{Getting ready}

确保您仍然从上一个食谱打开的拉动请求。只需使用VS代码添加其他更改,推动将自动触工作流程。

\mySubsubsection{2.4.2}{How to do it…}

\begin{enumerate}
\item 
打开 github/workflows/developinbranch.yml在vs代码中的新工作流分支中,并在结帐操之前直接添加以下代码:

\begin{shell}
- run: |
  echo "::debug::This is a debug message."
  echo "::notice::This is a notice message."
  echo "::warning::This is a warning message."
  echo "::error::This is an error message."
\end{shell}

这将为工作流日志和工作流程摘要写不同类型的消息。

\item 
提交并推动更改。这将自动触发新的工作流程。

\item 
打开工作流日志并检查输出。它应该看起来像图2.19:

\myGraphic{0.4}{content/chapter2/images/19.png}{图2.19  ---  将消息写入工作流日志}

请注意,调试消息不可见。检查工作流程摘要,并包含在那里的消息以及我们的覆盖的错误(结果应如图2.20):

\myGraphic{0.4}{content/chapter2/images/20.png}{图2.20  ---  摘要中的工作流注释}

\item 
要查看在行动中的调试消息,我们可以通过启用调试日志记录来重新运行工作流程。在工作流摘要中,单击重Re-run jobs | Re-run all jobs,选择“Enable debug logging”,然后击 Re-run jobs (如图2.21所示):

\myGraphic{0.4}{content/chapter2/images/21.png}{图2.21  ---  启用了带有调试记录的重播作业}

\item 
再次检查工作流日志,并查看其他消息(如图2.22所示):

\myGraphic{0.4}{content/chapter2/images/22.png}{图2.22  ---  带有调试登录的工作流日志}

\item 
但是,注意,警告和错误不仅可以写入日志。我们可以使用它们来注释文件。将以下段添加到工作流程:

\begin{shell}
- run: |
  echo "::notice file=.github/workflows/DevelopInBranch.yml,line=19,col=11,endColumn=51::There is a debug message that is not always visible!"
  echo "::warning file=.github/workflows/DevelopInBranch.yml,line=19,endline=21::A lot of messages"
  echo "::error title=Script Injection,file=.github/workflows/DevelopInBranch.yml,line=13,col=37,endColumn=68::Potentialscript injection"
\end{shell}

这将为第19行添加通知注释,对第19至21行的警告,以及第37至68列第13行的错误。如果您的行号和凹痕不同,请调整值!

\item 
提交并推动更改。打开拉动请求,并查看“文件更改”选项卡中的注释(请参见图2.23 ):

\myGraphic{0.4}{content/chapter2/images/23.png}{图2.23  ---  拉动请求中的工作流注释更改}
\end{enumerate}

\mySubsubsection{2.4.3}{How it works…}

就像匹配器的工作方式一样,您可以创建警告和错误消息并将其打印到日志中。消息将创注释,该注释可以将消息与存储库中的特定文件相关联。可选,您的消息可以指定文件的位置:

\begin{shell}
::notice file={name},line={line},endLine={el},title={title}::{message}
::warning
file={name},line={line},endLine={el},title={title}::{message}
::error file={name},line={line},endLine={el},title={title}::{message}
\end{shell}

参数如下:

\begin{itemize}
\item 
Title:消息的自定义标题

\item 
File:提出错误或警告的文件名

\item 
Col:列/角色编号,从1开始

\item 
EndColumn:末端列号

\item 
Line:文件中的行号以1开始

\item 
EndLine:终点号码
\end{itemize}

唯一无法注释文件的消息是调试消息。此工作流命令仅接受该消息作为参数。
