
在这个示例中,我们将添加一个代码检查操作,将检查工作流,并直接在拉取请求中给出反馈。

\mySubsubsection{2.4.1}{Getting ready}

打开开放的拉取请求中拥有的分支上的工作流程文件,先不要合并更改。

\mySubsubsection{2.4.2}{How to do it…}

\begin{enumerate}
\item 
转到市场并搜索 actionlint,找的操作来自 devops-actions。该操作需要访问工作流文件,所以必须使用checkout操作检出库。在作业的末尾添加以下两个步骤:

\begin{shell}
- uses: actions/checkout@v4.1.0
- uses: devops-actions/actionlint@v0.1.2
\end{shell}

\item 
由于我们希望该操作在拉取请求中注释错误,必须给工作流写入拉取请求的权限,稍后再对此进行解释。现在,只需向作业添加一个 permissions :

\begin{shell}
jobs:
  job1:
    runs-on: ubuntu-latest
    permissions:
      contents: read
      pull-requests: write
\end{shell}

\item 
将更改提交到 new-workflow 分支并推送到远程。这将触发构建,工作流就应该能够无错误地完成。

\item 
为了测试代码检查过程,需要添加一些恶意代码。如果在运行事件中使用用户控制的输入,例如:拉取请求的标题,攻击者可能会通过在标题中注入脚本来利用这一点。如果一个公共库,人们可以从分支创建拉取请求,这会变得相当危险。假设使用 echo 输出拉取请求的标题:

\begin{shell}
  - run: echo "${{ github.event.pull_request.title }}"
\end{shell}

创建一个标题为 \verb|"Hi";ls $GITHUB_WORKSPACE;echo "-" | 的拉取请求将执行以下命令:

\begin{shell}
$ echo "Hi";ls $GITHUB_WORKSPACE;echo "-"
\end{shell}

脚本\verb|ls $ GITHUB_WORKSPACE|将在无错误的情况下执行,还可以找到其他方法来注入更有害的内容。

在作业的步骤中,添加以下行:

\begin{shell}
steps:
  - run: echo "PR title is '${{ github.event.pull_request.title
}}'.
\end{shell}

\item 
将更改提交到 new-workflow 分支并推送到远程。这次,工作流程应该会失败,拉取请求的检查会显示一个错误,如图 2.17 所示:

\myGraphic{0.6}{content/chapter2/images/17.png}{图2.17 --- 如果工作流中存在潜在的脚本攻击,代码检查可能会使拉取请求失败}

\item 
在拉取请求中,导航到\textbf{Files changes}(文件更改)。请注意,代码检查操作已在正确的行号上对拉取请求进行了注释,该行包含潜在的脚本注入攻击(见图 2.18):

\myGraphic{0.6}{content/chapter2/images/18.png}{图2.18 --- 工作流文件中的拉取请求注释}
\end{enumerate}

\mySubsubsection{2.4.3}{How it works…}

devops-actions/actionlint (\url{https://github.com/marketplace/actions/rhysd-actionlint}) 操作是一个封装器,用于 @rhysd (这是 GitHub 用户) 的 actionlint 容器 (见 \url{https://github.com/rhysd/actionlint}),可以在本地或网络上运行 actionlint。其对工作流进行大量的检查 (见 \url{https://github.com/rhysd/actionlint/blob/main/docs/checks.md} 获取完整列表),该操作是一个封装器,会在库中找到的所有工作流程上运行 actionlint。因此,必须先检出库。然后,该操作使用问题匹配器 (见 \url{https://github.com/actions/toolkit/blob/main/docs/problem-matchers.md}) 来在工作流程文件中注释拉取请求。问题匹配器可以使用正则表达式模式,从结果文件中读取发现的问题,并用它们注释拉取请求。通过工作流命令 add-matcher 和 remove-matcher 激活和停用,工作流命令可以在工作流程步骤和操作中使用,用于与工作流程和运行器机器通信。也可以用于向工作流程日志写入消息,将值传递给其他步骤或操作,设置环境变量,或写入调试消息。

工作流程命令使用具有特定格式的 echo 命令:

\begin{shell}
  echo "::workflow-command param1={data},param2={data}::{command value}"
\end{shell}

如果使用 JavaScript,工具包 (\url{https://github.com/actions/toolkit}) 提供了许多封装器,可以用来替代 echo 写入标准输出。在后续部分中,我将介绍一些有用的写入工作流程日志和注释文件的工作流程命令示例。

问题匹配器通过以下工作流程命令添加:

\begin{shell}
  echo "::add-matcher::$GITHUB_ACTION_PATH/actionlint-matcher.json"
\end{shell}

该命令采用结果文件的路径,可以使用 remove-matcher 并传入所有者来禁用匹配:

\begin{shell}
  echo "::remove-matcher owner= actionlint::"
\end{shell}

为了访问拉取请求,工作流使用 GITHUB\_TOKEN,这是一个由 GitHub 自动创建的特殊令牌,可通过 github 上下文 (github.token) 或 secrets 上下文 (secrets.GITHUB\_TOKEN) 访问。即使工作流没有将其作为输入或环境变量提供,GitHub Action 也可以访问该令牌。该令牌可用于在访问 GitHub 资源时对工作流进行身份验证,默认权限可以设置为宽松 (读写) 或受限 (只读),但可以在工作流程中调整权限。可以在工作流程日志的\textbf{Set up job | GITHUB\_
TOKEN Permissions}(设置作业 | GITHUB\_TOKEN 权限)下查看工作流权限。请记住,最佳实践是始终是显式设置工作流所需的权限。

所有其他权限将自动设置为无。权限可以针对单个作业或整个工作流程设置。

我们示例的情况为,给出了工作流作业读取内容和写入拉取请求的权限:

\begin{shell}
permissions:
  contents: read
  pull-requests: write
\end{shell}

\mySubsubsection{2.4.4}{There's more…}

本示例,我们通过简单地添加 pull\_request 触发器到工作流程,将工作流程代码检查作为拉取请求的检查。然而,即使检查失败,仍然能够将更改合并回主分支。为了防止具有代码检查错误的工作流程被合并,可以启用分支保护 (\url{https://docs.github.com/en/repositories/configuring-branches-and-merges-in-your-repository/managing-protected-branches/about-protected-branches}) 或创建规则集 (\url{https://docs.github.com/en/repositories/configuring-branches-and-merges-in-your-repository/managing-rulesets/about-rulesets})。与 codeowners (\url{https://docs.github.com/en/repositories/managing-your-repositorys-settings-and-features/customizing-your-repository/about-code-owners}) 结合使用,可以确保只有没有代码检查错误,并且需要在团队或个人对工作流审查后,才能合并回主分支。












