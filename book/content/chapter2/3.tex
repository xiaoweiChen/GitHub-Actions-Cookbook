
全新的库中,最好在主分支上创建工作流。如果必须在开发人员正在工作的活跃库中创建工作流,并且不想影响到其他人的工作,那么可以在一个分支中编写工作流程,并通过拉取请求将其合并回主分支。

然而,某些触发器可能不会按预期工作。如果想要使用 workflow\_dispatch 触发器手动运行工作流,第一个操作必须是合并带有触发器的工作流程回主分支,或者使用 API 来触发工作流程。之后,可以在一个分支中编写工作流程,并在通过 UI 触发工作流程时选择该分支。

如果工作流需要 webhook 触发器,如 push、pull\_request 或 pull\_request\_target,根据计划如何使用这些触发器,可能需要在库的分支中创建工作流程。这样,可以在不干扰开发人员工作的前提下测试和调试工作流程,完成时可以将其合并回原始库。

\mySubsubsection{2.3.1}{Getting ready}

如果尝试前一个示例中的工作流后仍有本地更改,请撤销所有更改,使库保持一个干净的状态。可以通过执行以下命令来实现:

\begin{shell}
$ git reset --hard HEAD
\end{shell}

也可以在 VS Code 的 Git 窗口中这样做(见图 2.9):

\myGraphic{0.7}{content/chapter2/images/9.png}{图2.9 --- 丢弃本地更改}

\mySubsubsection{2.3.2}{How to do it…}

\begin{enumerate}
\item 
在 VS Code 中,点击左下角的 main,选择命令面板中的 \textbf{+ Create new branch… },输入 new-workflow 作为新分支的名称,然后按 [Enter](见图 2.10):

\myGraphic{0.6}{content/chapter2/images/10.png}{图2.10 --- 在 VS Code 中创建一个新分支}

或者,可以使用以下命令:

\begin{shell}
$ git switch -c new-workflow
\end{shell}

\item 
在 VS Code 的 资源管理器 窗口中创建一个新的工作流文件。找到并标记 .github/workflow 文件夹,然后点击 \textbf{New file…}(新建文件…) 图标。输入 DevelopInBranch.yml 作为文件名,然后点击回车(见图 2.11):

\myGraphic{0.6}{content/chapter2/images/11.png}{图2.11 --- 在 VS Code 中创建一个新的工作流文件}

请注意,VS Code 会自动检测这是一个工作流文件。创建具有 pull\_request 和 workflow\_dispatch 触发器的工作流,该触发器将一些上下文值输出到控制台,如清单 2.1 所示:

\filename{清单2.1 - 在分支中创建的工作流}

\begin{shell}
# workflow to show how to develop workflows in branches
name: Develop in a branch

on: [pull_request, workflow_dispatch]
jobs:
  job1:
    runs-on: ubuntu-latest
    steps:
      - run: |
          echo "Workflow triggered in branch '${{ github.ref }}'."
          echo "Workflow triggered by event '${{ github.event_name }}'."
          echo "Workflow triggered by actor '${{ github.actor}}''."
\end{shell}

\item 
添加新文件(暂存更改),输入提交信息,并提交更改(见图 2.12):

\myGraphic{0.6}{content/chapter2/images/12.png}{图2.12 --- 在 VS Code 中提交新文件}

也可以使用命令行:

\begin{shell}
$ git add .
$ git commit -m "Added a workflow file in local branch"
\end{shell}

\item 
在 VS Code 中,可以通过点击 \textbf{Publish Branch}(发布分支) 直接推送更改。使用命令行,可以这样:

\begin{shell}
$ git push -u origin new-workflow
\end{shell}

\item 
接下来,我们将为新分支创建一个拉取请求。因为使用了 pull\_request 触发器,这将自动运行新工作流。在浏览器中转到库并导航到 \textbf{Pull requests}(拉取请求)。Git 检测到已经推送了一个新分支,并会提供给创建拉取请求的选项(比较 \& 拉取请求,见图 2.13):

\myGraphic{0.6}{content/chapter2/images/13.png}{图2.13 --- 浏览器中创建一个新的拉取请求}

只需保留默认标题(你之前添加的提交信息)然后点击 、\textbf{Create pull request}(创建拉取请求)(见图 2.14):

\myGraphic{0.6}{content/chapter2/images/14.png}{图2.14 --- 创建一个带有标题和描述的拉取请求}

也可以使用 GitHub CLI 创建拉取请求:

\begin{shell}
$ gh pr create --fill
\end{shell}

\begin{myTip}{GitHub CLI}
在本书中,将大量使用 GitHub CLI (\url{https://cli.github.com/}),其适用于所有平台,可以通过许多包管理器(Homebrew、WinGet、RPM 等)获取。有关更多安装说明,请参阅 \url{https://github.com/cli/cli#installation}。安装后,需要使用 \verb|gh auth login| 进行身份验证(请参阅 \url{https://cli.github.com/manual/gh_auth_login})。
\end{myTip}

\item 
打开你的拉取请求,已经自动执行了工作流程作为检查(见图 2.15):

\myGraphic{0.6}{content/chapter2/images/15.png}{图2.15 --- 由于 pull\_request 触发器,工作流将自动运行}

你还可以在\textbf{Actions}(操作)选项卡中看到工作流,但不能手动运行工作流。即使有 workflow\_dispatch 触发器,只有该触发器的工作流与 main 合并后,运行工作流程的按钮才可用。之后,也将能够在分支上手动运行。

不过,可以在 VS Code 中运行工作流程。打开 \textbf{GitHub Actions} 扩展,请使用右上角的刷新图标刷新\textbf{Workflows}(工作流)窗口。现在应该看到新的工作流,并且可以使用箭头按钮手动触发(见图 2.16):

\myGraphic{0.6}{content/chapter2/images/16.png}{图2.16 --- 在 VS Code 中手动运行工作流}
\end{enumerate}

\mySubsubsection{2.3.3}{How it works…}

当涉及到某些触发器时,存在一些限制。但总的来说,在单独的分支中开发工作流,并使用拉取请求协作更改,这种方法非常有效。

\mySubsubsection{2.3.4}{There’s more…}

为了进一步发展,我们将向工作流程添加一个代码检查工具,能够在库中的所有工作流程中找出错误、安全问题,以及缺陷。
