
之前的示例中,可以使用启用了调试日志的选项重新运行失败的任务或所有任务,也可以在库级别启用或禁用调试日志。

\mySubsubsection{2.6.1}{How to do it…}

我们可以向库添加一个名为 ACTIONS\_STEP\_DEBUG 的变量,并将其值设置为 true 或 false 来启用或禁用调试日志。这将向工作流日志添加非常详细的输出,以及所有调试消息,并且这些内容将从所有操作中显示出来。

可以使用网页、GitHub CLI 或 VS Code 配置该变量。要使用网页设置变量,请转到 \textbf{Settings | Secrets and variables | Actions}(设置 | 秘钥和变量 | 操作) 并选择 \textbf{Variables}(变量) 选项卡 (/settings/variables/actions)。单击 \textbf{New repository variable}(新库变量) (这将重定向到 /settings/variables/actions/new),输入 ACTIONS\_STEP\_DEBUG 作为名称,true 作为值,然后单击 \textbf{Add variable}(添加变量)。

要使用 CLI 设置,只需执行以下命令:

\begin{shell}
$ gh variable set ACTIONS_STEP_DEBUG --body true
\end{shell}

如果想在 VS Code 中设置该变量,只需打开 Actions 扩展,导航到 \textbf{Variables | Repository Variables}(变量 |库变量),单击 + 符号(见图 2.24),输入 ACTIONS\_STEP\_DEBUG,然后按 [Enter];输入 true,然后再次按 [Enter]。在 VS Code 中,使用更新选项更改变量也非常方便:

\myGraphic{0.6}{content/chapter2/images/24.png}{图2.24 ---  在 VS Code 中将步骤调试设置为 true}

运行工作流,并检查详细的输出。

\mySubsubsection{2.6.2}{There’s more…}

也可以将变量 ACTIONS\_RUNNER\_DEBUG 设置为 true 来激活运行器的附加日志。运行器调试日志将包含在工作流程的日志存档中,可以从工作流程作业日志中下载该存档。如果想了解更多关于监控和故障排除的信息,可以参考\url{https://docs.github.com/en/actions/monitoring-andtroubleshooting-workflows/enabling-debug-logging}。