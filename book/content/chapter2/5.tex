
基于现有结果文件,问题匹配器所能做到的,也可以通过使用工作流命令将单个警告或错误事件写入日志来实现。在这个示例中,我们将向工作流程添加一些输出,并对工作流文件进行注释。

\mySubsubsection{2.5.1}{Getting ready}

确保仍然打开着上一个示例中的拉取请求。只需使用 VS Code 添加其他更改,推送将自动触发工作流程。

\mySubsubsection{2.5.2}{How to do it…}

\begin{enumerate}
\item 
在 VS Code 中的 new-workflow 分支中打开 .github/workflows/DevelopInBranch.yml,并在检出操作之前直接添加以下代码:

\begin{shell}
- run: |
  echo "::debug::This is a debug message."
  echo "::notice::This is a notice message."
  echo "::warning::This is a warning message."
  echo "::error::This is an error message."
\end{shell}

这将会将不同类型的消息写入工作流日志和工作流摘要。

\item 
提交并推送更改。这将自动触发一个新的工作流运行。

\item 
打开工作流日志并检查输出。它应该看起来像图 2.19:

\myGraphic{0.6}{content/chapter2/images/19.png}{图2.19 --- 将消息写入工作流日志}

请注意,调试消息是不可见的。检查工作流摘要,看看它是否包含了我们的消息,以及来自代码检查的错误(结果应该看起来像图 2.20):

\myGraphic{0.6}{content/chapter2/images/20.png}{图2.20 --- 摘要中的工作流注释}

\item 
为了看到调试消息的作用,可以使用启用调试日志重新运行工作流作业。在工作流摘要中,点击 \textbf{Re-run jobs | Re-run all jobs}(重新运行作业 | 重新运行所有作业),选择 \textbf{ Enable debug logging}(启用调试日志),然后点击 \textbf{Re-run jobs}(重新运行作业)(如图 2.21 所示)

\myGraphic{0.6}{content/chapter2/images/21.png}{图2.21 --- 启用调试日志记录后重新运行作业}

\item 
再次检查工作流程日志,看看其他的消息(如图 2.22 所示):

\myGraphic{0.6}{content/chapter2/images/22.png}{图2.22 --- 启用调试日志的工作流程日志}

\item 
但是,请注意,notice、warning 和 error 不只是写入日志。我们可以使用它们来注释文件。将以下代码添加到工作流中:

\begin{shell}
- run: |
  echo "::notice file=.github/workflows/DevelopInBranch.yml,line=19,col=11,endColumn=51::There is a debug message that is not always visible!"
  echo "::warning file=.github/workflows/DevelopInBranch.yml,line=19,endline=21::A lot of messages"
  echo "::error title=Script Injection,file=.github/workflows/DevelopInBranch.yml,line=13,col=37,endColumn=68::Potentialscript injection"
\end{shell}

这将在第 19 行添加一个 notice 注释,在第 19 到 21 行添加一个警告,并在第 13 行的 37 到 68 列添加一个错误。如果行号和缩进不同,请调整这些值!

\item 
提交并推送更改。打开拉取请求,并在 \textbf{Files changed }(文件更改) 选项卡中查看注释(见图 2.23):

\myGraphic{0.6}{content/chapter2/images/23.png}{图2.23 --- 拉取请求更改中的工作流注释}
\end{enumerate}

\mySubsubsection{2.5.3}{How it works…}

与匹配器的工作方式相同,可以创建警告和错误消息,并输出到日志中。这些消息将创建一个注释,该注释可以将消息与库中的特定文件关联起来。可选地,消息也可以指定文件中的位置:

\begin{shell}
::notice file={name},line={line},endLine={el},title={title}::{message}
::warning
file={name},line={line},endLine={el},title={title}::{message}
::error file={name},line={line},endLine={el},title={title}::{message}
\end{shell}

参数如下:

\begin{itemize}
\item 
标题 (Title): 消息的自定义标题

\item 
文件 (File): 引发错误或警告的文件名

\item 
列 (Col): 列/字符编号,从 1 开始

\item 
结束列 (EndColumn): 结束列编号

\item 
行 (Line): 文件中的行号,从 1 开始

\item 
结束行 (EndLine): 结束行号
\end{itemize}

唯一不能注释文件的命令是调试消息,此工作流命令仅接受消息作为参数。
