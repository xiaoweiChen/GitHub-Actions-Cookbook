
本示例中,我们将使用测试结果的详细信息来装饰拉取请求和工作流摘要,还将向 README 文件添加徽章,以指示分支或发布的质量。

\mySubsubsection{6.4.1}{Getting ready}

创建一个新分支来进行修改:

\begin{shell}
$ git switch -c add-badges
\end{shell}

\mySubsubsection{6.4.2}{How to do it…}

可以使用以下 URL 为工作流下载徽章:

\begin{shell}
https://github.com/OWNER/REPO/actions/workflows/FILE.yml/badge.svg
\end{shell}

还可以通过添加查询参数(例如,?branch=main 或 ?event=push)按分支或事件进行过滤。我们想要一个适用于 main 分支的徽章,请将以下图像添加到您的 README 的 Markdown 中:

\begin{shell}
![main](https://github.com/OWNER/package-recipe/actions/workflows/ i.yml/badge.svg?branch=main)
\end{shell}

徽章将使用工作流文件中的名称,并在预览中看起来像图 6.4:

\myGraphic{0.2}{content/chapter6/images/4.png}{图6.4 --- CI 工作流的徽章}

但我们想更进一步,向 README 添加一个代码覆盖率徽章,用代码覆盖率输出来注释拉取请求,并向工作流添加摘要。

\begin{enumerate}
\item 
要装饰拉取请求,工作流需要写入权限。由于使用矩阵构建多个版本,必须选择一个用于徽章创建的版本。将以下代码添加到构建作业中:

\begin{shell}
jobs:
  build:
    permissions:
      pull-requests: write
    env:
      MAIN_VERSION: "21.x"
\end{shell}

\item 
在工作流文件中,在 run: npm test 之后添加以下代码:

\begin{shell}
- name: Prepare coverage report in markdown
  uses: fingerprintjs/action-coverage-report-md@v1.0.6
  id: coverage
  with:
    textReportPath: coverage/coverage.txt"
\end{shell}

使用 fingerprintjs/action-coverage-report-md 操作创建一个 Markdown 格式的报告,把该报告写入作业摘要和拉取请求。

\item 
接下来,使用 marocchino/sticky-pull-request-comment 操作将 Markdown 报告写入拉取请求:

\begin{shell}
- name: Add coverage comment to the PR
  uses: marocchino/sticky-pull-request-comment@v2.8.0
  with:
    message: ${{ steps.coverage.outputs.markdownReport }}
\end{shell}

\item 
同时,将输出写入 \$GITHUB\_STEP\_SUMMARY。由于在矩阵中运行作业,请添加一个标头以指示版本号:

\begin{shell}
- name: Add coverage report to the job summary
  run: |
    echo "## Code Coverage v${{ matrix.node-version }}" >> "$GITHUB_STEP_SUMMARY"
    echo "${{ steps.coverage.outputs.markdownReport }}" >> "$GITHUB_STEP_SUMMARY"
\end{shell}

\item 
提交并推送您的更改,然后创建一个拉取请求:

\begin{shell}
$ git add .
$ git commit
$ git push -u origin add-badges
$ gh pr create –fill
\end{shell}

这将使用拉取请求触发器触发工作流,可以在工作流运行中检查作业摘要,应该看起来像图 6.5:

\myGraphic{0.6}{content/chapter6/images/5.png}{图6.5 --- 工作流摘要中显示代码覆盖率}

摘要也添加为拉取请求的注释(见图 6.6):

\myGraphic{0.6}{content/chapter6/images/6.png}{图6.6 --- 将代码覆盖率结果添加为拉取请求注释}

\item 
代码覆盖率徽章是在测试期间由 npm 包自动创建的。但为了添加徽章(即添加到 README),需要一个托管它的地方,为此使用 \textbf{GitHub Pages}。

将以下代码添加到构建作业的末尾,以将工件上传到工作流。请注意,这不是普通的 actions/upload-artifact 操作——是一个用于 GitHub Pages 的特殊操作:

\begin{shell}
- name: Upload page artifacts
  if: ${{ matrix.node-version == env.MAIN_VERSION }}
  uses: actions/upload-pages-artifact@v3
  with:
    path: coverage
\end{shell}

由于我们只能上传一个同名工件,因此只在矩阵的 node 版本是主版本时才运行此步骤。

\item 
在构建之后添加一个新的部署作业,该作业仅在推送到 main 时运行,并且依赖于构建作业:

\begin{shell}
deploy:
  if: ${{ github.ref == 'refs/heads/main' }}
  needs: build
  runs-on: ubuntu-latest
\end{shell}

\item 
创建一个名为 pages 的并发组,以便一次只将一个版本部署到 pages 环境。但不要取消正在进行的部署以部署所有版本:

\begin{shell}
concurrency:
  group: "pages"
  cancel-in-progress: false
\end{shell}

\item 
该作业需要以下权限:

\begin{shell}
permissions:
  contents: read
  pages: write
  id-token: write
\end{shell}

\item 
该作业部署到 github-pages 环境,并使用 actions/deploy-pages 操作的 URL 在工作流中显示它。该 URL 将指向包的根目录。由于我们的 index.html 文件所在的 HTML 报告在 lcov-report 文件夹中,需要将其添加到 URL 中:

\begin{shell}
environment:
  name: github-pages
  url: "${{ steps.deployment.outputs.page_url }}lcov-report"
\end{shell}

\item 
该作业只有两个简单的步骤—— configure-pages 和 deploy-pages:

\begin{shell}
steps:
  - name: Setup Pages
    uses: actions/configure-pages@v4
  - name: Deploy to GitHub Pages
    id: deployment
    uses: actions/deploy-pages@v4
\end{shell}

\item 
提交并推送更改,在工作流完成后合并拉取请求:

\begin{shell}
$ git add .
$ git commit
$ git push
$ gh merge -m
\end{shell}

\item 
拉取请求合并后,一个新的工作流运行将由推送到 main 触发。完成后,可以看到包含报告的网站 URL(见图 6.7):

\myGraphic{0.6}{content/chapter6/images/7.png}{图6.7 --- 检查 Pages 网站}

\item 
GitHub 创建的网站在目录根目录中包含徽章。URL 看起来像这样:https://\{OWNER\}.github.io/package-recipe/badge.svg。Markdown 中的徽章 URL 将看起来像这样:

\begin{shell}
[![Coverage](https://wulfland.github.io/package-recipe/badge.svg)]
\end{shell}

我们希望点击徽章时可定向到网站中的 lcov-report 文件夹,需要为图像添加一个链接。将以下 Markdown 添加到 README(将 wulfland 替换为相应的 GitHub 用户名):

\begin{shell}
[![Coverage](https://wulfland.github.io/package-recipe/badge.svg)] (https://wulfland.github.io/package-recipe/lcov-report)
\end{shell}

徽章将如图 6.8 所示,当点击它时,将重定向到覆盖率报告:

\myGraphic{0.2}{content/chapter6/images/8.png}{图6.8 --- 添加测试覆盖率徽章}
  
\end{enumerate}

\mySubsubsection{6.4.3}{How it works…}

来了解一下代码是如何工作的。

\mySamllsectionNoContent{添加工作流状态徽章}

可以在库中显示工作流的状态徽章,以指示工作流的状况。

\begin{shell}
https://github.com/OWNER/REPOSITORY/actions/workflows/WORKFLOW-FILE/badge.svg
\end{shell}

通常,可显示特定分支或事件的状态,可以通过提供相应的参数(例如 ?branch=main 或 ?event=push)来实现。这样,就可以为软件的不同版本显示多个徽章。有关更多信息,请参阅 \url{ https://docs.github.com/en/actions/monitoring-and-troubleshooting-workflows/adding-aworkflow-status-badge}。

\mySamllsectionNoContent{GitHub Pages}

GitHub Pages 是 GitHub 提供的一个静态网站托管服务,从库中获取静态文件并将其发布为网站。该网站托管在 GitHub 的 github.io 域名上,或者可以使用自己的自定义域名。

除非使用自定义域名,否则项目网站可通过以下 URL 访问:

\begin{itemize}
\item 
http(s)://<username>.github.io/<repository> 或

\item 
http(s)://<organization>.github.io/<repository>
\end{itemize}

页面可以直接从分支部署,并且可以选择使用 Jekyll(\url{https://github.com/jekyll/jekyll}) 进行预处理。这样,可以轻松地将 Markdown 文件渲染为网站——例如,用于托管博客。

可以在 \url{https://wulfland.github.io/AccelerateDevOps/} 找到一个示例,其渲染了 \url{https://github.com/wulfland/AccelerateDevOps/tree/main/docs} 文件夹的内容。

除了从分支部署外,还可以使用自己的工作流来部署页面,这就是我们在本示例中所做的。在撰写本文时,该功能仍处于测试阶段——但在我看来,它已经可以在生产环境中使用了。

要了解更多关于 GitHub Pages 的信息,请访问:\url{https://docs.github.com/en/pages/getting-started-with-github-pages/about-github-pages}。

\mySamllsectionNoContent{并发组}

默认情况下,GitHub Actions 允许同一工作流中的多个作业,以及同一库中的多个工作流运行并发执行——多个步骤可以同时运行。

GitHub Actions 还可以控制工作流运行的并发性,以便确保在特定上下文中一次只运行一个运行、一个作业或一个步骤,这适合于控制同时运行多个步骤可能引起冲突,或消耗比预期更多操作时间的情况。

并发组可以有一个静态名称,但也可以使用上下文表达式将某些上下文(例如:分支)分到一个组中:

\begin{shell}
concurrency:
  group: ${{ github.workflow }}-${{ github.ref }}
  cancel-in-progress: true
\end{shell}

cancel-in-progress 参数可用于在可用新版本时取消作业并运行该版本,有助于节省资源。如果将其设置为 false,工作流将等待前一个版本完成,然后运行并发队列中的下一个作业。这样,对于给定的组,一次只执行一个作业。

可以在这里了解有关并发组的更多信息:\url{https://docs.github.com/en/actions/running-jobs/using-concurrency}。

\mySubsubsection{6.4.4}{There’s more…}

自动验证代码并将详细信息带到对开发人员有价值的地方非常重要。除了测试结果、代码覆盖率和检查 之外,还可以集成 SonarQube 或 SonarCloud 等解决方案。它们提供了 GitHub 集成,并且对开源项目免费。只需使用 GitHub 凭据登录,就可以配置一个新项目。该项目可以创建徽章,可以将其添加到 README 中(见图 6.9):

\myGraphic{0.6}{content/chapter6/images/9.png}{图6.9 --- 在GitHub README 中使用 SonarCloud 徽章}

Sonar 徽章提供了与拉取请求的集成,并且可以用作规则集和分支保护规则中的状态检查。质量门在拉取请求中报告(见图 6.10):

\myGraphic{0.6}{content/chapter6/images/10.png}{图6.10 --- SonarCloud 质量门在拉取请求中的集成}

要了解有关SonarCloud集成GitHub的更多信息,请参阅\url{https://docs.sonarsource.com/sonarcloud/getting-started/github/}。














