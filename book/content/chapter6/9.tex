在本示例中,我们将使用缓存来优化操作的速度。

\mySubsubsection{6.9.1}{Getting ready}

切换到一个新分支:

\begin{shell}
$ git switch -c cache-npm-packages
\end{shell}

\mySubsubsection{6.9.2}{How to do it…}

\begin{enumerate}
\item 
编辑 .github/workflows/ci.yml 文件。在 setup-node 操作之后,添加以下脚本以获取正确的 npm 版本的 npm 缓存目录,并将其存储为输出变量:

\begin{shell}
- name: Get npm cache directory
  id: npm-cache-dir
  run: echo "dir=$(npm config get cache)" >> "${GITHUB_OUTPUT}"
\end{shell}

\item 
紧接着添加实际的缓存步骤。同时,给它一个名称,并从 package-lock.json 文件的哈希值创建一个密钥:

\begin{shell}
- uses: actions/cache@v3
  id: npm-cache
  with:
    path: ${{ steps.npm-cache-dir.outputs.dir }}
    key: ${{ runner.os }}-node-${{ hashFiles('**/package-lock.json') }}
      restore-keys: |
        ${{ runner.os }}-node-
\end{shell}

\item 
下一步将仅列出依赖项,将其添加到缓存中:

\begin{shell}
- name: List the state of node modules
  if: ${{ steps.npm-cache.outputs.cache-hit != 'true' }}
  continue-on-error: true
  run: npm list
\end{shell}

\item 
提交更改并创建一个拉取请求:

\begin{shell}
$ git add.
$ git commit
$ gh pr create --fill
\end{shell}

\item 
在拉取请求中打开 CI 工作流运行,并检查添加的步骤的输出。重新运行两个作业,并注意操作如何从缓存中恢复包,以及 cache-hit 如何阻止下一步执行(见图 6.21):

\myGraphic{0.6}{content/chapter6/images/21.png}{图6.21 --- 缓存成功恢复且 cache-hit 为 true}

\item 
合并拉取请求:

\begin{shell}
$ gh pr merge -s --auto
\end{shell}

\end{enumerate}

\mySubsubsection{6.9.3}{How it works…}

只有在遇到性能问题时才应使用缓存。如果使用的是 setup 动作,可能不会看到显著的改进。但了解缓存的工作原理很重要,这样在需要时就可以使用该工具。

缓存操作(\url{https://github.com/actions/cache})将信息存储在 GitHub 拥有的云存储中,并在后续运行中从该存储中检索。

假设您有一个长时间运行的操作(例如:计算质数),并且后续步骤中使用该输出。(这里我使用 sleep 来模拟长时间运行的任务):

\begin{shell}
- name: Generate Prime Numbers
  run: |
    sleep 60
    echo "1 2 3..." > primes

- name: Use Prime Numbers
  run: cat primes
\end{shell}

现在,可以在该步骤之前添加一个缓存步骤,并缓存 primes 文件夹:

\begin{shell}
- name: Cache Primes
  id: cache-primes
  uses: actions/cache@v3
  with:
    path: primes
    key: ${{ runner.os }}-primes
\end{shell}

这里的键是确定缓存是否已更改所必需的唯一标识。我们的示例中,使用了 package-lock.json 文件的哈希值。对于质数,可能针对操作系统使用不同的格式。

在长时间运行的操作中,可以检查缓存是否有效,并且仅在当前键未命中时才执行:

\begin{shell}
- name: Generate Prime Numbers
  if: steps.cache-primes.outputs.cache-hit != 'true'
  run: |
    sleep 60
    echo "1 2 3..." > primes
\end{shell}

这个工作流在第一次运行长时间操作时会执行,并将文件填充到缓存中。如果第二次运行,将从缓存中加载文件,并跳过长时间运行的步骤。

\mySubsubsection{6.9.4}{There’s more…}

一个库最多可以存储 10 GB 的缓存数据。当达到该限制,较早的文件将根据它们上次访问的时间被移除,超过一周未被使用的缓存也会进行清理。

可以在 \textbf{Actions | Caches}下管理缓存(见图6.22):

\myGraphic{0.6}{content/chapter6/images/22.png}{图6.22 --- 管理缓存}

缓存操作(\url{https://github.com/actions/cache})有良好的文档和针对大多数编程语言的示例。想要实现缓存操作时,也可以参考文档(\url{https://docs.github.com/en/actions/using-workflows/caching-dependencies-to-speed-up-workflows})。