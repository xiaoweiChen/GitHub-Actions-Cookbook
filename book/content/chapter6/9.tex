在这个配方中,我们将使用缓存来优化工作流程的速度。

\mySubsubsection{6.9.1}{Getting ready}

切换到一个新分支:

\begin{shell}
$ git switch -c cache-npm-packages
\end{shell}

\mySubsubsection{6.9.2}{How to do it…}

\begin{enumerate}
\item 
编辑 .github/workflows/ci.yml 文件。在 setup-node 操作之后,添加以下脚本以获取正确 npm 版本的 npm 缓存目录,并将其存储为输出变量:

\begin{shell}
- name: Get npm cache directory
  id: npm-cache-dir
  run: echo "dir=$(npm config get cache)" >> "${GITHUB_OUTPUT}"
\end{shell}

\item 
在那之后添加实际的缓存步骤。还要给它一个名称,并从 package-lock.json 文件的哈希值中创建一个键:

\begin{shell}
- uses: actions/cache@v3
  id: npm-cache
  with:
    path: ${{ steps.npm-cache-dir.outputs.dir }}
    key: ${{ runner.os }}-node-${{ hashFiles('**/package-lock.json') }}
      restore-keys: |
        ${{ runner.os }}-node-
\end{shell}

\item 
下一步将仅在将它们添加到缓存时列出依赖项:

\begin{shell}
- name: List the state of node modules
  if: ${{ steps.npm-cache.outputs.cache-hit != 'true' }}
  continue-on-error: true
  run: npm list
\end{shell}

\item 
提交您的更改并创建一个拉取请求:

\begin{shell}
$ git add.
$ git commit
$ gh pr create --fill
\end{shell}

\item 
在您的拉取请求中打开 CI 工作流程运行并检查我们添加的步骤的输出。重新运行这两个作业,并注意操作如何从缓存中恢复软件包,以及 cache-hit 如何防止下一步执行(参见图 6.21):

\myGraphic{0.4}{content/chapter6/images/21.png}{图6.21 --- 缓存成功恢复并且 cache-hit 为 true}

\item 
合并您的拉取请求:

\begin{shell}
$ gh pr merge -s --auto
\end{shell}

\end{enumerate}

\mySubsubsection{6.9.3}{How it works…}

重要的是只在有性能问题时使用缓存。如果你使用设置操作,你可能看不到很大的改进。但是了解缓存如何工作是很重要的,以便在需要时可以使用这个工具。

缓存操作(\url{https://github.com/actions/cache})在GitHub拥有的云存储上存储信息,并在后续运行中从那里检索它。

假设你有一个长时间运行的操作——比如计算质数——你在后续步骤中使用输出。(我这里只是使用sleep来模拟一个长时间运行的任务):

\begin{shell}
- name: Generate Prime Numbers
  run: |
    sleep 60
    echo "1 2 3..." > primes

- name: Use Prime Numbers
  run: cat primes
\end{shell}

你现在可以在那个步骤之前添加一个缓存步骤,并缓存primes文件夹:

\begin{shell}
- name: Cache Primes
  id: cache-primes
  uses: actions/cache@v3
  with:
    path: primes
    key: ${{ runner.os }}-primes
\end{shell}

这里的关键是必须唯一,以确定缓存是否已更改。在我们的示例中,我们使用了package-lock.json文件的哈希值。对于质数,我们可能对操作系统有不同的格式。

在长时间运行的操作中,你现在可以检查缓存是否有效,并且只在当前键没有命中时执行它:

\begin{shell}
- name: Generate Prime Numbers
  if: steps.cache-primes.outputs.cache-hit != 'true'
  run: |
    sleep 60
    echo "1 2 3..." > primes
\end{shell}

这个工作流程将在第一次长时间运行的操作中运行,并在缓存中填充文件。如果你第二次运行它,它将从缓存中加载文件,并省略长时间运行的步骤。

\mySubsubsection{6.9.4}{There’s more…}

一个存储库可以在缓存中存储最多10 GB的数据。一旦达到这个限制,旧的文件将根据最后一次访问的时间被删除。一周内未使用的缓存也将被清理。

你可以在 Actions | Caches下管理缓存(见图6.22):

\myGraphic{0.4}{content/chapter6/images/22.png}{图6.22 --- 管理缓存}

缓存操作(\url{https://github.com/actions/cache})对大多数编程语言都有很好的文档和示例。当你想实现缓存时,也可以参考文档(\url{https://docs.github.com/en/actions/using-workflows/caching-dependencies-to-speed-up-workflows})。