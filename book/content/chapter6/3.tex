
在本示例中,将为不同的版本构建和测试软件。在我们的示例中,使用的是 NodeJS 环境。

\mySubsubsection{6.3.1}{Getting ready}

确保已从上一个示例中克隆了库,并创建一个新分支来修改工作流:

\begin{shell}
$ git switch -c build-matrix
\end{shell}

在编辑器中打开 .github/workflows/ci.yml 文件。

\mySubsubsection{6.3.2}{How to do it…}

\begin{enumerate}
\item 
将以下代码添加到工作流文件中:

\begin{shell}
strategy:
  matrix:
    node-version: ["21.x", "20.x"]
\end{shell}

根据需要调整版本。

\item 
在 actions/setup-node 操作中,将 node 版本设置为矩阵上下文中的对应值:

\begin{shell}
- uses: actions/setup-node@v4
  with:
    node-version: ${{ matrix.node-version }}
    check-latest: true
\end{shell}

\item 
提交并推送更改,并创建一个拉取请求:

\begin{shell}
$ git add .
$ git commit
$ git push -u origin build-matrix
$ gh pr create --fill
\end{shell}

\item 
检查工作流的输出,其将为矩阵数组中的每个条目运行一个单独的作业(见图 6.3):

\myGraphic{0.6}{content/chapter6/images/3.png}{图6.3 --- 矩阵为每个条目运行一个不同的作业}

\item 
等待工作流完成,合并拉取请求并清理库:

\begin{shell}
$ gh pr merge -m
\end{shell}
\end{enumerate}

\mySubsubsection{6.3.3}{How it works…}

矩阵是一种方便的方法,可以使用不同的组合来使用相同的工作流作业。可以包含一个或多个数组,这些数组可以包含许多值。矩阵将运行所有数组中所有值的所有组合,可以将矩阵视为嵌套的 for 循环。一个很好的例子是在不同平台上运行和测试不同版本:

\begin{shell}
jobs:
  example_matrix:
    strategy:
      matrix:
        os: [ubuntu-22.04, ubuntu-20.04]
        version: [10, 12, 14]
    runs-on: ${{ matrix.os }}
    steps:
      - uses: actions/setup-node@v3
      with:
        node-version: ${{ matrix.version }}
\end{shell}

这样,可以重用相同的工作流逻辑,同时测试许多不同的值组合。

\mySubsubsection{6.3.4}{There’s more…}

矩阵有一些功能。可以设置 fail-fast 来指示如果矩阵中的一个作业失败,是否会取消工作流,或者应该继续。可以使用 max-parallel 定义并行作业的数量,并且可以为某些元素包含和排除值。这是一个更加复杂的示例:

\begin{shell}
jobs:
  test:
    runs-on: ubuntu-latest
    continue-on-error: ${{ matrix.experimental }}
    strategy:
      fail-fast: true
      max-parallel: 2
      matrix:
        version: [5, 6, 7, 8]
        experimental: [false]
        include:
          - version: 9
            xperimental: true
\end{shell}

要了解更多关于矩阵策略的信息,请访问:\url{https://docs.github.com/en/actions/running-jobs/using-a-matrix-for-your-jobs}。















