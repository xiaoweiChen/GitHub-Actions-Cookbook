
如果您现在创建一个新的发布,工作流程将会失败,因为它会试图再次在软件包注册表上发布版本 1.0.0。您将不得不手动在 package.json 文件中设置版本号。在这个指南中,我们将使用 GitVersion 来自动化这个过程。

\mySubsubsection{6.7.1}{Getting ready}

切换到一个新的分支:

\begin{shell}
$ git switch -c add-gitversion
\end{shell}

\mySubsubsection{6.7.2}{How to do it…}

\begin{enumerate}
\item 
为了让 GitVersion 自动确定您 git 工作流程的版本,您必须下载所有引用,而不仅仅是 HEAD 分支。我们通过向检出操作添加 fetch-depth 参数并将其设置为 0 来实现这一点:

\begin{shell}
- uses: actions/checkout@v4
  with:
    fetch-depth: 0
\end{shell}

\item 
设置特定版本的 GitVersion:

\begin{shell}
- name: Install GitVersion
  uses: gittools/actions/gitversion/setup@v0.10.2
  with:
    versionSpec: '5.x'
\end{shell}

\item 
执行 GitVersion 以确定版本号:

\begin{shell}
  - name: Determine Version
    uses: gittools/actions/gitversion/execute@v0.10.2'
\end{shell}

\item 
更改 npm 包的版本:

\begin{shell}
  - name: 'Change NPM version'
    uses: reedyuk/npm-version@1.2.2
    with:
      version: $GITVERSION_SEMVER
\end{shell}

\item 
提交您的更改,创建一个拉取请求,并在所有检查都通过后合并它:

\begin{shell}
$ git add .
$ git commit -m '(build): Add GitVersion to automatically set
version'
$ gh pr create --fill
$ gh pr merge -m --auto
$ gh pr checks
\end{shell}

\item 
等待所有检查都通过。然后,创建发布:

\begin{shell}
$ gh release create v1.0.1 --generate-notes
\end{shell}

您将拥有一个如图 6.18 所示的新发布——这次是从 CLI 创建的:

\myGraphic{0.4}{content/chapter6/images/18.png}{图6.18 ---  从 CLI 创建一个发布,这将触发工作流程以发布一个与发布版本相同的软件包}

发布将触发工作流程,并且将发布版本为 1.0.1 的新软件包。
\end{enumerate}

\mySubsubsection{6.7.3}{How it works…}

GitVersion(参见 \url{https://gitversion.net/docs/})是一个根据您的 Git 历史记录自动生成语义化版本号的工具。GitVersion 使用默认配置运行,该配置适用于 GitHub 流程(\url{https://docs.github.com/en/get-started/usinggithub/github-flow})和 Git 流程(\url{https://nvie.com/posts/a-successfulgit-branching-model/})。您可以运行 GitVersion init 来启动一个向导,该向导会指导您创建一个配置文件(GitVersion.yml)。在我们的示例中,我们使用持续交付模式——这意味着我们显式使用标签创建一个版本。但也有其他模式,如持续部署模式(从对特定分支的每次提交创建版本)或主线模式。

gittools/actions/gitversion/execute 操作将执行 GitVersion 并将结果保存在 \$GITVERSION\_SEMVER 环境变量中。您还可以使用操作的输出来访问版本和配置的各个部分:

\begin{shell}
- name: Determine Version
  id: gitversion
  uses: gittools/actions/gitversion/execute@v0

- name: Display GitVersion outputs (step output)
  run: |
    echo "Major: ${{ steps.gitversion.outputs.major }}"
    echo "Minor: ${{ steps.gitversion.outputs.minor }}"
    echo "Patch: ${{ steps.gitversion.outputs.patch }}"
\end{shell}

\mySubsubsection{6.7.4}{There’s more…}

您还可以使用传统提交(参见 \url{https://www.conventionalcommits.org})来自动确定提交消息中的语义化版本。传统提交是一个规范,提供了一组规则,通过在提交消息中描述功能、修复和破坏性更改,创建一个显式的提交历史。您可以使用传统提交来创建发布说明——并且您可以将其与 GitVersion 结合使用,以自动确定新版本是补丁、次要还是主要版本号。这是在 GitVersion.yml 配置中完成的:

\begin{shell}
mode: Mainline
major-version-bump-message:
"^(build|chore|ci|docs|feat|fix|perf|refactor|revert|style|test)(\\([\\w\\s-]*\\))?(!:|:.*\\n\\n((.+\\n)+\\n)?BREAKING CHANGE:\\s.+)"minor-version-bump-message: "^(feat)(\\([\\w\\s-]*\\))?:"
patch-version-bump-message:
"^(build|chore|ci|docs|fix|perf|refactor|revert|style|test)(\\([\\w\\s-]*\\))?:"
\end{shell}

在 CI 构建中,您可以添加一个作业,该作业根据传统提交消息确定版本并创建一个新发布:

\begin{shell}
publish:
  if: ${{ github.ref == 'refs/heads/main' }}
  needs: build
  runs-on: ubuntu-latest
  permissions:
    contents: write

  steps:
    - uses: actions/checkout@v4
      with:
        fetch-depth: 0

    - name: Install GitVersion
      uses: gittools/actions/gitversion/setup@v0.10.2
      with:
        versionSpec: '5.x'

    - name: Determine Version
      uses: gittools/actions/gitversion/execute@v0.10.2
      with:
        useConfigFile: true

    - name: Create a new release
      env:
        GH_TOKEN: ${{ github.token }}
      run: |
        gh release create ${{ env.GITVERSION_SEMVER }}
--generate-notes
\end{shell}

这两个工作流一起将在每次生成主版本后完全自动化新版本的创建。