
如果创建了一个新发布,工作流将失败,因为它会尝试再次将版本 1.0.0 发布到软件包注册表。必须手动在 package.json 文件中设置版本号。

在本示例中,我们将使用 GitVersion 来自动化此过程。

\mySubsubsection{6.7.1}{Getting ready}

切换到一个新的分支:

\begin{shell}
$ git switch -c add-gitversion
\end{shell}

\mySubsubsection{6.7.2}{How to do it…}

\begin{enumerate}
\item 
为了让 GitVersion 自动确定 Git 工作流的版本,需要下载所有引用,而不仅仅是 HEAD 分支。我们通过将 fetch-depth 参数添加到 checkout 动作,并将其设置为 0 来进行实现:

\begin{shell}
- uses: actions/checkout@v4
  with:
    fetch-depth: 0
\end{shell}

\item 
以特定版本设置 GitVersion:

\begin{shell}
- name: Install GitVersion
  uses: gittools/actions/gitversion/setup@v0.10.2
  with:
    versionSpec: '5.x'
\end{shell}

\item 
执行 GitVersion 以确定版本号:

\begin{shell}
  - name: Determine Version
    uses: gittools/actions/gitversion/execute@v0.10.2'
\end{shell}

\item 
更改 npm 包的版本:

\begin{shell}
  - name: 'Change NPM version'
    uses: reedyuk/npm-version@1.2.2
    with:
      version: $GITVERSION_SEMVER
\end{shell}

\item 
提交更改,创建一个拉取请求,并在所有检查通过后进行合并:

\begin{shell}
$ git add .
$ git commit -m '(build): Add GitVersion to automatically set
version'
$ gh pr create --fill
$ gh pr merge -m --auto
$ gh pr checks
\end{shell}

\item 
等待所有检查通过,再创建发布:

\begin{shell}
$ gh release create v1.0.1 --generate-notes
\end{shell}

将拥有一个如图 6.18 所示的新发布——这次使用 CLI 创建:

\myGraphic{0.6}{content/chapter6/images/18.png}{图6.18 --- 从 CLI 创建发布,这将触发工作流发布一个与发布版本相同的新软件包}

该发布将触发工作流,并发布一个版本为 1.0.1 的新软件包。
\end{enumerate}

\mySubsubsection{6.7.3}{How it works…}

GitVersion(\url{https://gitversion.net/docs/})是一个工具,可根据您的 Git 历史记录自动生成语义化版本号。GitVersion 默认使用与 GitHub flow(\url{https://docs.github.com/en/get-started/usinggithub/github-flow})和 Git flow(\url{https://nvie.com/posts/a-successfulgit-branching-model/})兼容的配置运行。

可以运行 GitVersion init 来启动一个向导,该向导将引导您创建一个配置文件(GitVersion.yml)。在我们的示例中,使用持续交付模式——需要显式地使用标签创建一个版本。但还有其他模式,例如持续部署模式(从特定分支的每个提交创建版本)或主线模式。

gittools/actions/gitversion/execute 操作将执行 GitVersion 并将结果保存在 \$GITVERSION\_SEMVER 环境变量中,还可以使用该操作的输出访问版本的各个部分和配置:

\begin{shell}
- name: Determine Version
  id: gitversion
  uses: gittools/actions/gitversion/execute@v0

- name: Display GitVersion outputs (step output)
  run: |
    echo "Major: ${{ steps.gitversion.outputs.major }}"
    echo "Minor: ${{ steps.gitversion.outputs.minor }}"
    echo "Patch: ${{ steps.gitversion.outputs.patch }}"
\end{shell}

\mySubsubsection{6.7.4}{There’s more…}

还可以使用规范提交(Conventional Commits)(\url{https://www.conventionalcommits.org})从提交消息中自动确定语义化版本。规范提交是一种规范,提供了一组规则,用于通过描述提交消息中的特性、修复和破坏性变更来创建明确的提交历史。可以使用规范提交来创建发布说明——可以将其与 GitVersion 一起使用,以自动确定新版本是补丁、次要还是主要版本。可以通过 GitVersion.yml 配置文件完成:

\begin{shell}
mode: Mainline
major-version-bump-message:
"^(build|chore|ci|docs|feat|fix|perf|refactor|revert|style|test)(\\([\\w\\s-]*\\))? (!:|:.*\\n\\n((.+\\n)+\\n)?BREAKING CHANGE:\\s.+)"minor-version-bump-message: "^(feat)(\\([\\w\\s-]*\\))?:"
patch-version-bump-message:
"^(build|chore|ci|docs|fix|perf|refactor|revert|style|test)(\\([\\w\\s-]*\\))?:"
\end{shell}

在 CI 构建中,可以添加一个作业,该作业从规范提交消息中确定版本,并创建一个新发布:

\begin{shell}
publish:
  if: ${{ github.ref == 'refs/heads/main' }}
  needs: build
  runs-on: ubuntu-latest
  permissions:
    contents: write

  steps:
    - uses: actions/checkout@v4
      with:
        fetch-depth: 0

    - name: Install GitVersion
      uses: gittools/actions/gitversion/setup@v0.10.2
      with:
        versionSpec: '5.x'

    - name: Determine Version
      uses: gittools/actions/gitversion/execute@v0.10.2
      with:
        useConfigFile: true

    - name: Create a new release
      env:
        GH_TOKEN: ${{ github.token }}
      run: |
        gh release create ${{ env.GITVERSION_SEMVER }}
--generate-notes
\end{shell}

这两个工作流结合在一起,将完全自动化在每次合并到 main 分支后创建新版本的过程。