
在这个指南中,我们将创建一个工作流程,每当发布被创建时,它将发布我们的软件包。

\mySubsubsection{6.6.1}{Getting ready}

创建一个新的分支:

\begin{shell}
$ git switch -c add-release-workflow
\end{shell}

\mySubsubsection{6.6.2}{How to do it…}

\begin{enumerate}
\item 
创建一个新的 .github/workflows/release.yml 工作流程文件。将工作流程命名为 Release,并在 GitHub 发布创建时触发它:

\begin{shell}
name: Release

on:
  release:
    types: [created]
\end{shell}

\item 
添加一个发布作业,并给它对存储库的读取权限和对软件包的写入权限:

\begin{shell}
jobs:
  publish:
    runs-on: ubuntu-latest
    permissions:
      packages: write
      contents: read
\end{shell}

\item 
检出代码并使用正确版本配置 NodeJS 环境。此外,设置用于 GitHub 的注册表:

\begin{shell}
steps:
  - uses: actions/checkout@v4

  - uses: actions/setup-node@v4
    with:
      node-version: 21.x
      registry-url: https://npm.pkg.github.com/
\end{shell}

\item 
构建和测试应用程序,并使用 GitHub 令牌将其发布到注册表:

\begin{shell}
- name: Install dependencies
  run: npm install

- name: Run tests
  run: npm test

- run: npm publish
  env:
    NODE_AUTH_TOKEN: ${{secrets.GITHUB_TOKEN}}
\end{shell}

\item 
提交您的更改,创建一个拉取请求,并在检查通过后合并更改:

\begin{shell}
$ git add .
$ git commit
$ gh pr create --fill
$ gh pr merge -m --auto
\end{shell}

\item 
一旦您的拉取请求被合并,转到您存储库的Code选项卡上的Releases,然后点击Draft a new Release。

\item 
点击“ Choose a tag”,输入 v1.0.0,然后点击“ Create new tag”。不要在 Markdown 中添加发布内容,只需点击Generate release notes(见图 6.14):

\myGraphic{0.4}{content/chapter6/images/14.png}{图6.14 --- 为标签草拟新发布并生成发布说明}

这将根据您的拉取请求生成发布的详细信息。点击 Publish release以创建发布。发布应该看起来像图 6.15:

\myGraphic{0.4}{content/chapter6/images/15.png}{图6.15 ---  GitHub 发布的详细信息}

\item 
默认情况下,它还应该包含源代码作为 .zip 和 .tar 归档文件(见图 6.16):

\myGraphic{0.4}{content/chapter6/images/16.png}{图6.16 --- 发布的资产包含源代码作为 .zip 和 .tar 归档文件}

\item 
发布的创建将触发工作流程,它将把版本 1.0.0 的软件包发布到您的存储库(请注意,这来自您的 package.json 文件,而不是标签!)。

软件包看起来像图 6.17:

\myGraphic{0.4}{content/chapter6/images/17.png}{图6.17 --- 存储库中的软件包}

它包含安装说明、有关您存储库的信息以及带有我们徽章的 README 文件。

\end{enumerate}

\mySubsubsection{6.6.3}{How it works…}

现在,我们将看看它是如何工作的。

\mySamllsectionNoContent{语义化版本控制}

软件包通常使用语义化版本控制来创建,这是一种用于指定软件版本号的正式约定。它由具有不同含义的不同部分组成。语义化版本号的示例是 1.0.0 或 1.5.99-beta。格式如下:

\begin{shell}
<major>.<minor>.<patch>-<pre>
\end{shell}

\begin{itemize}
\item 
主版本号(Major version):一个数字标识符,如果不向后兼容且有破坏性更改,则该标识符会增加。更新到新的主版本号必须谨慎处理!主版本号为零表示初始开发。

\item 
次版本号(Minor version):一个数字标识符,如果添加了新功能,但版本与之前版本向后兼容,并且如果您需要新功能,可以更新而不会破坏任何内容,则该标识符会增加。

\item 
补丁号(Patch):一个数字标识符,如果您发布了向后兼容的错误修复,则该标识符会增加。应该始终安装新的补丁。

\item 
预发布版本(Pre-version):使用连字符附加的文本标识符。标识符必须仅使用 ASCII 字母数字字符和连字符([0-9A-Za-z-])。文本越长,预发布版本越小(即 -alpha < -beta < -rc)。预发布版本始终小于正常版本(1.0.0-alpha < 1.0.0)。
\end{itemize}

有关完整规范,请参阅 \url{https://semver.org/}。

在我们的情况下,软件包的语义化版本在 package.json 文件中设置。在下一个指南中,我们将使用发布流程自动将软件包的版本号设置为发布版本,以便您不必手动执行此操作。

\mySamllsectionNoContent{发布}

您可以创建一个发布来打包软件、发布说明和其他人可以下载的二进制文件。

GitHub 发布是可以打包并供更广泛的受众下载和使用的可部署软件版本。它们可以包含发布说明和可下载的其他二进制文件。

发布基于 Git 标签,这些标签标记了您存储库历史中的特定点。

在接下来的指南中,我们将使用 GitHub 发布以及标签来自动版本化我们的软件包,并以自动方式附加 SBOM。要了解有关 GitHub 发布的更多信息,请参阅 \url{https://docs.github.com/en/repositories/releasing-projects-on-github}。

\mySubsubsection{6.6.4}{There’s more…}

软件的发布和版本控制很大程度上取决于您的工作流程 --- 尤其是您如何处理分支和标签。在我们的示例中,我假设您通过直接创建发布来启动发布流程。您也可以使用推送标签来触发工作流程:

\begin{shell}
on:
  push:
    tags:
      - v*.**
\end{shell}

工作流程将被任何以 v 开头的标签的推送触发,您可以使用它来自动创建一个发布:

\begin{shell}
gh release create ${{ github.ref_name }} --generate-notes
\end{shell}

您也可以在推送发布分支时执行此操作:

\begin{shell}
on:
  push:
    branches:
      - release/*
\end{shell}






