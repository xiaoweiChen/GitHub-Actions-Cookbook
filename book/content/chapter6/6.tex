
本示例中,我们将创建一个工作流,以便在创建发布时发布软件包。

\mySubsubsection{6.6.1}{Getting ready}

创建一个新的分支:

\begin{shell}
$ git switch -c add-release-workflow
\end{shell}

\mySubsubsection{6.6.2}{How to do it…}

\begin{enumerate}
\item 
创建一个新的 .github/workflows/release.yml 工作流文件。将工作流命名为 Release,并在创建 GitHub 发布时触发:

\begin{shell}
name: Release

on:
  release:
    types: [created]
\end{shell}

\item 
添加一个 publish 作业,并为其授予对库的读取权限和对软件包的写入权限:

\begin{shell}
jobs:
  publish:
    runs-on: ubuntu-latest
    permissions:
      packages: write
      contents: read
\end{shell}

\item 
检出代码,并使用正确的版本配置 NodeJS 环境,并将注册表设置为使用 GitHub:

\begin{shell}
steps:
  - uses: actions/checkout@v4

  - uses: actions/setup-node@v4
    with:
      node-version: 21.x
      registry-url: https://npm.pkg.github.com/
\end{shell}

\item 
构建并测试应用程序,并使用 GitHub 令牌将其发布到注册表:

\begin{shell}
- name: Install dependencies
  run: npm install

- name: Run tests
  run: npm test

- run: npm publish
  env:
    NODE_AUTH_TOKEN: ${{secrets.GITHUB_TOKEN}}
\end{shell}

\item 
提交更改,创建一个拉取请求,并在检查通过后合并更改:

\begin{shell}
$ git add .
$ git commit
$ gh pr create --fill
$ gh pr merge -m --auto
\end{shell}

\item 
拉取请求合并后,转到库的\textbf{Code}(代码)选项卡下的\textbf{Releases}(发布)部分,然后点击\textbf{Draft a new Release}(草拟新发布)。

\item 
点击\textbf{Choose a tag}(选择标签),输入 v1.0.0,然后点击\textbf{Create new tag}(创建新标签)。不要以 Markdown 格式添加发布说明正文,只需点击\textbf{Generate release notes}(生成发布说明)(见图 6.14):

\myGraphic{0.6}{content/chapter6/images/14.png}{图6.14 --- 为标签草拟新发布并生成发布说明}

这将根据拉取请求生成发布详情,点击\textbf{Publish release}来创建发布。该发布应如图 6.15 所示:

\myGraphic{0.6}{content/chapter6/images/15.png}{图6.15 --- GitHub 发布的详细信息}

\item 
默认情况下,也应包含作为 .zip 和 .tar 归档的源代码(见图 6.16):

\myGraphic{0.6}{content/chapter6/images/16.png}{图6.16 --- 发布的资产包含作为 .zip 和 .tar 归档的源代码}

\item 
创建发布将触发工作流,并将软件包以版本 1.0.0 发布到库(请注意,这来自 package.json 文件,而非标签!)。

该软件包看起来如图 6.17 所示:

\myGraphic{0.6}{content/chapter6/images/17.png}{图6.17 --- 库中的软件包}

包含安装说明、有关您库的信息,以及带有徽章的 README。

\end{enumerate}

\mySubsubsection{6.6.3}{How it works…}

现在,我们将看看它是如何工作的。

\mySamllsectionNoContent{语义化版本控制}

软件包通常使用语义化版本控制(Semantic Versioning)创建,这是一种为软件指定版本号的正式约定。它由具有不同含义的不同部分组成。语义化版本号的示例有 1.0.0 或 1.5.99-beta。其格式如下:

\begin{shell}
<major>.<minor>.<patch>-<pre>
\end{shell}

\begin{itemize}
\item 
主版本号(Major):当版本不向后兼容并具有破坏性更改时,此数字标识符会增加。升级到新的主版本号必须谨慎处理!主版本号为零表示处于初始开发阶段。

\item 
次版本号(Minor):当添加新功能但版本与先前版本向后兼容时,此数字标识符会增加。如果需要新功能,则可以更新而不会破坏任何内容。

\item 
修订版本号(Patch):当发布向后兼容的错误修复时,此数字标识符会增加。新的修订号应始终安装。

\item 
预发布标识符(Pre-release):使用连字符附加的文本标识符。标识符只能使用 ASCII 字母数字字符和连字符([0-9A-Za-z-])。文本越长,预发布版本越小(例如 -alpha < -beta < -rc)。预发布版本始终小于常规版本(例如 1.0.0-alpha < 1.0.0)。
\end{itemize}

有关完整规范,请参阅 \url{https://semver.org/}。

在我们的示例中,软件包的语义化版本是在 package.json 文件中设置的。在下一个示例中,将使用发布流程,来自动将软件包的版本号设置为发布版本。

\mySamllsectionNoContent{发布}

可以创建发布来打包软件、发布说明和供他人下载的二进制文件。

GitHub 发布是可部署的软件版本,可以将其打包并提供给更广泛的受众下载和使用,可以包含发布说明和其他可下载的二进制文件。

发布基于 Git 标签,这些标签标记了库历史中的特定点。

在接下来的示例中,将使用 GitHub 发布与标签一起,自动为软件包设置版本号,并自动添加 SBOM。有关 GitHub 发布的更多信息,请参阅 \url{https://docs.github.com/en/repositories/releasing-projects-on-github}。

\mySubsubsection{6.6.4}{There’s more…}

软件的发布和版本控制很大程度上取决于工作流——特别是如何使用 Git 分支和标签。在示例中,我假设可以通过直接创建发布来启动发布流程。当然,也可以使用标签进行推送:

\begin{shell}
on:
  push:
    tags:
      - v*.**
\end{shell}

该工作流将由以 v 开头的标签推送触发,可以使用它来自动创建发布:

\begin{shell}
gh release create ${{ github.ref_name }} --generate-notes
\end{shell}

也可以在推送到发布分支时这样做:

\begin{shell}
on:
  push:
    branches:
      - release/*
\end{shell}






