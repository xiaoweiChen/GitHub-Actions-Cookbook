在示例方中,将创建一个简单的持续集成 (CI) 流水,该流水构建和验证代码,并集成到拉取请求验证中。我们将在后续配方中使用此代码——这也是使用一个非常简单的 JavaScript 包的原因。

\mySubsubsection{6.2.1}{Getting ready}

我认为最好从头开始构建一个新库和 npm 包,即使不熟悉 JavaScript,也没什么问题,也可以直接使用该库:\url{https://github.com/wulfland/package-recipe}。

\begin{enumerate}
\item 
创建一个名为 package-recipe 的新库。使用 .gitignore 文件和 README 文件初始化,并选择 Node 作为 .gitignore 文件的模板。

\item 
将库克隆到本地,或使用 Codespaces 打开它。

\item 
运行以下命令:

\begin{shell}
$ npm init
\end{shell}

按照向导的提示操作。包的名称为 @<github-user-name>/packagerecipe。将测试命令添加为以下代码:

\begin{shell}
jest && make-coverage-badge
\end{shell}

Jest 是我们将用来测试应用程序的测试框架,而 make-coverage-badge 将用于稍后创建一个徽章,并将其添加到 README 文件中。

可以将其余属性保留为默认值。

\item 
完成后,运行以下命令:

\begin{shell}
$ npm install
\end{shell}

\item 
安装依赖项:

\begin{shell}
$ npm install --save-dev jest
$ npm install --save-dev make-coverage-badge
\end{shell}

\item 
接下来,让我们添加代码。创建一个新的 src/index.js 文件并添加以下行:

\begin{shell}
module.exports = function greet () {
  return 'Hello world!'
}
\end{shell}

我们正在创建一个简单的包,只会返回 Hello world!。

\item 
添加一个新的 \_\_tests\_\_/index.test.js 文件(注意双下划线)。添加以下测试:

\begin{shell}
describe('index.js', () => {
  it('greet function returns Hello world!', () => {
    const greet = require('../src/index')
    expect(greet()).toBe('Hello world!')
  })
})
\end{shell}

这只是一个简单的测试,用于检查 index.js 是否按预期将文本写入控制台。

\item 
打开 package.json 文件,并添加以下 Jest 的配置:

\begin{shell}
"jest": {
  "verbose": true,
  "coverageReporters": [
    "json-summary",
    [
      "text",
      {
        "file": "coverage.txt",
        "path": "./coverage"
      }
    ],
    "lcov"
  ],
  "collectCoverage": true,
  "collectCoverageFrom": [
    "./src/**"
  ]
},
\end{shell}

这将添加多个代码覆盖率报告输出。

\item 
执行测试: 

\begin{shell}
$ npm run test
\end{shell}

如果一切设置正确,这将执行一个测试并在 coverage 文件夹中写入报告,还会在 coverage/badge.svg 处生成一个徽章。

\item 
提交文件,并将它们推送到 GitHub。

\item 
在库中,转到 \textbf{Settings}(设置)并向下滚动到 \textbf{Pull Requests}(拉取请求)。检查 \textbf{Allow auto-merge}(允许自动合并)和 \textbf{Automatically delete head branches}(自动删除头分支)。
\end{enumerate}

\mySubsubsection{6.2.2}{How to do it…}

现在,已经准备好了一个包库,我们将用于下一个示例,开始添加 CI 工作流:

\begin{enumerate}
\item 
创建一个新的\_github/workflows/ci.yml 文件。

将工作流命名为 CI,并在每次拉取请求时触发以验证更改。同时,在每次推送到 main 分支时触发,以构建一个可以发布的新版本:

\begin{shell}
name: CI

on:
  pull_request:
  push:
    branches:
      - main
\end{shell}

\item 
添加一个构建作业,该作业在 ubuntu-latest 上运行并检出到库:

\begin{shell}
jobs:
  build:
    runs-on: ubuntu-latest

    steps:
      - uses: actions/checkout@v4
\end{shell}

\item 
配置工作流运行器以针对特定的 NodeJS 版本。我们可以为次要版本使用通配符,并让操作使用最新可用版本:

\begin{shell}
- uses: actions/setup-node@v4
  with:
    node-version: "21.x"
    check-latest: true
\end{shell}

\item 
构建并测试代码:

\begin{shell}
- name: Install dependencies
  run: npm install

- name: Run tests
  run: npm test
\end{shell}

\item 
提交并推送工作流到 main 分支。由于推送触发器,这应该已经触发了工作流,并且构建和测试应该成功。

\item 
为了测试验证,创建一个新分支:

\begin{shell}
$ git switch -c fail-pr
\end{shell}

\item 
修改 index.js 以执行会使测试失败的更改(例如,\verb|console.log('Hello Mars!);|)。添加并提交更改,然后创建一个拉取请求:

\begin{shell}
$ git add index.js
$ git commit
$ git push -u origin fail-pr
$ gh pr create --fill
\end{shell}

拉取请求将触发工作流,并且由于 npm run test 命令将返回一个非零值,工作流将失败。

\end{enumerate}

\mySubsubsection{6.2.3}{How it works…}

再来了解如何进行验证工作的。

\mySamllsectionNoContent{检出库}

CI 的第一步是使用 checkout 操作 (\url{https://github.com/actions/checkout}) 检出到库。该操作有许多参数。例如,可以设置要拉取到运行器的 Git 记录深度。默认值为 1,这只会下载 HEAD 分支。在从 Git 生成版本号的示例中,我们将设置为 0 以下载所有分支和标签:

\begin{shell}
steps:
  - uses: actions/checkout@v4
    with:
      fetch-depth:0 # Default: 1
\end{shell}

其他选项包括 lfs,用于确定是否应下载 Git 大文件存储 (LFS) 文件(默认为 false),或 submodules:

\begin{shell}
steps:
  - uses: actions/checkout@v4
    with:
      lfs: true # Default: false
      submodules: true # Default: false
\end{shell}
  
对于大型单库,还可以执行稀疏检出,并仅获取库特定区域的数据。此示例仅检出 .github、src 和 \_\_tests\_\_ 文件夹:

\begin{shell}
steps:
  - uses: actions/checkout@v4
    with:
      sparse-checkout: |
        .github
        src
        __tests__
\end{shell}

这可以极大地减少工作流所需的时间和存储空间。

\mySamllsectionNoContent{设置环境}

可为不同语言进行设置操作,包括:

\begin{itemize}
\item 
Node

\item
Python

\item
Java

\item
Go

\item
.NET

\item
Ruby

\item
Elixir

\item
Haskell
\end{itemize}

这里,我们使用 setup-node (\url{https://github.com/actions/setup-node})。确保设置操作在构建过程中,设置了正确的二进制文件和环境变量。所有这些操作都接受某种形式的版本参数:

\begin{shell}
- uses: actions/setup-node@v4
  with:
    node-version: 21
\end{shell}

通配符和别名也受支持。示例包括 21.x、21.5.0、>=21.5.0、lts/Hydrogen、21-nightly、latest。如果您使用通配符,则可以设置 check-latest: true 以检查可用的最新版本。

大多数 checkout 操作还支持不同的注册表以下载依赖项。对于 node,参数是 registry-url,将使用提供的 URL 和来自 env.NODE\_AUTH\_TOKEN 的令牌连接到特定注册表。

Checkout 操作通常还会缓存依赖项。在许多情况下,使用相应的设置操作比自己实现更有效。

\mySubsubsection{6.2.4}{There’s more…}

验证工作流最好与分支保护和规则集结合使用——还可以添加一个检查工具来进一步提高质量。

\mySamllsectionNoContent{超级代码检查工具}

在第2章中,我们使用了一个检查工具来验证和注释拉取请求中的工作流。同样,也可以用于代码。有一个名为 superlinter(\url{https://github.com/super-linter/super-linter})的 GitHub 操作,它基本上将所有可用的 检查工具 合并到一个操作中。可以这样使用它来 检查代码:

\begin{shell}
permissions:
  contents: read
  packages: read
  # To report GitHub Actions status checks
  statuses: write

steps:
  - name: Checkout code
    uses: actions/checkout@v4
    with:
    # super-linter needs the full git history to get the
    # list of files that changed across commits
    fetch-depth: 0

  - name: Super-linter
    uses: super-linter/super-linter@v5.7.2
    env:
      DEFAULT_BRANCH: main
      # To report GitHub Actions status checks
      GITHUB_TOKEN: ${{ secrets.GITHUB_TOKEN }}
\end{shell}

super-linter 的镜像相当大,可能会影响工作流的性能。如果不需要所有语言,还可以使用精简版本:

\begin{shell}
super-linter/super-linter/slim@[VERSION]
\end{shell}

排除了 rust、dotenv、armttk、pwsh 和 C\# 的 linter,并且镜像大小要小得多。在 v5 中,这将使大小从 7.35 GB 减少到 slim-v5 的 4.88 GB,并且将下载镜像的时间从大约 2 分钟减少到大约 1 分钟。

\mySamllsectionNoContent{分支保护}

在第2章中,我也介绍了分支保护。可以使用规则保护库中的一个或多个分支,并强制要求提交必须通过拉取请求合并,并且某些检查必须成功。除了手动审核外,还可以要求状态检查成功(见图 6.1):

\myGraphic{0.6}{content/chapter6/images/1.png}{图6.1 --- 将 GitHub Action 工作流作业添加为受保护分支的状态检查}

状态检查可以是一个工作流中的单个作业——或者是使用状态 API 报告状态的集成。可以为下一个示例在库中为 main 启用策略,并将状态检查策略设置为工作流的作业。有关受保护分支的更多信息,请参阅 \url{https://docs.github.com/en/repositories/configuring-branches-and-merges-in-your-repository/managing-protected-branches/about-protected-branches}。

\mySamllsectionNoContent{规则集}

规则集是分支保护的更强大的继任者,具有与受保护分支相同的功能。还可以像在受保护分支中一样定义状态检查(见图 6.2)

\myGraphic{0.6}{content/chapter6/images/2.png}{图6.2 --- 在规则集中添加 GitHub Action 工作流作业作为状态检查}

规则集可以使用的大多数规则与分支保护规则相似,并且可以组合使用而无需更改现有规则。

规则集比保护规则更高级,具有以下优势:

\begin{itemize}
\item 
规则集可以在组织级别定义,并且可以针对多个库。

\item
可以通过同时应用多个规则集来分层规则。

\item
规则集具有状态,允许在不删除的情况下在库中激活或停用规则集。

\item
只需要库的读取权限即可查看所有活动的规则集,可使得贡献者可以更清楚地了解适用的规则。

\item
有一些规则在保护规则中不可用,例如:控制提交消息或作者电子邮件地址的规则。
\end{itemize}

可以在这里了解更多关于规则集的信息:\url{https://docs.github.com/en/repositories/configuring-branches-and-merges-in-your-repository/managing-rulesets/about-rulesets}。