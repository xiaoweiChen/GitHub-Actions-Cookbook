在本示例中,我们将使用环境审批来在创建版本库之前获取审批,将使用一个 GitHub App 进行身份验证。因为版本库的创建通常发生在组织范围内,并且不能使用 GITHUB\_TOKEN,所以必须使用一个 GitHub App 或具有正确范围的个人访问令牌(PAT)。

\mySubsubsection{5.4.1}{Getting ready}

确保你已完成前面的示例,并继续在同一个库中进行操作。

\mySubsubsection{5.4.2}{How to do it…}

\begin{enumerate}
\item 
在库中,进入 \textbf{Settings}(设置) → \textbf{Environments}(环境),然后点击 \textbf{New environment}(新建环境)。

\item 
将环境命名为 repo-creation,然后点击 \textbf{Configure environment}(配置环境)。

\item 
将自己添加为 \textbf{Required reviewer}(必需审阅者),并且不允许管理员绕过此规则(请参阅图 5.7)。

\myGraphic{0.6}{content/chapter5/images/7.png}{图 5.7 --- 配置环境保护规则}

\item 
点击 \textbf{Save protection rules}(保存保护规则)。

\item 
下一步是创建一个应用(App)来进行身份验证。我建议使用组织(organization)来尝试此示例,也可以使用您的个人账户。前往 \url{https://github.com/settings/apps},然后点击 \textbf{New GitHub App}(新建 GitHub 应用)。

\item 
为其指定一个名称(例如,\{your username\}-repo-creation),并将 Homepage URL(主页 URL)设置为版本库的 URL。

\item 
在 Repository permissions(库权限) 部分,可选择以下权限:

\begin{itemize}
\item 
Administration(管理): Read and write(读写)

\item 
Contents(内容): Read and write(读写)

\item 
Issues(议题): Read and write(读写)
\end{itemize}

在 Organization permissions(组织权限) 部分,选择:

\begin{itemize}
\item 
Administration(管理): Read and write(读写)
\end{itemize}

\item 
点击 Save(保存) 应用。在新创建的应用页面中,点击 \textbf{Generate a private key}(生成私钥),私钥将会自动下载到本地设备上。

\item 
从应用的 \textbf{General}(常规) 选项卡中复制 App ID。

\item 
在 GitHub 上的应用中,选择 \textbf{Install App}(安装应用)并点击 \textbf{Install}(安装)。选择组织或账户,点击\textbf{Install}(安装),保持 \textbf{All repositories}(所有版本库)被选中,然后点击 \textbf{Install}(安装)。

\item 
返回到版本库中的环境。添加一个新的密钥 PRIVATE\_KEY,并将之前下载的密钥文件内容添加进去。另外,添加一个 APP\_ID 变量,其值为应用 ID(请参阅图 5.8)。

\myGraphic{0.6}{content/chapter5/images/8.png}{图 5.8 --- 向环境添加变量和密钥}

\item 
编辑工作流文件,并添加一个 create-repo 作业。使该作业依赖于之前的作业(needs: issue-ops),并将其分配给之前创建的环境(environment: repo-creation):

\begin{shell}
create-repo:
  needs: issue-ops
  runs-on: ubuntu-latest
  environment: repo-creation
\end{shell}

\item 
要使用 GitHub 令牌与问题进行交互,请设置工作流的权限:

\begin{shell}
  permissions:
    issues: write
    contents: read
\end{shell}

\item 
可以使用全局环境变量,这些变量可以在所有步骤中使用,将 REPO\_OWNER 设置为安装了应用的组织或账户,也可以将此值保存为环境中的变量。将 REPO\_NAME 设置为上一个作业的输出。USER 和 ISSUE\_NUMBER 设置为上下文的值,以便于访问:

\begin{shell}
env:
  REPO_OWNER: ${{ vars.ORGANIZATION }}
  REPO_NAME: ${{ needs.issue-ops.outputs.REPO_NAME }}
  USER: ${{ github.event.issue.user.login }}
  ISSUE_NUMBER: ${{ github.event.issue.number }}
\end{shell}

\item 
要使用应用进行身份验证,可以使用 actions/create-github-app-token 操作。给它一个 ID 以便后续引用:

\begin{shell}
steps:
  - name: Create app token
    uses: actions/create-github-app-token@v1.6.2
    id: get-workflow-token
    with:
      app-id: ${{ vars.APP_ID }}
      private-key: ${{ secrets.PRIVATE_KEY }}
      owner: ${{ vars.ORGANIZATION }}
\end{shell}

\item 
创建版本库,并将 URL 设置为输出参数:

\begin{shell}
- name: Create repository
  id: create-repo
  env:
    GH_TOKEN: ${{ steps.get-workflow-token.outputs.token }}
  run: |
    REPO_URL=$(gh repo create $REPO_OWNER/$REPO_NAME --private --clone)
    echo "repo_url=$REPO_URL" >> "$GITHUB_OUTPUT"
    echo "Repositeory '$REPO_NAME' has been successully created: $REPO_URL"
\end{shell}

\item 
如果操作成功,请在issue下发表评论,通知用户版本库已创建:

\begin{shell}
- name: Notify User
  if: ${{ success() }}
  env:
    GH_TOKEN: ${{ github.token }}
    REPO_URL: ${{ steps.create-repo.outputs.repo_url }}
  run: |
    gh issue comment $ISSUE_NUMBER \
      -b "@$USER: Repository '$REPO_OWNER/$REPO_NAME' has been created successfully: $REPO_URL" \
      --repo ${{ github.event.repository.full_name }}
\end{shell}

\item 
如果发生错误,也通过在问题下发表评论通知用户:

\begin{shell}
- name: Handle Exception
  if: ${{ failure() }}
  env:
    GH_TOKEN: ${{ github.token }}
  run: |
    gh issue comment $ISSUE_NUMBER \
      -b "@$USER: Repository '$REPO_OWNER/$REPO_NAME' creation failed. Please contact the administrator."\
      --repo ${{ github.event.repository.full_name }}
\end{shell}

\item 
提交并推送工作流,并创建一个新的 \textbf{Repository Request} issue。可以根据通知设置收到通知,以审查工作流。该工作流如图 5.9 所示。

\myGraphic{0.6}{content/chapter5/images/9.png}{图 5.9 --- 使用环境进行手动工作流审批}

\item 
点击 \textbf{Review deployments}(审核部署),选择环境,然后点击 \textbf{Approve and deploy}(批准并部署)(请参阅图 5.10)。

\myGraphic{0.6}{content/chapter5/images/10.png}{图 5.10 --- 批准部署到环境}

该工作流将创建版本库,并使用问题中的评论通知发起请求的用户(请参阅图 5.11)。

\myGraphic{0.6}{content/chapter5/images/11.png}{图 5.11 --- 工作流使用问题评论通知用户关于版本库创建成功}

检查链接是否有效,以及版本库是否正确创建。

\end{enumerate}

\mySubsubsection{5.4.3}{How it works…}

可以将工作流中的作业分配给环境,从而为该环境添加保护规则,以及特定的变量和密钥。

\mySamllsectionNoContent{环境}

环境通过网页界面或 API 在版本库中创建:

\begin{shell}
curl -L \
  -X PUT \
  -H "Accept: application/vnd.github+json" \
  -H "Authorization: Bearer <YOUR-TOKEN>" \
  -H "X-GitHub-Api-Version: 2022-11-28" \
  https://api.github.com/repos/OWNER/REPO/environments/<NAME> \
  -d '{"wait_timer":30,"prevent_self_review":false,"reviewers": [{"-type":"User","id":1}, {"type":"Team","id":1}], "deployment_branch_policy":{"protected_branches":false, "custom_branch_policies":true}}'
\end{shell}

大多数情况下,可以像本示例中那样使用网页界面来创建。在工作流中,环境通过其名称引用:

\begin{shell}
jobs:
  deployment:
    runs-on: ubuntu-latest
    environment: production
\end{shell}

可以添加一个 URL,该 URL 将在工作流中显示:

\begin{shell}
environment:
  name: production
  url: https://writeabout.net
\end{shell}

可以在工作流中创建动态环境并向其部署,可以将每个拉取请求部署到一个隔离的环境中进行测试。

可以使用不同的保护规则来保护环境:

\begin{itemize}
\item 
必需审阅者:可以列出最多六个用户或团队作为必需审阅者,以批准引用该环境的工作流作业。审阅者必须至少拥有版本库的读取权限,只需要一个必需审阅者批准作业即可继续。

\item 
等待计时器:可以在作业最初触发后暂停工作流一段时间。时间(以分钟为单位)必须是一个介于 0 和 43,200(30 天)之间的整数,可以使用 API 在该时间内取消工作流。

\item 
部署分支和标签:使用部署分支和标签来限制哪些分支和标签可以部署到该环境。环境针对部署分支和标签的选项为无限制、仅受保护分支或选定的分支和标签。

在后者设置中,可以添加名称模式以针对单个或一组分支或标签进行操作——例如: main 或 release/*。将环境连接到分支保护规则非常强大,可以为这些规则提供更多的保护规则——例如:强制执行代码所有者或部署到特定环境。
\end{itemize}

环境也有特定的密钥和变量,允许在同一个工作流中使用不同的配置。

还有自定义部署规则来保护环境,此功能在编写本书时仍处于公开测试阶段。自定义部署规则基本上是 GitHub Apps,允许编写自己的集成。这使得 Datadog、Honeycomb 和 ServiceNow 等服务能够为部署提供自动审批。

要了解有关环境的更多信息,请参阅:\url{https://docs.github.com/en/actions/ deployment/targeting-different-environments/using-environments-fordeployment}。

\mySamllsectionNoContent{认证}

可以使用 GitHub 令牌和工作流权限做很多事情。但在自动化组织级别的操作时,可能需要使用个人访问令牌(PAT)或 GitHub App。GitHub App 是推荐的方式,因其与用户无关。

我们已经在上一章中了解了 GitHub App。要在工作流中使用 GitHub App 进行身份验证,可以使用 actions/create-github-app-token 操作:

\begin{shell}
steps:
  - name: Create app token
    uses: actions/create-github-app-token@v1.6.2
    id: get-workflow-token
    with:
      app-id: ${{ vars.APP_ID }}
      private-key: ${{ secrets.PRIVATE_KEY }}
\end{shell}

需要应用 ID 和私钥,可将其存储为环境变量和密钥。然后,通过工作流步骤的输出来访问该令牌:

\begin{shell}
- name: Create repository
  env:
    GH_TOKEN: ${{ steps.get-workflow-token.outputs.token }}
\end{shell}

通过此操作,在工作流中使用 GitHub App 非常简单。

\mySubsubsection{5.4.4}{There’s more…}

issue是与用户交互的好方法——可能还希望将自动化的状态存储在其他地方,可以更新 YAML 或 Markdown 文件,或调用外部系统。

也可以使用 GitHub Projects 来可视化issue,这样issue就代表了自动化对象生命周期中的状态。

GitHub Projects 非常灵活,可以将其中的问题和拉取请求来自不同的版本库。所以它相当复杂,需要使用内部 ID 来引用字段。

在调整工作流之前,运行以下命令来列出用于跟踪问题的项目的字段(我们的例子中,ID 是 19,所有者是 wulfland):

\begin{shell}
$ gh project field-list <ID> --owner <OWNER> --format json | jq
\end{shell}

这将生成一个包含项目中所有字段的 JSON 对象。查找这些 ID。对于状态(Status)字段,还需要选项的 ID:

\begin{shell}
{
    "id": "PVTSSF_lAHOAFCCsc4AZoDtzgQZkEg",
    "name": "Status",
    "type": "ProjectV2SingleSelectField",
    "options": [
      {
        "id": "f75ad846",
        "name": "Request"
      },
      {
        "id": "e05aa0a3",
        "name": "Repository Created"
      },
      {
        "id": "98236657",
        "name": "Deleted"
      }
    ]
  },
\end{shell}

将内部 ID 作为变量存储在环境中,如图 5.12 所示。

\myGraphic{0.6}{content/chapter5/images/12.png}{图 5.12 --- 使用内部项目 ID 来引用字段和选项}

在工作流中,添加一个步骤并设置环境变量:

\begin{shell}
- name: Update Project
  if: ${{ success() }}
  env:
    GH_TOKEN: ${{ secrets.PROJECT_TOKEN }}
    REPO_URL: ${{ steps.create-repo.outputs.repo_url }}
    PROJECTNUMBER: ${{ vars.PROJECT_ID }}
    PROJECTOWNER: ${{ vars.PROJECT_OWNER}}
\end{shell}

首先,必须接收内部项目 ID,因为后续命令需要。普通的数字项目 ID 并不能在所有命令中使用:

\begin{shell}
run: |
  project_id=$(gh project list --owner "$PROJECTOWNER" --format json | jq -r '.projects[] | select(.number=='$PROJECTNUMBER') | .id')
\end{shell}

接下来,获取内部issue ID:

\begin{shell}
issue_id=$(gh project item-list $PROJECTNUMBER \
  --owner "$PROJECTOWNER" \
  --format json \
  | jq -r '.items[] \
  | select(.content.number=='$ISSUE_NUMBER') | .id')
\end{shell}

现在,可以使用工作流中的值来更新字段,将项目的状态设置为created选项:

\begin{shell}
gh project item-edit \
  --id $issue_id \
  --field-id ${{ vars.RPOJECT_STATUS_FIELD_ID }} \
  --single-select-option-id ${{ vars.PROJECT_REPO_CREATED_OPTION_ID }} \
  --project-id $project_id
\end{shell}

将创建的版本库的 URL 设置为 URL 字段:

\begin{shell}
gh project item-edit \
  --id $issue_id \
  --field-id ${{ vars.PROJECT_URL_FIELD_ID }} \
  --text $REPO_URL \
  --project-id $project_id
\end{shell}

最后,将创建日期字段设置为当前日期:

\begin{shell}
gh project item-edit --id $issue_id \
  --field-id ${{ vars.PROJECT_CREATED_FIELD_ID }} \
  --date $(date +%Y-%m-%d) \
  --project-id $project_id
\end{shell}

现在,可以在 Projects(项目)中跟踪版本库请求的状态(图 5.13)。

\myGraphic{0.6}{content/chapter5/images/13.png}{图 5.13 --- 在 GitHub Projects 中跟踪 IssueOps 的状态}

元数据也可以在每个单独的问题卡片上看到(请参阅图 5.14)。

\myGraphic{0.6}{content/chapter5/images/14.png}{图 5.14 --- GitHub Projects 卡片在 GitHub Issues 中}

GitHub Projects 非常适合跟踪问题状态,但使用 GraphQL API 或 CLI 自动化它并不容易。































