本示例中,将创建一个简单的模板,以后可以扩展它以收集 IssueOps 工作流中的用户输入。

\mySubsubsection{5.2.1}{Getting ready}

我们将把issue模板添加到前几章中使用过的库中,可以本地库并在 Visual Studio Code 中打开,也可以在浏览器中完成此部分。可以按照我库中的示例(\url{https://github.com/wulfland/GitHubActionsCookbook})进行操作。

\mySubsubsection{5.2.2}{How to do it…}

\begin{enumerate}
\item 
在库中创建一个名为 .github/ISSUE\_TEMPLATE/repo\_request.yml 的文件。只要文件是 YAML 或 Markdown 文件,GitHub 就会自动将 .github/ISSUE\_TEMPLATE 文件夹中的文件视为issue模板。

\item 
为模板添加名称和描述:

\begin{tcolorbox}[ breakable,colback = bashcodebg, colframe= black!50!white]
\scriptsize{
name: '\emoji{🗒️} Repository Request'\\
description: 'Request a new repository.'
}
\end{tcolorbox}

\item 
使用默认值预填新问题的标题:

\begin{tcolorbox}[ breakable,colback = bashcodebg, colframe= black!50!white]
\scriptsize{
title: '\emoji{🗒️} Repository Request: '
}
\end{tcolorbox}

\item 
将一个或多个标签应用于新问题:

\begin{shell}
labels:
  - 'repo-request'
  - 'issue-ops'
\end{shell}

注意,这些标签必须存在于仓库中。如果不可用,请创建它们(使用 gh label list 进行检查):

\begin{shell}
$ gh label create repo-request
$ gh label create issue-ops
\end{shell}

还可以提供描述或明确的颜色字符串:

\begin{shell}
$ gh label create repo-request \
> -c=#D541D0 \
> -d="Request a new repository"
\end{shell}

\item 
将问题分配给一个或多个用户或团队,只需使用 GitHub 用户名即可:

\begin{shell}
assignees:
  - wulfland
\end{shell}

\item 
可以自动将新问题分配给一个 GitHub 项目,语法为 \{owner\}/\{project id\}。

\begin{shell}
projects: 'wulfland/19
\end{shell}

请注意,创建问题的人需要对该项目具有写入权限。

如果没有项目,请创建一个新项目。点击右上角的 + 图标,然后选择 \textbf{New project}(新建项目)(见图 5.1)。

\myGraphic{0.6}{content/chapter5/images/1.png}{图 5.1 --- 创建新项目}

选择一个项目模板或从头开始。对于一个简单的管理仓库请求的项目,您可以直接从一个简单的 Board(看板)开始(见图 5.2)。

\myGraphic{0.6}{content/chapter5/images/2.png}{图 5.2 --- 选择模板或从头开始}

为项目命名——例如,Repository Requests,然后点击 \textbf{Create project}(创建项目)。从项目的 URL 中获取项目 ID:https://github.com/users/\{owner\}/projects/\{id\}。

\item 
在表单正文中,可以定义不同的字段。首先,为请求的仓库名称添加一个简单的文本输入框,可以使字段成为必填项,并添加附加标签、默认值或占位符:

\begin{shell}
body:
  - type: input
    id: name
    attributes:
      label: 'Name'
      description: 'Name of the repository in lower-case and
kebab casing.'
      placeholder: 'your-name-kebab'
    validations:
      required: true
\end{shell}

\item 
通常,会有一个部门、区域或团队,用于权限或命名约定。添加一个简单的下拉菜单,从中选择两个示例部门:

\begin{shell}
- type: dropdown
  id: department
  attributes:
    label: 'Department'
    description: 'Pick your department. It will be used as a   prefix for the repository name.'
    multiple: false
    options:
      - dep1
      - dep2
    default: 0
  validations:
    required: true
\end{shell}

\item 
将文件提交到库。

\item 
在 \textbf{Issues | New issue} 下,可以选择模板并点击 \textbf{Get started}(见图 5.3)。

\myGraphic{0.6}{content/chapter5/images/3.png}{图 5.3 --- 使用模板创建issue}

标签、项目和分配对象已自动设置,控件为必填字段,并设置了正确的默认值(见图 5.4)。

\myGraphic{0.6}{content/chapter5/images/4.png}{图 5.4 --- 使用问题模板创建新issue}

填写表单并保存新issue。

\end{enumerate}

\mySubsubsection{5.2.3}{How it works…}

问题(Issue)和拉取请求(Pull Request)模板,是引导用户创建问题或拉取请求的有力工具。
可以通过界面生成模板,使其更易于查找(\url{https://docs.github.com/en/communities/using-templates-to-encourage-useful-issuesand-pull-requests/syntax-for-issue-forms})。模板可以是纯 Markdown,但通过此示例中使用的新自定义模板,也可以创建包含多种表单元素(如 Markdown、文本区域、输入框、下拉菜单和复选框),还可以添加验证并设置默认值。有关 GitHub 表单模式的完整语法,请参阅 \url{https://docs.github.com/en/communities/using-templates-toencourage-useful-issues-and-pull-requests/syntax-for-githubs-formschema}。填写完表单后,数据会以 Markdown 格式添加到问题或拉取请求的正文中。模板仅在用户创建问题或拉取请求时提供支持。之后,编辑时就只是 Markdown 了。

\mySubsubsection{5.2.4}{There’s more…}

GitHub 会在创建新问题时显示 .github/ISSUE\_TEMPLATE 文件夹中的所有有效 Markdown 或 YAML 表单模板。但您可以配置额外的链接到外部系统,并可以配置是否允许空白问题或强制用户选择模板(请参阅图 5.5)。

\myGraphic{0.6}{content/chapter5/images/5.png}{图 5.5 --- 配置模板选择器表单}

要配置模板选择器,请将 config.yml 文件添加到 .github/ISSUE\_TEMPLATE 文件夹中。将 blank\_issues\_enabled 设置为 true 或 false,并将额外的链接添加到 contact\_links 数组中:

\begin{tcolorbox}[ breakable,colback = bashcodebg, colframe= black!50!white]
\scriptsize{
blank\_issues\_enabled: false \\
contact\_links: \\
\hspace*{1em}- name: \emoji{👥} Discussions \\
\hspace*{2.2em}url: https://github.com/wulfland/AccelerateDevOps/discussions/new \\
\hspace*{2.2em}about: Please use discussions for issues that are not a bug, enhancement or feature request.
}
\end{tcolorbox}

更多详细信息,请参考官方文档:\url{https://docs.github.com/en/communities/usingtemplates-to-encourage-useful-issues-and-pull-requests/configuringissue-templates-for-your-repository}


