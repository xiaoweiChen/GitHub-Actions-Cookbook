在这个配方中,你将创建一个简单的问题模板,之后可以扩展这个模板以收集用户输入用于你的 IssueOps 工作流。

\mySubsubsection{5.2.1}{Getting ready}

我们将把问题模板添加到你在前几章中使用的仓库中。你可以将仓库克隆到本地并在 Visual Studio Code 中操作,或者你也可以直接在浏览器中完成这部分操作——这无关紧要。你可以参考我的仓库中的示例(\url{https://github.com/wulfland/GitHubActionsCookbook})。

\mySubsubsection{5.2.2}{How to do it…}

\begin{enumerate}
\item 
创建新文件:在你的仓库中创建一个名为 .github/ISSUE_TEMPLATE/repo\_request.yml 的新文件。只要该文件位于 .github/ISSUE\_TEMPLATE 文件夹内,并且是 YAML 或 Markdown 格式,GitHub 就会自动将其视为一个问题模板。

\item 
添加名称和描述:为模板添加名称和描述。

\myGraphic{0.4}{content/chapter5/images/0.1.png}{}

\item 
预填充新问题的标题:使用默认值预填充新问题的标题。

\myGraphic{0.4}{content/chapter5/images/0.2.png}{}

\item 
应用一个或多个标签给新问题:

\begin{shell}
labels:
  - 'repo-request'
  - 'issue-ops'
\end{shell}

注意,这些标签必须存在于仓库中。如果它们不存在,请创建它们(可以使用 gh label list 来检查):

\begin{shell}
$ gh label create repo-request
$ gh label create issue-ops
\end{shell}

如果需要,你还可以提供描述或显式的颜色字符串:

\begin{shell}
$ gh label create repo-request \
> -c=#D541D0 \
> -d="Request a new repository"
\end{shell}

\item 
将问题分配给一个或多个用户或团队:在这个例子中,只需使用你的 GitHub 用户名即可。

\begin{shell}
assignees:
  - wulfland
\end{shell}

\item 
自动将新问题分配给 GitHub 项目:语法为 \{owner\}/\{project id\}。

\begin{shell}
projects: 'wulfland/19
\end{shell}
\end{enumerate}

\mySubsubsection{5.2.3}{How it works…}

\mySubsubsection{5.2.4}{There’s more…}