
在本示例中,我们将解析上一章中的问题,并在工作流中使用 GitHub CLI 与问题进行交互。

\mySubsubsection{5.3.1}{Getting ready}

需要上一章中的问题模板,可以在 Visual Studio Code 中创建工作流,也可以直接在 GitHub 中创建。

\mySubsubsection{5.3.2}{How to do it…}

\begin{enumerate}
\item 
创建一个新的工作流文件 .github/workflows/issue-ops.yml,并将其命名为 issue-ops:

\begin{shell}
# Issue ops
name: issue-ops
\end{shell}

\item 
为工作流使用 issues 触发器。请注意,我们没有使用 created 或 edited 事件,而是使用了 labeled 事件。这允许用户在修改请求时重新标记问题:

\begin{shell}
on:
  issues:
    types: [labeled]
\end{shell}

\item 
添加一个 issue-ops 作业:

\begin{shell}
jobs:
  issue-ops:
\end{shell}

我们只想在特定标签存在时运行此作业,并向作业添加如下条件:

\begin{shell}
if: ${{ github.event.label.name == 'repo-request' }}
\end{shell}

该作业可以在最新的 Ubuntu 镜像上运行:

\begin{shell}
runs-on: ubuntu-latest
\end{shell}

\item 
要与问题进行交互,工作流需要对问题具有写入权限。要使用 GitHub CLI,还需要对内容具有读取权限:

\begin{shell}
permissions:
  issues: write
  contents: read
\end{shell}

\item 
市场中有一个操作可以帮助解析作为表单创建的问题的主体。添加 zentered/issue-forms-body-parser 并给它一个 id 属性,以便稍后访问输出:

\begin{shell}
steps:
  - name: Issue Forms Body Parser
    id: parse
    uses: zentered/issue-forms-body-parser@v2.0.0
\end{shell}

\item 
接下来,添加主脚本。设置 id 以在作业级别访问输出。还将 GH\_TOKEN 环境变量设置为 GITHUB\_TOKEN。此令牌将由 CLI 用于与问题进行交互:

\begin{shell}
- name: Repository Request Validation
  id: repo-request
  env:
    GH_TOKEN: ${{ github.token }}
  run: |
\end{shell}

\item 
作为第一步,使用 jq 从具有 parse ID 的步骤的输出中读取 name 和 department 的值,并将它们存储在变量中。在 JavaScript 中,可以直接访问对象,但在 bash 脚本中,必须使用 jq:

\begin{shell}
repo_name=$(echo '${{ steps.parse.outputs.data }}' | jq -r '.name.text')
repo_dept=$(echo '${{ steps.parse.outputs.data }}' | jq -r '.department.text')
\end{shell}

将这两个变量与最终名称(我们的例子中,部门是前缀)组合起来:

\begin{shell}
repo_full_name=$repo_dept-$repo_name
\end{shell}

将名称设置为输出,以便在以后的步骤或作业中使用:

\begin{shell}
echo "REPO_NAME=$repo_full_name" >> "$GITHUB_OUTPUT"
\end{shell}

\item 
添加一些验证逻辑。这里我将添加两个示例,可以查看工作流文件中的其余部分:\url{https://github.com/wulfland/GitHubActionsCookbook/blob/main/.github/workflows/issue-ops.yml}。

首先,设置默认消息和退出代码。默认消息由两部分组成:首先,想要提及创建问题的人员,然后根据验证的输出添加消息:

\begin{shell}
mention="@${{ github.event.issue.user.login }}: "
message="Requested repository '$repo_full_name' will be sent for approval."
exitcode=0
\end{shell}

接下来,添加一个验证规则,即名称不能为空:

\begin{shell}
# shall not be empty
if [ -z $repo_full_name ]; then
  message="Repository name is empty.";
  exitcode=1;
fi;
\end{shell}

此外,添加一个验证规则,即只能使用字母数字字符和连字符(如果您希望使用 kebab casing(短横线) 命名法):

\begin{shell}
# shall be alphanumeric and minus only
if [[ "$repo_full_name" =~ [^\-a-zA-Z0-9] ]]; then
  message="Repository name shall be alphanumeric and minus only.";
  exitcode=1;
fi;
\end{shell}

\item 
如果验证失败,请从问题中删除标签,并告诉用户修复问题并重新应用标签:

\begin{shell}
if [ $exitcode -ne 0 ]; then
  gh issue edit ${{ github.event.issue.number }} \
    --remove-label repo-request
  message=$message" Please fix the issue and try again by applying the label 'repo-request' again to the issue.";
fi;
\end{shell}

\item 
最后,在问题下评论消息,并在验证失败时使作业失败:

\begin{shell}
gh issue comment ${{ github.event.issue.number }} \
  -b „$mention $message"
exit $exitcode
\end{shell}

\item 
将 REPO\_NAME 输出设置为步骤中设置的输出。我们将在下一个作业中使用它来创建库:

\begin{shell}
outputs:
  REPO_NAME: ${{ steps.repo-request.outputs.REPO_NAME }}
\end{shell}

\item 
现在,使用表单模板创建一个新问题。首先使用一个无效的名称(例如 my\_repo),看看它如何向问题添加评论(请参阅图 5.6)。

\myGraphic{0.6}{content/chapter5/images/6.png}{图 5.6 --- 工作流与问题的交互}

修复名称(例如改为 my-repo),然后再次将 repo-request 标签应用到问题上。

\end{enumerate}

\mySubsubsection{5.3.3}{How it works…}

来了解一下其工作流程。

\mySamllsectionNoContent{工作流权限和 GITHUB\_TOKEN}

每个工作流作业开始时,GitHub 会自动创建一个唯一的 GITHUB\_TOKEN,可以在工作流中使用它来与 GitHub 进行交互,也可以使用此令牌在工作流作业中进行身份验证。

可以为个人账户,以及组织和库配置默认权限——默认是只读权限。最佳实践是保持只读权限,并在工作流或作业中授予显式权限。

GITHUB\_TOKEN 的权限可以配置为顶层键,以应用于工作流中的所有作业,也可以在特定作业中配置。当特定作业中添加 permissions 键时,该作业中使用 GITHUB\_TOKEN 的所有操作和运行命令,都会获得指定的访问权限。

对于每个可用的范围,您可以分配以下权限之一:read(读)、write(写)或 none(无)。

\begin{myNotic}{Note}
如果为这些范围中的一个指定了访问权限,则所有未指定的范围将自动设置为 none!
\end{myNotic}

还可以一次性设置所有权限。以下设置将所有权限设置为只读:

\begin{shell}
permissions: read-all
\end{shell}

以下将授予所有范围的写入访问权限:

\begin{shell}
permissions: write-all
\end{shell}

最后一个将所有范围设置为 none:

\begin{shell}
permissions: {}
\end{shell}

在我们的示例中,我们需要对问题进行写入权限,并且 CLI 需要对库的读取权限:

\begin{shell}
permissions:
  issues: write
  contents: read
\end{shell}

所有其他权限将自动设置 none。

有关 GITHUB\_TOKEN 和工作流权限的更多信息,请参阅 \url{https://docs.github.com/en/actions/using-jobs/assigning-permissionsto-jobs}。

\mySamllsectionNoContent{步骤和作业输出}

在第 3 章中,介绍了有关环境文件的知识。我们在本示例中使用它们来设置,将在后续作业中使用的输出,还会使用它们来访问表单解析器的数据。

表单解析器是来自市场的操作,可帮助我们访问使用issue模板创建的issue正文。

使用 id 来访问输出:

\begin{shell}
steps:
- name: Issue Forms Body Parser
  id: parse
  uses: zentered/issue-forms-body-parser@v2.0.0
\end{shell}

在 JavaScript(即 GitHub Script 操作)中,可以直接访问 JSON 对象:

\begin{shell}
console.log(data.name.text);
console.log(data.department.text);
\end{shell}

在 bash 中,不能像在 JavaScript 中那样直接访问 JSON 属性。需要使用一个命令行 JSON 处理器(例如 jq),来解析 JSON 字符串并访问其属性:

\begin{shell}
repo_name=$(echo '${{ steps.parse.outputs.data }}' | jq -r '.name.text')
\end{shell}

\mySamllsectionNoContent{使用 GitHub CLI 对issue进行评论}

在第 3 章中,使用 octokit 和 REST API 对问题进行了评论。在本示例中,我们使用 GitHub CLI 来完成这个操作。

为了让 CLI 工作,必须首先检出版本库。还必须在工作流步骤中将 GH\_TOKEN 环境变量设置为工作流令牌:

\begin{shell}
env:
  GH_TOKEN: ${{ github.token }}
\end{shell}

使用 CLI 很简单——可以在评论中使用 @ 加上用户名来提及用户:

\begin{shell}
mention="@${{ github.event.issue.user.login }}: "
message="Requested repository '$repo_full_name' will be sent for approval."

gh issue comment ${{ github.event.issue.number }} \
  -b "$mention $message"
\end{shell}









