
在这个食谱中,我们将解析上一章中的问题,并在工作流程中使用GitHub CLI与问题进行交互。

\mySubsubsection{5.3.1}{Getting ready}

您将需要上一章中的问题模板。您可以在Visual Studio Code中创建工作流程,也可以直接在GitHub中创建。

\mySubsubsection{5.3.2}{How to do it…}

\begin{enumerate}
\item 
创建一个新的工作流程 .github/workflows/issue-ops.yml 并命名为 issue-ops:

\begin{shell}
# Issue ops
name: issue-ops
\end{shell}

\item 
使用工作流程的问题触发器。请注意,我们不是使用创建或编辑事件,而是使用标记。这允许用户在修改请求时重新标记问题:

\begin{shell}
on:
  issues:
    types: [labeled]
\end{shell}

\item 
添加一个 issue-ops 作业:

\begin{shell}
jobs:
  issue-ops:
\end{shell}

我们只想针对特定的标签运行此任务。请向任务中添加如下条件:

\begin{shell}
if: ${{ github.event.label.name == 'repo-request' }}
\end{shell}

该任务可以在最新的 Ubuntu 镜像上运行:

\begin{shell}
runs-on: ubuntu-latest
\end{shell}

\item 
为了与议题(issues)进行交互,工作流需要具有对议题的写权限。同时,为了使用 GitHub CLI,还需要对内容(contents)具有读权限:

\begin{shell}
permissions:
  issues: write
  contents: read
\end{shell}

\item 
在 Marketplace 中有一个动作可以帮助解析以表单形式创建的议题内容。添加 zentered/issue-forms-body-parser 并为其设置一个 id 属性,以便稍后访问输出结果:

\begin{shell}
steps:
  - name: Issue Forms Body Parser
    id: parse
    uses: zentered/issue-forms-body-parser@v2.0.0
\end{shell}

\item 
接下来,添加我们的主脚本。设置 id 以在任务级别访问输出。同时,将 GH\_TOKEN 环境变量设为 GITHUB\_TOKEN。这个令牌将被 CLI 用来与议题进行交互:

\begin{shell}
- name: Repository Request Validation
  id: repo-request
  env:
    GH_TOKEN: ${{ github.token }}
  run: |
\end{shell}

\item 
首先,使用 jq 从 parse 步骤的输出中读取 name 和 department 的值,并将它们存储在变量中。在 JavaScript 中我们可以直接访问对象,但在 Bash 脚本中我们必须使用 jq:

\begin{shell}
repo_name=$(echo '${{ steps.parse.outputs.data }}' | jq -r '.name.text')
repo_dept=$(echo '${{ steps.parse.outputs.data }}' | jq -r '.department.text')
\end{shell}

将两个变量与你的最终名称组合起来(在我们的例子中,部门是前缀):

\begin{shell}
repo_full_name=$repo_dept-$repo_name
\end{shell}

将名称设置为输出,以便在后续步骤或任务中使用:

\begin{shell}
echo "REPO_NAME=$repo_full_name" >> "$GITHUB_OUTPUT"
\end{shell}

\item 
我们将添加一些验证逻辑。这里我提供两个示例,你可以在工作流文件中查看其余部分:\url{https://github.com/wulfland/GitHubActionsCookbook/blob/main/.github/workflows/issue-ops.yml}。

首先,设置默认的消息和退出码。默认消息由两部分组成:首先,我们想提及创建该议题的用户,然后根据验证结果添加相应的消息:

\begin{shell}
mention="@${{ github.event.issue.user.login }}: "
message="Requested repository '$repo_full_name' will be sent for approval."
exitcode=0
\end{shell}

接下来,添加一条验证规则:名称不能为空:

\begin{shell}
# shall not be empty
if [ -z $repo_full_name ]; then
  message="Repository name is empty.";
  exitcode=1;
fi;
\end{shell}

此外,再添加一条验证规则:名称只能使用字母数字字符和连字符(如果你希望使用短横线命名格式):

\begin{shell}
# shall be alphanumeric and minus only
if [[ "$repo_full_name" =~ [^\-a-zA-Z0-9] ]]; then
  message="Repository name shall be alphanumeric and minus only.";
  exitcode=1;
fi;
\end{shell}

\item 
如果验证失败,则从问题中移除标签,并告知用户修复问题并重新添加标签:

\begin{shell}
if [ $exitcode -ne 0 ]; then
  gh issue edit ${{ github.event.issue.number }} \
    --remove-label repo-request
  message=$message" Please fix the issue and try again by applying the label 'repo-request' again to the issue.";
fi;
\end{shell}

\item 
最后,在议题上评论该消息,并在验证失败时使任务失败:

\begin{shell}
gh issue comment ${{ github.event.issue.number }} \
  -b „$mention $message"
exit $exitcode
\end{shell}

\item 
将 REPO\_NAME 输出设置为步骤中定义的输出。我们将在下一个任务中使用它来创建实际的仓库:

\begin{shell}
outputs:
  REPO_NAME: ${{ steps.repo-request.outputs.REPO_NAME }}
\end{shell}

\item 
现在,使用表单模板创建一个新的议题。开始时使用一个无效的名称(例如 my\_repo),看看它是如何向议题添加评论的(参见图 5.6)。

\myGraphic{0.4}{content/chapter5/images/6.png}{图 5.6 --- 从工作流中与议题互动}

将名称修复为有效格式(例如 my-repo),然后再次为该议题添加 repo-request 标签。

\end{enumerate}

\mySubsubsection{5.3.3}{How it works…}

让我们来理解其背后的工作流程。

\mySamllsectionNoContent{工作流权限和 GITHUB\_TOKEN}

在每个工作流任务开始时,GitHub 会自动创建一个唯一的 GITHUB\_TOKEN,你可以在工作流中使用该令牌与 GitHub 进行交互。你可以使用此令牌在工作流任务中进行身份验证。

你可以为个人账户、组织以及仓库配置默认权限——默认权限是只读的。最佳实践是保持这一默认设置即只读,并在工作流或任务中明确授予所需的权限。

GITHUB\_TOKEN 的权限既可以在顶级键中配置,以应用于工作流中的所有任务,也可以在特定任务内进行配置。当你在特定任务内添加权限键时,该任务内使用 GITHUB\_TOKEN 的所有操作和运行命令都将获得你指定的访问权限。这样可以确保最小权限原则,提升安全性。

对于每个可用的范围,你可以分配以下权限之一:read(读)、write(写)或 none(无)。

\begin{myNotic}{Note}
如果你为任何一个范围指定了访问权限,那么所有未明确指定的范围将自动设置为 none!
\end{myNotic}

你也可以一次性设置所有权限。以下命令会将所有权限设置为只读:

\begin{shell}
permissions: read-all
\end{shell}

以下将授予所有范围的写入权限:

\begin{shell}
permissions: write-all
\end{shell}

最后这个示例会将所有范围的权限设置为 none:

\begin{shell}
permissions: {}
\end{shell}

在我们的示例中,我们需要对 issues 有写权限,并且 GitHub CLI 需要对内容(contents)有读权限:

\begin{shell}
permissions:
  issues: write
  contents: read
\end{shell}

所有其他未列出的权限将自动被设置为 none。

\mySamllsectionNoContent{步骤和任务的输出}

在第 3 章《构建 GitHub Action》中,你学习了环境文件(environment files)。我们在本示例中使用它们来设置输出内容,以便在后续的任务中使用。我们也通过环境文件访问表单解析器(form parser)提供的数据。

表单解析器是一个来自 Marketplace 的 Action,它帮助我们解析使用议题模板创建的议题正文内容。我们通过设置 id 来访问它的输出结果:

\begin{shell}
steps:
- name: Issue Forms Body Parser
  id: parse
  uses: zentered/issue-forms-body-parser@v2.0.0
\end{shell}

在 JavaScript 中(例如在 GitHub Script Action 中),你可以直接访问 JSON 对象:

\begin{shell}
console.log(data.name.text);
console.log(data.department.text);
\end{shell}

而在 Bash 中,你不能像在 JavaScript 中那样直接访问 JSON 属性。你需要使用命令行 JSON 处理工具,例如 jq,来解析 JSON 字符串并访问其属性:

\begin{shell}
repo_name=$(echo '${{ steps.parse.outputs.data }}' | jq -r '.name.text')
\end{shell}

\mySamllsectionNoContent{使用 GitHub CLI 在议题上发表评论}

在第 3 章中,你曾使用 octokit 和 REST API 在议题上发表评论。在本示例中,我们将使用 GitHub CLI 来完成相同的操作。

要使 CLI 正常工作,你首先需要检出仓库。你还必须在工作流步骤中将 GH\_TOKEN 环境变量设置为工作流提供的令牌:

\begin{shell}
env:
  GH_TOKEN: ${{ github.token }}
\end{shell}

使用 CLI 非常简单——我们可以利用 @ 加用户名的方式来在评论中提及用户:

\begin{shell}
mention="@${{ github.event.issue.user.login }}: "
message="Requested repository '$repo_full_name' will be sent for approval."

gh issue comment ${{ github.event.issue.number }} \
  -b "$mention $message"
\end{shell}









