如果开始使用 IssueOps 进行自动化,工作流会很快变得非常复杂。最终会在作业和步骤上使用大量的 if 语句。为了保持这些解决方案的可维护性,可以使用组合操作或可重用工作流,不仅可以重用功能,还可以将复杂的工作流分解为更小的部分。

我们已经在之前的章节中介绍了组合操作。在这个示例中,将使用一个可重用工作流来为 IssueOps 解决方案添加删除功能。

\mySubsubsection{5.5.1}{Getting ready}

请确保已完成本章前面的所有示例。

\mySubsubsection{5.5.2}{How to do it…}

\begin{enumerate}
\item 
创建一个新的 delete-repo.yml 工作流文件。

\item 
作为触发器,我们使用 workflow\_call 触发器。这表明该工作流是可重用工作流,定义工作流所需的输入:

\begin{shell}
on:
  workflow_call:
    inputs:
      REPO_NAME:
        description: 'Repository name'
        required: true
        type: string
      ISSUE_USER:
        description: 'User who created the issue'
        required: true
        type: string
      ISSUE_NUMBER:
        description: 'Issue number'
        required: true
        type: number
\end{shell}

\item 
可重用工作流就是一个普通的工作流,可以有一个或多个作业,这些作业也与一个环境相关联:

\begin{shell}
  jobs:
    delete:
      runs-on: ubuntu-latest
      environment: repo-cleanup
      steps:
\end{shell}

可以创建一个新的环境用于删除版本库,并添加 PRIVATE\_KEY、APP\_ID、ORGANIZATION、PROJECT\_OWNER 和 REPO\_OWNER。或者,为了简单起见,可以重用 repo-creation 环境。

\item 
从应用获取令牌进行身份验证,就像我们在 issue-ops 工作流中所做的那样:

\begin{shell}
- name: Create app token
  uses: actions/create-github-app-token@v1.6.2
  id: get-workflow-token
  with:
    app-id: ${{ vars.APP_ID }}
    private-key: ${{ secrets.PRIVATE_KEY }}
    owner: ${{ vars.ORGANIZATION }}
\end{shell}

\item 
接下来,使用提供的令牌删除版本库:

\begin{shell}
- name: Delete repository
  id: delete-repo
  env:
    GH_TOKEN: ${{ steps.get-workflow-token.outputs.token }}
    REPO_NAME: ${{ inputs.REPO_NAME }}
    REPO_OWNER: ${{ vars.REPO_OWNER }}
  run: |
    gh repo delete $REPO_OWNER/$REPO_NAME --yes
    echo "Repository '$REPO_NAME' has been successfully deleted."
\end{shell}

\item 
通知用户并关闭issue:

\begin{shell}
- name: Notify User
  if: ${{ success() }}
  env:
    GH_TOKEN: ${{ github.token }}
    ISSUE_NUMBER: ${{ inputs.ISSUE_NUMBER }}
    ISSUE_USER: ${{ inputs.ISSUE_USER }}
    REPO_NAME: ${{ inputs.REPO_NAME }}
    REPO_OWNER: ${{ vars.REPO_OWNER }}
  run: |
    gh issue comment $ISSUE_NUMBER \
      -b "@$ISSUE_USER: Repository '$REPO_OWNER/$REPO_NAME' has been deleted successfully." \
      --repo ${{ github.event.repository.full_name }}
    gh issue close $ISSUE_NUMBER \
      --repo ${{ github.event.repository.full_name }}
\end{shell}

\item 
如果失败,也会通知用户:

\begin{shell}
- name: Handle Exception
  if: ${{ failure() }}
  env:
    GH_TOKEN: ${{ github.token }}
    ISSUE_NUMBER: ${{ inputs.ISSUE_NUMBER }}
    ISSUE_USER: ${{ inputs.ISSUE_USER }}
  run: |
    gh issue comment $ISSUE_NUMBER \
      -b "@$ISSUE_USER: Repository '$REPO_OWNER/$REPO_NAME' deletion failed. Please contact the administrator." \
      --repo ${{ github.event.repository.full_name }}
\end{shell}

\item 
要使用这个工作流,创建一个名为 handle-issue.yml 的新工作流文件。我们让它运行在标记了标签的问题上,并授予它对问题的写权限:

\begin{shell}
name: Handle Issue
on:
  issues:
    types: [labeled]
permissions:
  contents: read
  issues: write
\end{shell}

\item 
为了解析问题,我们添加一个通用作业,该作业使用我们在 issue-ops 工作流中使用的相同逻辑(在设置输出变量之前直接复制过来):

\begin{shell}
jobs:
  parse-issue:
    runs-on: ubuntu-latest
    outputs:
      REPO_NAME: ${{ steps.repo-request.outputs.REPO_NAME }}
    steps:
      - name: Issue Forms Body Parser
        id: parse
        uses: zentered/issue-forms-body-parser@v2.0.0
      - name: Repository Request Validation
        id: repo-request
        env:
          GH_TOKEN: ${{ github.token }}
        run: |
          repo_name=$(echo '${{ steps.parse.outputs.data }}' | jq -r '.name.text')
          repo_dept=$(echo '${{ steps.parse.outputs.data }}' | jq -r '.department.text')
          repo_full_name=$repo_dept-$repo_name
          echo "REPO_NAME=$repo_full_name" >> "$GITHUB_OUTPUT"
\end{shell}

\item 
然后,添加一个将调用另一个工作流文件的作业。当应用的标签是 delete-repo 时,有条件地执行该作业。使用 with 部分传递仓库名称和其他参数:

\begin{shell}
repo-deletion:
  name: "Delete a repository"
  if: github.event.label.name == 'delete-repo'
  uses: ./.github/workflows/delete-repo.yml
  with:
    REPO_NAME: ${{ needs.parse-issue.outputs.REPO_NAME }}
    ISSUE_USER: ${{ github.event.issue.user.login }}
    ISSUE_NUMBER: ${{ github.event.issue.number }}
    needs: parse-issue
  with:
    REPO_NAME: ${{ needs.parse-issue.outputs.REPO_NAME }}
    ISSUE_USER: ${{ github.event.issue.user.login }}
    ISSUE_NUMBER: ${{ github.event.issue.number }}
  secrets: inherit
\end{shell}

使用 secrets: inherit,可以允许访问父工作流中的所有密钥,而无需将它们全部指定为密钥参数。

\item 
提交并推送您的文件,并将 delete-issue 标签应用于您用来测试仓库创建的问题。批准部署,仓库将删除。

请注意,可重用工作流中作业是如何嵌套在工作流作业部分中的,以及它们在设计器中的显示方式(请参阅图 5.15)。

\myGraphic{0.6}{content/chapter5/images/15.png}{图 5.15 --- 可重用工作流中的嵌套作业}

此外,在删除成功评论之后,会关闭issue。

\end{enumerate}

\mySubsubsection{5.5.3}{How it works…}

可重用工作流是结构化复杂工作流,和重用依赖多个作业或环境的更复杂功能的绝佳方式,但也存在一些限制:

\begin{itemize}
\item 
最多可以连接 4 层嵌套 的工作流。

\item 
单个工作流文件中最多可以调用 20 个可重用工作流,包括从顶层调用者开始的所有嵌套工作流。

\item 
在调用者工作流中于 env 上下文中定义的环境变量不会传递到调用的工作流中。

\item 
同样地,在被调用工作流中定义的 env 环境变量也无法在调用者工作流中访问。
\end{itemize}

在多个工作流中重用变量,请在组织、仓库或环境级别设置它们,并使用 vars 上下文引用它们。

要了解有关可重用工作流的更多信息,请参阅:\url{https://docs.github.com/en/actions/using-workflows/reusing-workflows}。

注意,我没有在删除版本库后更新项目字段。GitHub Projects 也支持可用于在问题关闭时更新字段的作业。

只需导航到项目中的 Workflows(作业),启用 Item closed(项目关闭),并将状态字段的值设置为所需值(请参阅图 5.16)。

\myGraphic{0.6}{content/chapter5/images/16.png}{图 5.16 --- 在 GitHub Projects 中配置作业}

当问题关闭时,这将使状态在我们的案例中自动设置为 Deleted(已删除)。

\mySubsubsection{5.5.4}{There’s more…}

当然,这只是一个起点。要在实际场景中采用此方案,需要将解决方案扩展到以下方面:

\begin{itemize}
\item 
团队的生命周期

\item 
授予团队权限

\item 
为版本库设置基本配置

\item 
为不同解决方案类型提供不同的模板
\end{itemize}

但所使用的核心构建模块是相同的。

还需要重构当前解决方案,将其创建版本库的功能包含为一个可重用工作流,并将其包含在 handle-issue 工作流中。我在库中将其进行了保留,以尽可能降低各个步骤的复杂性,同时提供一个实际的解决方案。



















