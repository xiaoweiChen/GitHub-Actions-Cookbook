随着来自世界各地的数百万开发人员,以各种类型和规模的项目在Github平台上进行进行合作,GitHub不仅仅是托管和共享代码的平台,它已成为开源社区的核心。借助GitHub操作,其拥有自己的工作流平台,可以自动化各种重复的工程任务 - 从从持续集成(CI)和持续部署(CD)到 IssueOps、问题的自动分类和 ChatOps。

本书将展示如何在日常生活中充分利用 GitHub Actions。这是一本实用的书  ---  会让你动手实践,并在每个示例中结合理论进行讲解。

\mySubsectionNoFile{}{适读人群}

如果正在寻找一种学习GitHub动作的实用方法,那么本书将非常适合您,无论您是软件开发人员还是DevOps工程师。如果已经独自发挥了动作,但想了解更多;或有其他CI/CD工具的经验,例如Jenkins或Azure Pipelines;或者对于相关领域从未涉及  ---  没关系,本书会在这方面对您进行帮助。

翻开本书之前,您应该至少了解一种编程或脚本语言、作为版本控制系统的Git,以及包括Docker、Linux和Windows文件系统、认证等知识。

\mySubsectionNoFile{}{本书内容}

第1章,\textit{GitHub Actions工作流},介绍GitHub Actions工作流及其功能,将学习YAML基础、触发工作流的事件和表达式,以及如何从市场中使用GitHub Actions进行任务的自动化。

第2章,\textit{编写与调试工作流},介绍编写工作流的最佳实践:如何使用Visual Studio Code和GitHub Codespaces及各种插件,高效创建、编辑和运行工作流;使用强大的代码检查工具排查错误,在分支中开发工作流并在本地运行。还将会了解如何调试工作流,并启用高级日志记录。

第3章,\textit{构建GitHub Actions},介绍不同类型的GitHub Actions,并了解如何使用输入和输出。将尝试自行编写Docker容器动作、TypeScript动作和复合动作。

第4章,\textit{工作流运行时},介绍了工作流的不同运行时选项。将了解如何使用由GitHub托管的运行器,以及如何在Docker容器和Kubernetes中使用GitHub Actions Controller (GHAC) 进行设置和扩展临时的、自托管的运行器。

第5章,\textit{使用GitHub Actions自动化任务},将展示如何使用Issue-Ops在GitHub内自动化常见任务。将了解如何通过GitHub Apps进行身份验证,使用GITHUB\_TOKEN和工作流权限,使用GitHub CLI自动化任务,使用环境进行审批和检查,以及使用可重用的工作流和组合操作,在不同工作流和库之间共享逻辑。

第6章,\textit{构建和验证代码},主要介绍持续集成(CI)。将了解如何使用同一工作流构建和测试不同版本的代码,使用CodeQL查找代码中的安全漏洞,将软件物料清单(SBOM)添加到发布中,自动化软件版本控制,并使用缓存加快工作流的速度。

第7章,\textit{使用GitHub Actions发布软件},涵盖了持续交付和持续部署(CD)。将了解如何使用OpenID Connect (OIDC) 安全地部署到云端,以及如何将容器部署到Kubernetes  ---  无论是Microsoft Azure Kubernetes服务(AKS)、Google Kubernetes引擎(GKE),还是Elastic Container Services(ECS)。此外,还将了解如何将Dependabot与GitHub Actions结合使用,以完全自动化更新依赖项。

\mySubsectionNoFile{}{环境配置}

\begin{longtable}{|p{3cm}|p{12cm}|} % 设置列宽
\hline
\textbf{软件要求} & \textbf{操作系统} \\ \hline
\endfirsthead

\multicolumn{2}{c}{{\bfseries Table \thetable\ 上一页续}} \\ \hline
\endhead

GitHub & 适用于所有操作系统,需要在 \url{https://github.com} 上拥有一个账户。 \\ \hline
Visual Studio Code & 适用于所有操作系统。可以选择使用 GitHub Codespaces 来完成所有示例,无需配置本地环境。如果想要在本地进行实践,则需要安装 Visual Studio Code(下载地址:\url{https://code.visualstudio.com/download})以及其他一些后续的工具。 \\ \hline
Git & 仅在本地实践时需要。适用于所有操作系统,应该安装最新版本的 Git(至少版本 2.23)。 \\ \hline
Node.js & 仅在本地实践时需要。需要安装最新版本的 Node.js(本书撰写时使用的是版本21)。适用于所有操作系统,可以从这里下载最新版本:\url{https://nodejs.org/en/download/current}。 \\ \hline
Docker & 仅在本地实践时需要。可以在此处获取适用于所有操作系统的 Docker:\url{https://docs.docker.com/get-docker/}。 \\ \hline
Azure 和 Azure CLI & 某些章节需要一个 Azure 账户和 Azure CLI,免费试用版即可(\url{https://azure.microsoft.com/en-us/free})。如果想在本地进行实践,还需要安装 Azure CLI。\\ \hline
\end{longtable}

所有示例都可以使用免费的 GitHub 账户在公共仓库中完成。你可以使用 GitHub Codespaces 在虚拟环境中完成所有操作。这将消耗你每月 120 小时的免费时长(GitHub Pro 用户为 180 小时),请注意这一点。一旦免费时长用尽,你将需要按分钟支付 Codespaces 的使用费用。

\mySubsectionNoFile{}{下载源码}

可以从 GitHub 下载本书的示例代码文件,网址为 \url{https://github.com/PacktPublishing/github-actions-cookbook}。如果代码有更新,将在 GitHub 库中更新。

还有丰富的书籍和视频目录中的其他代码包,可在 \url{https://github.com/PacktPublishing/} 上找到。快来看看吧!
