本书中的所有示例可在 \url{https://github.com} 上完成,这是 GitHub 提供的“软件即服务”(SaaS)版本。注册 GitHub 完全免费,用户可以拥有无限的私有和公共库。

对于开源项目(公共库),GitHub 上几乎所有功能都可以免费使用;但对于私有库,则需要购买付费许可证。公共库中,对于 GitHub Actions的使用并无时间限制,建议将所有示例操作都在公共库中进行--- 否则,可能会迅速耗尽每月仅有 2,000 分钟的免费额度。

GitHub 的定价模型按月进行计费,包含三个层级:免费版(Free)、团队版(Team) 和 企业版(Enterprise)(见图 1.2):

\myGraphic{0.6}{content/chapter1/images/2.png}{图1.2 --- GitHub定价层级}

如前所述,公共库完全免费 -- 包括 GitHub Actions、Packages 以及 Dependabot 和 Secret Scanning 等安全特性。私有库同样免费,但仅限于有限的协作功能,不包括受保护的分支、代码所有者(Codeowners)及一些高级拉取请求(PR)功能。对于私有库,免费版每月有 2,000 分钟的 GitHub Actions 使用额度。

要解锁更多的协作功能,需要获取 团队版 许可证,每个用户每月费用为 4 美元。团队版 方案还包括 3,000 分钟的 GitHub Actions 使用时限。

GitHub Enterprise 方案则提供了所有企业级特性,例如:通过 Security Assertion Markup Language (SAML) 和跨域身份管理系统 (SCIM) 实现的单点登录 (SSO)、企业托管用户和 IP 白名单等。还提供了 50,000 分钟的 GitHub Actions 使用时限 --- 但每位用户每月需支付 21 美元。

\begin{myTip}{GHES 与 GitHub Actions}
如果使用的是 GHES(GitHub Enterprise Server),则无法使用 GitHub 托管的运行器来运行工作流。需要自行提供工作流所需的运行器,并确保其安全性,以及清理其工作流产生的产物(artifacts)。通常情况下,这可以通过 Kubernetes 配合 Actions Runner Controller(ARC – \url{https://github.com/actions/actions-runner-controller})完成。我们将在第 4 章中了解到更多。
\end{myTip}








