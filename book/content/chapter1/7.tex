
可以在库中设置变量和密钥,以便在工作流程中访问。这个示例中,我们将添加这两者并在工作流程中进行访问。

\mySubsubsection{1.7.1}{Getting ready}

在这个示例中,将使用 Web UI 来设置变量和密钥,也可以使用 GitHub CLI(\url{https://cli.github.com})来完成这项工作(需要安装)。但对于这个示例来说,GitHub CLI并不是必需的。

\mySubsubsection{1.7.2}{How to do it…}

\begin{enumerate}
\item 
导航到\textbf{Settings | Secrets and Variables | Actions}(设置|密钥和变量|操作),可查看库中现有的密钥,并且可以在\textbf{Secrets}(密钥)(settings/secrets/actions)和\textbf{Variables}(变量)(settings/variables/actions)选项卡之间切换,参考图 1.20:

\myGraphic{0.7}{content/chapter1/images/20.png}{图1.20 --- 为代码仓库配置密钥和变量}

\item 
点击\textbf{New repository secret}(新建代码仓库密钥)将打开\textbf{New secret}(新建密钥)对话框(settings/secrets/actions/new;请参阅图 1.21):

\myGraphic{0.7}{content/chapter1/images/21.png}{图1.21 --- 添加新密钥}

将 MY\_SECRET 作为密钥名称,将随机单词(例如 Abracadabra)作为密钥,然后点击\textbf{Add secret}(添加密钥),日志中将屏蔽密钥!因此,不要使用可能在其他作业或步骤的输出中出现的常见单词。

\begin{myTip}{密钥和变量的命名约定}
密钥名称不区分大小写,并且只能包含普通字符([a-z] 和 [A-Z])、数字([0-9])和下划线()。它们不能以 GITHUB 或数字开头。

最好是使用大写单词,并通过下划线字符分隔来命名密钥。
\end{myTip}

\item 
重复该过程以\textbf{New repository variable}(创建新的代码仓库变量)(settings/variables/actions/new),并创建一个名为 WHO\_TO\_GREET 且值为 World 的变量。

\item 
打开来自上一个示例的 .github/workflows/MyFirstWorkflow.yml 文件,并点击编辑图标(请参阅图 1.22):

\myGraphic{0.6}{content/chapter1/images/22.png}{图1.22 --- 编辑 MyFirstWorkflow.yml}

将单词World更改为\verb|${{vars.WHO_TO_GREET}}|表达式,并使用\verb|${{ secrets.MY_SECRET }}| 添加秘钥:

\begin{tcolorbox}[ breakable,colback = bashcodebg, colframe= black!50!white]
\scriptsize{
- run: | \\
\hspace*{2em}echo "Hello \$\{\{ vars.WHO\_TO\_GREET \}\} \emoji{👋} from \$\{\{ github.actor \}\}." \\
\hspace*{2em}echo "My secret is \emoji{🤫} '\$\{\{ secrets.MY\_SECRET \}\}'."
}
\end{tcolorbox}

\item 
提交更改。工作流将自动运行,检查工作流程日志中的输出,看起来类似图 1.23:

\myGraphic{0.6}{content/chapter1/images/23.png}{图1.23 --- 日志中密钥和变量的输出}

\end{enumerate}

\mySubsubsection{1.7.3}{There’s more…}

可以通过在以下某个级别定义,来创建用于多个工作流的配置变量:

\begin{itemize}
\item 
组织级别

\item 
库级别

\item 
环境级别
\end{itemize}

这三个级别像一个层次结构:可以通过为相同的键提供新值,来覆盖较低级别的变量或密钥。图 1.24 展示了这个层次结构:

\myGraphic{0.6}{content/chapter1/images/24.png}{图1.24 --- 配置变量和密钥的层次结构}

组织的密钥和变量与代码仓库的密钥和变量工作方式相同。您可以在\textbf{Settings | Secrets and variables | Actions}(设置|密钥和变量|操作)下创建一个密钥或变量。新的组织密钥或变量可以为以下内容设置访问策略:

\begin{itemize}
\item 
所有库(All repositories)

\item 
私有库(Private repositories)

\item 
选定库(Selected repositories)
\end{itemize}

选择“选定库”时,可以授予对单个存储库的访问。除了通过UI设置这些值外,还可以使用GitHub CLI。您可以使用 gh secret 或 gh variable 来创建新条目:

\begin{shell}
$ gh secret set secret-name
$ gh variable set var-name
\end{shell}

系统会提示输入密钥或变量的值,或者从文件中读取该值,将其通过管道传递给命令,或使用 -b 或 -{}-body 选项进行指定:

\begin{shell}
$ gh secret set secret-name < secret.txt
$ gh variable set var-name --body config-value
\end{shell}
