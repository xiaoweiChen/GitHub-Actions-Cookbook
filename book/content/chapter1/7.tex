
您可以在可以在工作流中访问的存储库中设置变量和秘密。在此食谱中,我们将同时添加并在工作流程中访问它们。

\mySubsubsection{1.7.1}{Getting ready}

在此食谱中,我们将使用Web UI设置变量和秘密。您也可以将GitHub CLI(\url{https://cli.github.com/})使用。如果您想尝试一下,则必须安装它。但是,没有必要遵循食谱。

\mySubsubsection{1.7.2}{How to do it…}

\begin{enumerate}
\item 
在您的存储库中,导航到Settings | Secrets and Variables | Actions。您可以在存储库中查看所有现有的秘密,并且可以在Secrets(settings/
secrets/actions)和Variables(settings/variables/actions;请参见图1.20)之间切换选项卡:

\myGraphic{0.6}{content/chapter1/images/20.png}{图1.20 --- 为存储库配置秘密和变量}

\item 
单击New repository secret 将打开New secret 对话框(settings/secrets/actions/new;请参见图1.21):

\myGraphic{0.6}{content/chapter1/images/21.png}{图1.21 --- 添加一个新秘密}

将my\_secret添加为秘密名称和一个随机单词,例如abracadabra作为秘密,然后单击“ Add secret”。秘密将在日志中掩盖!因此,不要使用可能在随机作业或步骤的其他输出中发生的常见单词。

\begin{myTip}{命名秘密和变量的约定}
秘密名称不是对病例敏感的,它们只能包含普通字符([A-Z]和[A-Z]),数字([0-9] )和下划线(\_)。他们一定不能从github\_或一个数字开始。

最好的做法是用大写单词命名秘密,被下划线角色隔开
\end{myTip}

\item 
重复New repository variable(settings/variables/actions/new)的过程,并创建一个具有值世界的WHO\_TO\_GREET变量。

\item 
从上一个食谱打开.github/workflows/myfirstworkflow.yml文件,然后单击“编辑”图标(请参见图1.22):

\myGraphic{0.6}{content/chapter1/images/22.png}{图1.22 --- 编辑myfirstworkflow.yml}

将单词World更改为\verb|${{vars.WHO_TO_GREET}}|表达式,并使用\verb|${{ secrets.MY_SECRET }} |添加新行:

\begin{tcolorbox}[ breakable,colback = bashcodebg, colframe= black!50!white]
\scriptsize{
- run: | \\
\hspace*{2em}echo "Hello \$\{\{ vars.WHO\_TO\_GREET \}\} \emoji{👋} from \$\{\{ github.actor \}\}." \\
\hspace*{2em}echo "My secret is \emoji{🤫} '\$\{\{ secrets.MY\_SECRET \}\}'."
}
\end{tcolorbox}

\item 
提交更改。工作流将自动运行。检查工作流日志中的输出。看起来应该像图1.23:

\myGraphic{0.6}{content/chapter1/images/23.png}{图1.23 --- 日志中秘密和变量的输出}

\end{enumerate}

\mySubsubsection{1.7.3}{There’s more…}

您可以通过在以下级别之一上定义它们来创建配置变量,以在多个工作流中使用:

\begin{itemize}
\item 
组织级别

\item 
存储库级别

\item 
环境水平
\end{itemize}

这三个级别像层次结构一样工作:您可以通过为同一键提供新值来覆盖较低级别的变量或秘密。图1.24说明了层次结构:

\myGraphic{0.6}{content/chapter1/images/24.png}{图1.24 --- 配置变量和秘密的层次结构}

组织的秘密和变量的工作方式与存储库相同。您可以在Settings | Secrets and variables | Actions下创建一个秘密或变量。新的组织秘密或变量可以具有以下访问策略:

\begin{itemize}
\item 
所有存储库(All repositories)

\item 
私人存储库(Private repositories)

\item 
选定的存储库(Selected repositories)
\end{itemize}

选择"选定的存储库"时,您可以授予对单个存储库的访问。

除了通过UI设置这些值外,还可以使用GitHub CLI。

您可以使用GH Secret或GH变量创建新条目:

\begin{shell}
$ gh secret set secret-name
$ gh variable set var-name
\end{shell}

您将提示您的秘密或变量值,或者您可以从文件中读取值,将其输送到命令,或将其指定为正文(-b或-{}-body):

\begin{shell}
$ gh secret set secret-name < secret.txt
$ gh variable set var-name --body config-value
\end{shell}
