
GitHub 在工作流程设计器中会指导用户如何编写工作流,可以使用工作流编辑器编写工作流。所以,最好的办法是,编写自己的工作流,并熟悉这个平台。

\mySubsubsection{1.6.1}{Getting ready}

创建第一个工作流程之前,必须先在 GitHub 上创建一个库,导航到\url{https://github.com/new}。如果尚未认证,请进行认证,并填写如图 1.6 所示的数据:

\myGraphic{0.6}{content/chapter1/images/6.png}{图1.6 --- 创建一个新库}

选择 GitHub 用户作为所有者,并设置名称——例如,ActionsCookBook。将其设为公共库,以便所有工作流程和存储都免费。使用 README 文件初始化仓库——这样,库中就已经有了文件,工作流也可以进行操作了。

\mySubsubsection{1.6.2}{How to do it…}

GitHub Action 工作流程是位于仓库的 .github/workflows 文件夹中,扩展名为 .yml 或 .yaml 的 \textbf{YAML} 文件。可以手动创建该文件,这样的话工作流编辑器只有在第一次提交后才能工作,我建议从菜单中创建一个新的工作流。

\begin{enumerate}
\item 
在新仓库中,点击到\textbf{Actions}。由于仓库是新的,还没有任何工作流,所以会重定向到\textbf{Create new workflow}(创建新的工作流程)页面(actions/new)。如果库包含工作流,可以在这里看到这些工作流(如图 1.16 所示),需要点击\textbf{New workflow}(新工作流程)按钮才能到达该页面。

在这个页面上,有很多可以作为起点使用的模板工作流。适配大多数云的部署、适配大多数语言的 CI、代码安全扫描、通用自动化,以及将内容部署到 GitHub Pages 的启动工作流。

本示例中,我们将专注于熟悉编辑器,并通过点击\textbf{set up a workflow yourself}(自己设置工作流程)(见图 1.7)从头创建一个工作流:

\myGraphic{0.6}{content/chapter1/images/7.png}{图1.7 --- 在 GitHub 中创建一个新的工作流}

\item 
GitHub 会在默认分支的 .github/workflows 中创建一个新的 main.yml 文件,并在Web编辑器中显示。编辑器的右侧是文档,可以在 GitHub Marketplace 中搜索 actions。在编辑器中,可以使用 Ctrl + Space(或 Option + Space)自动完成。编辑器会捕获 Tab 键,并且默认情况下将其用于两个空格的缩进。要使用 Tab 键导航到页面上的其他控件,必须使用 Esc 或 Ctrl + Shift + M 先进行退出。

将文件名修改为 MyFirstWorkflow.yml(见图 1.8):

\myGraphic{0.7}{content/chapter1/images/8.png}{图1.8 --- GitHub Actions 的工作流编辑器}

\item 
在编辑器中,点击 Ctrl + Space(或 Option + Space)查看在工作流文件中有效的根元素列表(见图 1.9):

\myGraphic{0.7}{content/chapter1/images/9.png}{图1.9 --- 编辑器显示工作流文件中某个级别的有效选项}

\item 
可在文件顶部添加注释,并使用自动完成设置 name 属性。请注意,编辑器具有错误检查功能,并指示当前缺少所需的根键 on(见图 1.10):

\myGraphic{0.7}{content/chapter1/images/10.png}{图1.10 --- 代码编辑器中的错误检查}

通常,工作流以 name 属性开始,该属性在工作流 UI 中设置显示名称。在文件顶部添加注释以总结工作流的意图,是一个好习惯。

\item 
接下来,配置应触发工作流的事件。请注意,工作流程可以有多个触发器,可根据在工作流设计器中的位置自动完成,并得到出不同的结果。如果与 on: 在同一行,将得到 JSON 语法的结果(参见 YAML 集合类型部分);也就是,\verb|on: [push]|。

如果在第一个元素后面添加一个逗号,然后再次点击 Control + Space,则可以从自动完成中选择其他元素(见图 1.11):

\myGraphic{0.7}{content/chapter1/images/11.png}{图1.11 --- 自动补全也适用于方括号内}

每个触发器都是一个映射,并且可以包含附加参数。如果将光标放在 on: 下面的行上并添加两个空格的缩进,自动完成将使用完整的 YAML 语法给出结果,还会给出可用于配置每个触发器的属性(见图 1.12):

\myGraphic{0.4}{content/chapter1/images/12.png}{图1.12 --- 自动完成有助于触发器的选项}

请注意,大多数参数 - 例如,分支或路径 - 都是序列,如果不使用 JSON 语法,则需要每个条目使用一个短划线。

我们希望测试工作流程在每次推送到 main 分支时运行,还希望能够手动进行触发(请参阅触发工作流的事件部分)。工作流触发器代码好似如下所示:

\begin{shell}
on:
  push:
    branches:
      - main
  workflow_dispatch:
\end{shell}

\begin{myNotic}{通配符}
* 字符可以用作路径中的通配符,** 用作递归通配符。* 是 YAML 中的特殊字符,所以需要使用引号:

\begin{shell}
push:
  branches:
    - 'release/**'
  paths:
    - 'doc/**'
\end{shell}
\end{myNotic}

\item 
配置好工作流程的触发器后,下一步是添加另一个根元素:作业(jobs)。作业在 YAML 中是一个映射 - 所以下一行使用两个空格缩进,自动完成将不起作用,编辑器期望设置一个名称。可将作业命名为 first\_job,并转到下一行。作业对象的名称只能包含字母数字值、短划线(-)和下划线(\_)。如果希望在工作流程中显示其他字符,可以使用 name 属性:

\begin{shell}
jobs:
  first_job:
    name: My first job
\end{shell}

\item 
每个作业都需要一个执行运行器。运行器通过标签标识,将在第 4 章中了解更多关于运行器的信息。我们希望工作流在 GitHub 提供的最新版本的 Ubuntu 运行器上执行,因此使用 ubuntu-latest 标签:

\begin{shell}
runs-on: ubuntu-latest
\end{shell}

\item 
作业由一系列依次执行的步骤组成。最基本的步骤是 run: 命令,将执行一个命令行命令:

\begin{shell}
steps:
  - name: Greet the user
    run: echo "Hello world"
    shell: bash
\end{shell}

name 是可选的,并设置步骤在日志中的输出。shell 也可选,并在非 Windows 平台上默认为 bash,可再回退到 sh。Windows 上,默认是 PowerShell Core(pwsh),可再回退到 cmd。可以使用 \{0\} 占位符,为步骤输入配置 shell(例如,shell: perl \{0\})。

要添加变量输出,可以使用写在 \$\{\{ 和 \}\} 之间的表达式。在表达式中,可以使用来自上下文对象的值,例如 GitHub 上下文。请注意,自动完成也适用于这些上下文对象(见图 1.13):

\myGraphic{0.6}{content/chapter1/images/13.png}{图1.13 --- 自动完成也适用于上下文对象}

从值列表中选择 actor:

\begin{tcolorbox}[ breakable,colback = bashcodebg, colframe= black!50!white]
\scriptsize{
- run: echo "Hello world \emoji{👋} from \$\{\{ github.actor \}\}."
}
\end{tcolorbox}

可参考表达式(\url{(https://docs.github.com/en/actions/learn-github-actions/expressions})和上下文(\url{https://docs.github.com/en/actions/learn-github-actions/contexts})的在线文档。

\item 
YAML 允许编写多行脚本,而无需与引号和换行符作斗争。只需在 run: 后面添加管道操作符(|),并在下一行使用四个空格缩进编写脚本。YAML 将其视为一个块,直到下一个元素 - 即使其中包含新行和空行:

\begin{tcolorbox}[ breakable,colback = bashcodebg, colframe= black!50!white]
\scriptsize{
- run: | \\
\hspace*{2em}echo "Hello world \emoji{👋} from \$\{\{ github.actor \}\}." \\
\hspace*{2em}echo "Current branch is '\$\{\{ github.ref \}\}'."
}
\end{tcolorbox}

\item 
GitHub Actions 工作流程不会自动下载库中的代码。如果想对代码仓库中的文件进行操作,必须先checkout内容。这可以使用 GitHub Action 完成 - 一个可重用的流程步骤,可以轻松地与多个流程共享。

在工作流编辑器的右侧是市场,可以直接在那里搜索 Action。搜索检出(checkout)并找到来自 actions 的 Action(这些是 GitHub 的内置 Action)。列表包含一个安装部分,可以将其复制到工作流中以使用该 Action(见图 1.14):

\myGraphic{0.6}{content/chapter1/images/14.png}{图1.14 --- 工作流程编辑器中的市场列表}

请注意,许多参数是可选的。为了checkout代码库,只需要以下几行信息即可:

\begin{shell}
- name: Checkout
  uses: actions/checkout@v4.1.0
\end{shell}

\begin{myNotic}{使用GitHub Actions}
指向 GitHub 上的一个位置。语法是 \{path\}@\{ref\}。路径指向 GitHub 上的物理位置,如果 Actions 在代码仓库的根目录中,则为 \{owner\}/\{repo\},如果 Actions 在子文件夹中,则为 \{owner\}/\{repo\}/\{path\}。@\{ref\} 之后的引用指向提交的 git 引用,可以是一个标签、分支或单个提交 SHA 码。
\end{myNotic}

\item 
为了在检出后显示库中的文件,需要添加一个步骤:

\begin{shell}
- run: tree
\end{shell}

这将以树形结构输出代码库中的文件。

\item 
要运行工作流程,只需将工作流程文件提交到 main 分支。点击\textbf{Commit changes…}(提交更改…),保留提交信息和分支,并在对话框中点击\textbf{Commit changes}(提交更改)以完成操作(见图 1.15):

\myGraphic{0.6}{content/chapter1/images/15.png}{图1.15 --- 提交工作流文件}

\item 
已经为主分支设置了一个推送触发器,所以提交已经自动触发了工作流程。如果现在导航到库中的操作,将能够看到工作流和最新的工作流运行的情况(见图 1.16):

\myGraphic{0.7}{content/chapter1/images/16.png}{图1.16 --- 操作的默认视图显示所有工作流程的最新运行情况}

请注意,工作流程运行的名称是提交信息,还可以看到触发工作流程的提交和推送更改的操作者。

\item 
点击工作流程运行以查看更多详细信息,工作流摘要页面包含左侧的作业和一个右侧的可视化表示(见图 1.17),包含触发器、状态和持续时间等元数据:

\myGraphic{0.7}{content/chapter1/images/17.png}{图1.17 --- 工作流摘要页面}

\item 
点击作业以查看更多详细信息。在工作流程日志中,可以检查各个步骤。请注意,工作流文件的每一行都有一个可点击的数字 - 那是一个 URL,可以用来识别每一行。设置作业步骤是个特殊步骤,提供了有关工作流运行者和工作流程权限的许多背景信息(见图 1.18)。检查工作流程所有步骤的输出:

\myGraphic{0.7}{content/chapter1/images/18.png}{图1.18 --- 单个作业工作流的日志}

\item 
作为最后一步,可以手动触发工作流程,以便也查看工作流程运行中的差异。返回操作,选择左侧的工作流程(见图 1.19),然后运行工作流程:

\myGraphic{0.6}{content/chapter1/images/19.png}{图1.19 --- 通过UI手动触发工作流}

检查新的工作流及其输出。

\end{enumerate}

\mySubsubsection{1.6.3}{How it works…}

工作流文件是位于代码仓库 .github/workflows 文件夹中的 YAML 文件。

\mySamllsectionNoContent{YAML基础知识}

YAML 是“YAML Ain’t Markup Language”的缩写,是一种数据序列化语言,经过优化,可以直接由人类编写和阅读。它是 JSON 的一个严格超集,但使用语法上重要的换行符和缩进代替了大括号。

可以在文本前加上井号(\#)来编写注释。

YAML 中,可以使用以下语法将值分配给变量:key: value。

key 是变量的名称。根据 value 的数据类型,变量的类型将有所不同。请注意,键和值可以包含空格,不需要引号!只有在使用某些特殊字符或想强制某些值为字符串时才添加,可以使用单引号或双引号引用键和值。双引号使用反斜杠作为转义模式\verb|("Foo \"bar \" foo ")|,而单引号则使用单引号\verb|('Foo''bar''foo')|。

\mySamllsectionNoContent{YAML集合类型}

在 YAML 中,有两种不同的集合类型:嵌套类型称为映射(map)和列表(list)- 也称为序列(sequence)。映射使用两个空格的缩进:

\begin{shell}
parent_type:
  key1: value1
  key2: value2
  nested_type:
    key1: value1
\end{shell}

一个序列是一个有序的项目列表,每行前都有一个减号:

\begin{shell}
sequence:
  - item1
  - item2
  - item3
\end{shell}

由于 YAML 是 JSON 的一个超集,也可以使用 JSON 语法在一行中放置集合:

\begin{shell}
key: [item1, item2, item3]
key: {key1: value1, key2: value2}
\end{shell}

\mySamllsectionNoContent{触发工作流的事件}

工作流程有三种类型的触发器:webhook 触发器、定时触发器和手动触发器。

\textbf{Webhook 触发器}根据 GitHub 中的事件启动工作流程。有许多可用的 webhook 触发器,例如:可以在 issues 事件、代码仓库事件或 discussions 事件上运行工作流程。我们示例中的 push 触发器是一个 webhook 触发器。

\textbf{定时触发器}可以在多个预定时间运行工作流程,语法与用于 cron 作业的语法相同:

\begin{shell}
on:
  schedule:
    # Runs at every 15th minute
    - cron: '*/15 * * * *'
    # Runs every hour from 9am to 5pm
    - cron: '0 9-17 * * *'
    # Runs every Friday at midnight
    - cron: '0 2 * * FRI'
\end{shell}

\textbf{手动触发器}允许手动启动工作流程。workflow\_dispatch 触发器将使用 Web UI 或 GitHub CLI 启动工作流程。可以使用 inputs 属性为此触发器定义输入参数,repository\_dispatch 触发器可用于使用 API 触发工作流程。此触发器也可以通过某些事件类型进行过滤,并可在工作流中访问的 JSON 的有效信息。

要了解更多关于触发器的信息,请查看文档\url{https://docs.github.com/en/actions/anction/lusther-work flows/events-th-trigger-workflows}。

\mySamllsectionNoContent{工作}

每个作业都需要一个执行运行器,运行器通过标签标识。我们的示例中,使用 ubuntu-latest 标签。这意味着我们的作业将在 GitHub 托管的最新 Ubuntu 镜像上执行。

\mySamllsectionNoContent{使用 GitHub Actions}

Actions 指向 GitHub 上的一个位置。语法是 \{path\}@\{ref\}。路径指向 GitHub 上的物理位置,如果 Actions 在代码仓库的根目录中,则为 \{owner\}/\{repo\},如果 Actions 在子文件夹中,则为 \{owner\}/\{repo\}/\{path\}。@\{ref\} 之后的引用指向提交的 git 引用,可以是一个标签、分支或单个提交 SHA 码。

\begin{shell}
# Reference a version using a tag
- uses: actions/checkout@v4.1.0

# Reference the current head of a branch
- uses: actions/checkout@main

# Reference a specific commit
- uses: actions/checkout@8e5e7e5ab8b370d6c329ec480221332ada57f0ab
\end{shell}

对于同一代码仓库中的本地 actions,如果仅检出库,则可以省略引用。

如果 action 已定义输入,可以使用 with 属性进行指定:

\begin{shell}
- uses: ActionsInAction/HelloWorld@v1
  with:
    WhoToGreet: Mona
\end{shell}

输入为可选或必需,还可以使用 env 属性为步骤设置环境变量:

\begin{shell}
- uses: ActionsInAction/HelloWorld@v1
  env:
    GITHUB_TOKEN: ${{ secrets.GITHUB_TOKEN }}
\end{shell}

\mySubsubsection{1.6.4}{There’s more…}

这只是一个非常基础的工作流程,使用一个 action 来检出代码并在命令行上运行一些命令。接下来的两个示例中,将展示如何使用密钥、变量和保护环境来实现更复杂的工作流程。