因为使用自己的计算资源,所以自我托管运行器上运行工作流完全免费。公共库中运行工作流也免费——即使在 GitHub 也是如此。GitHub 托管的运行器在 Linux、Windows 和 macOS 上可用,并且有不同的规格。如果想在私有库中使用这些运行器,将按分钟计费。不同的运行器使用不同的分钟乘数(见表 1.1)。在 Linux 上运行工作流程将每分钟消耗 1 分钟的免费时长——如果超出免费时长,则需支付 \$0.008。

Windows 将以两倍的速度消耗免费时长,之后每分钟费用为 \$0.08。macOS 将以十倍的速度消耗时长,当达到包含的分钟数限制时,每分钟收费\$0.016

% Please add the following required packages to your document preamble:
% \usepackage{longtable}
% Note: It may be necessary to compile the document several times to get a multi-page table to line up properly
\begin{longtable}{|l|l|l|}
\hline
\textbf{操作系统} & \textbf{分钟倍率} & \textbf{每分钟价格} \\ \hline
\endfirsthead
%
\multicolumn{3}{c}%
{{\bfseries Table \thetable\ 上一页续}} \\
\endhead
%
Linux                     & 1                          & \$0.008                   \\ \hline
Windows                   & 2                          & \$0.080                   \\ \hline
macOS                     & 10                         & \$0.016                   \\ \hline
\end{longtable}

\begin{center}
表1.1 --- GitHub 托管运行器的每分钟定价
\end{center}

这就是为什么本书中的大多数示例中使用 Linux,并且我会鼓励我的客户尽可能在 Linux 上运行更多任务的原因。

如果使用 GHEC 或团队计划,并且需要功能更强大的机器,可以使用更大的 GitHub 托管运行器。它们按分钟计费(见表 1.2),并具有静态 IP 范围等功能:

% Please add the following required packages to your document preamble:
% \usepackage{longtable}
% Note: It may be necessary to compile the document several times to get a multi-page table to line up properly
\begin{longtable}{|l|l|l|l|}
\hline
\textbf{虚拟CPU数量} & \textbf{Linux} & \textbf{Windows} & \textbf{macOS} \\ \hline
\endfirsthead
%
\multicolumn{4}{c}%
{{\bfseries Table \thetable\ 上一页续}} \\
\endhead
%
2              & \$0.008        & \$0.016          &                \\ \hline
3              &                &                  & \$0.08         \\ \hline
4              & \$0.016        &                  &                \\ \hline
8              & \$0.032        & \$0.064          &                \\ \hline
12             &                &                  & \$0.32         \\ \hline
16             & \$0.064        & \$0.128          &                \\ \hline
32             & \$0.128        & \$0.256          &                \\ \hline
64             & \$0.256        & \$0.512          &                \\ \hline
\end{longtable}

\begin{center}
表1.2 --- 更大运行器的每分钟费率
\end{center}

\begin{myTip}{私有网络}
除了静态 IP 范围,还可以使用 Azure 私有网络直接将 GitHub 托管的运行器连接到相关资源。在撰写本文时,此功能仍处于测试阶段,可能会发生变化。有关更多信息,请参阅以下链接: \url{https://docs.github.com/en/enterprise-cloud@latest/admin/configuration/configuringprivate-networking-for-hosted-compute-products/about-networkingfor-hosted-compute-products}
\end{myTip}

GitHub Actions 还会消耗存储空间 --- 例如,用于日志、工作流程构件或缓存。如果超出包含的存储空间,将收取每天每 GB \$0.008 的费用。

价格可能会变化,并参考 GitHub 文档以获取最新信息(\url{https://docs.github.com/en/billing/managing-billing-forgithub-actions/about-billing-for-github-actions})。

要学习 GitHub Actions 并尝试工作流程 --- 只需在公共库中执行所有操作,则无需支付计算或存储费用。