
环境用于描述一般的部署目标,例如:开发、测试、暂存或生产。可以使用保护规则来保护环境,并且可以为特定环境提供配置变量和密钥。

\mySubsubsection{1.8.1}{Getting ready}

我们将首先使用 Web UI 创建一些环境,并添加一些保护规则、密钥和变量。然后,将它们添加到现有的工作流中。

\mySubsubsection{1.8.2}{How to do it…}

\begin{enumerate}
\item 
导航到\textbf{Settings | Environments}(设置|环境)并点击\textbf{New environment}(新环境)(见图 1.25):

\myGraphic{0.6}{content/chapter1/images/25.png}{图1.25 --- 库中管理环境}

输入名称 Production 并点击\textbf{Configure environment}(配置环境)(见图 1.26):

\myGraphic{0.6}{content/chapter1/images/26.png}{图1.26 --- 创建新环境}

\item 
将自己添加为必需的审查者,并点击\textbf{Save protection rule}(保存保护规则)(见图 1.27):

\myGraphic{0.6}{content/chapter1/images/27.png}{图1.27 --- 配置部署保护规则}

\item 
在\textbf{Deployment branches and tags}(部署分支和标签)下,选择\textbf{Selected branches and tags}(选定的分支和标签),点击加号,并为 main 分支添加一个名称模式(见图 1.28):

\myGraphic{0.7}{content/chapter1/images/28.png}{图1.28 --- 配置部署分支和标签}

\item 
在\textbf{Environment secrets}(环境密钥)下,点击\textbf{Add secret}(添加密钥)并添加一个新的 MY\_SECRET 密钥,其值为 Open Sesame(见图 1.29)。重复此操作,使用\textbf{Add variable}(添加变量)添加一个 WHO\_TO\_GREET 变量,其值为 Production users:

\myGraphic{0.7}{content/chapter1/images/29.png}{图1.29 --- 向环境添加密钥和变量}

\item 
重复步骤 1 并创建两个附加环境,Test 和 Load-Test。我们将在接下来的步骤中使用这些环境来展示如何并行执行作业。不必配置部署分支或必需的审查者,只需添加一个 WHO\_TO\_GREET 变量并设置相应的值即可。结果应该类似于图 1.30:

\myGraphic{0.6}{content/chapter1/images/30.png}{图1.30 --- 设置多个环境}

\item 
现在,返回工作流程文件并编辑。在 first\_job 下面添加一个名为 Test 的新作业,在最新的 Ubuntu 镜像上运行。我们将此作业与 Test 环境关联,为了在 first\_job 之后运行此作业,可以使用 needs 属性,并将其设置为所依赖的作业:

\begin{shell}
Test:
  runs-on: ubuntu-latest
  environment: Test
  needs: first_job
\end{shell}

为了查看密钥如何被环境所覆盖,可以使用一个小技巧。由于 GitHub 会搜索日志输出的密钥值对其进行屏蔽,所以必须修改实际文本。例如,可以使用 \verb|sed 's/./& /g'|命令来完成此操作。这将把空格添加到密钥的每个字符之间。通过这个小技巧,Test 作业的步骤如下所示:

\begin{tcolorbox}[ breakable,colback = bashcodebg, colframe= black!50!white]
\scriptsize{
- run: | \\
\hspace*{2em}echo "Hello \$\{\{ vars.WHO\_TO\_GREET \}\} \emoji{👋} from \$\{\{ github.actor \}\}." \\
\hspace*{2em}sec=\$(echo \$\{\{ secrets.MY\_SECRET \}\} | sed 's/./\& /g') \\
\hspace*{2em}echo "My secret is \emoji{🤫} '\$sec'."
}
\end{tcolorbox}

\item 
接下来,添加一个新的 Load-Test 作业,将其与 Load-Test 环境关联,并在 first\_job 之后执行:

\begin{shell}
Load-Test:
  runs-on: ubuntu-latest
  environment: Load-Test
  needs: first_job
\end{shell}

只需复制 Test 中的步骤即可,无需进行更改。

\item 
最后一个作业是 Production 作业。除了名称之外,environment 属性还接受一个 URL,该 URL 将在工作流程设计器中显示。将其设置为相应的 URL。为了展示在工作流程并行执行作业后如何再次合并,我们将在 Test 和 Load-Test 之后运行 Production:

\begin{shell}
Production:
  runs-on: ubuntu-latest
  environment:
    name: Production
    url: https://writeabout.net
  needs: [Test, Load-Test]
\end{shell}

只需复制前一个作业的步骤即可。

\item 
将更改提交到 main 分支,工作流程将自动运行。导航到新的工作流程运行并检查工作流程设计器,其很好地展示了并行执行。工作流程在执行 Production 之前将暂停,并等待批准(请参阅图 1.31):

\myGraphic{0.6}{content/chapter1/images/31.png}{图1.31 --- 工作流程将在具有必需审查者的环境之前停止,并等待批准}

\item 
点击\textbf{Review deployment}(审查部署),选中 \textbf{Production} 并添加可选评论。点击\textbf{Approve and deploy}(批准并部署)开始执行 Production 作业(见图 1.32):

\myGraphic{0.6}{content/chapter1/images/32.png}{图1.32 --- 批准受保护的环境}

工作流将完全执行,结果应该看起来像图 1.33。请注意,URL 显示在 \textbf{Production} 环境中。同时,请注意工作流程摘要中的批准历史记录:

\myGraphic{0.6}{content/chapter1/images/33.png}{图1.33 --- 工作流的最终摘要}

打开各个作业并检查添加的步骤的输出(见图 1.34)。密钥和变量来自代码库,并且只有在设置环境时才会进行覆盖:

\myGraphic{0.6}{content/chapter1/images/34.png}{图1.34 --- 生产密钥只有在批准后才对生产环境可用}

\end{enumerate}

\mySubsubsection{1.8.3}{There’s more…}

如果正在使用 GitHub CLI 为环境设置密钥或变量,可以使用 -{}-env (-e) 参数指定。对于组织密钥,可以将可见性 (-{}-visibility 或 -v) 设置为 all、private 或 selected。对于 selected,必须使用 -{}-repos (-r) 指定一个或多个代码库:

\begin{shell}
$ gh secret set secret-name --env environment-name
$ gh secret set secret-name --org org -v private
$ gh secret set secret-name --org org -v selected -r repo
\end{shell}

环境比本指南中使用的选项更多。还可以配置一个等待计时器,在工作流执行特定环境的部署作业之前,将工作流暂停 n 分钟(最多 30 天)。

还有一个名为“自定义部署保护规则”的新功能仍处于测试阶段。此功能允许创建 GitHub 应用程序,该应用程序可以暂停部署并等待特定条件。 Datadog,Honeycomb,Sentry,New Relic和Service Now已经有应用程序(请参阅\url{https://doc.github.com/en/actions/deployment/deployment/protecting-deployments/configuring-configuring-configuring-custom-custom-deployment-protection-protection-protection-protection-protection-protection-protection-protection-custom-custom-customdeployment-prection-prection-prection-es})。

环境保护规则的力量在于部署分支或标签规则,这可以限制不适用于分支保护规则的代码部署到某些环境。可以包括各种检查 - Codeowners 批准、代码审查者、部署到某些其他环境、SonarQube 质量门\footnote{质量门:是一种创新的管理策略,它将产品设计流程中的关键节点设定为严格的检查点,就像一道道严密的门坎。}和其他许多自动代码检查(请参阅\url{https://docs.github.com/en/repositories/configuring-branches-and-merges-inyour-repository/managing-protected-branches/about-protected-branches})。

























