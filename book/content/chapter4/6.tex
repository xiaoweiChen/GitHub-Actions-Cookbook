在本章的最后一个示例中,我们将在运行器组中创建一个具有网络隔离的大型 GitHub 托管运行器。

\mySubsubsection{4.6.1}{Getting ready}

需要在前一个配方中创建的运行器组,该组位于具有付费计划的企业或组织中!

\mySubsubsection{4.6.2}{How to do it…}

\begin{enumerate}
\item 
在你创建的运行器组中,点击\textbf{New runner | New GitHub-hosted runner}(新建运行器 | 新建 GitHub 托管运行器)(见图 4.20):

\myGraphic{0.6}{content/chapter4/images/20.png}{图4.20 --- 创建新的 GitHub 托管运行器}

\item 
给运行器一个名称。名称必须在 1 到 100 个字符之间,并且只能包含大写字母(A-Z)和小写字母(a-z)、数字(0-9)、点(.)、破折号(-)和下划线(\_)。选择一个运行器镜像值(Ubuntu 或 Windows)及其对应的版本(见图 4.21):

\myGraphic{0.6}{content/chapter4/images/21.png}{图4.21 --- 配置运行器的名称和镜像}

\item 
选择运行器的大小(见图 4.22),较大的运行器费用更高:

\myGraphic{0.6}{content/chapter4/images/22.png}{图4.22 --- 为新的运行器选择规模}

\item 
可以限制最大并发作业数。最大值为 500。将其保留在默认值 50,并将其保留在自动设置创建的运行器组中(见图 4.23):

\myGraphic{0.6}{content/chapter4/images/23.png}{图4.23 --- 设置作业并发和运行器组}

\item 
可以通过为运行器分配一个唯一且静态的公共 IP 地址范围来启用网络隔离(见图 4.24)。点击创建运行器以完成该过程:

\myGraphic{0.6}{content/chapter4/images/24.png}{图4.24 --- 为运行器启用网络隔离}

\item 
请注意,供应需要一些时间。当运行器准备就绪,就可以检查与它关联的 IP 范围,此时会看到标签与运行器的名称相同(见图 4.25)。现在,可以开始在更大的运行器上执行工作流了!

\myGraphic{0.6}{content/chapter4/images/25.png}{图4.25 --- 具有网络隔离的大型运行器}

\end{enumerate}

\mySubsubsection{4.6.3}{How it works…}

GitHub 将提供更大的运行器,也能分配一个静态的公共 IP 范围。请注意,必须使用该运行器;如果不使用,GitHub 在一段时间后关闭。网络隔离允许运行器访问本地资源,无需公共访问。