在这个章节的最后一个配方中,我们将创建一个具有网络隔离功能的、更大规模的GitHub托管运行器,并将其加入到运行器组中。

\mySubsubsection{4.6.1}{Getting ready}

你需要在拥有付费计划的企业或组织中使用上一个配方中创建的运行器组!

\mySubsubsection{4.6.2}{How to do it…}

\begin{enumerate}
\item 
在你创建的运行器组中,点击“新建运行器”|“新建GitHub托管运行器”(参见图4.20):

\myGraphic{0.4}{content/chapter4/images/20.png}{图4.20 --- 将运行器分配给运行器组}

\item 
给运行器命名。名称长度必须在1到100个字符之间,且只能包含大写字母(A-Z)、小写字母(a-z)、数字(0-9)、点(.)、短横线(-)和下划线(\_)。选择一个运行器镜像值(Ubuntu或Windows)以及相应的版本(参见图4.21)。

\myGraphic{0.4}{content/chapter4/images/21.png}{图4.21 --- 配置运行器的名称和镜像}

\item 
选择运行器的规模(参见图4.22)。请注意,较大规模的运行器成本更高。关于更大规模运行器的定价详情,请参阅第1章:

\myGraphic{0.4}{content/chapter4/images/22.png}{图4.22 --- 为新的运行器选择规模}

\item 
你可以限制并发作业的最大数量,最大值为500。将其保持在默认的50。同时,保持自动设置的运行器组不变,即开始创建运行器时所在的组(参见图4.23):

\myGraphic{0.4}{content/chapter4/images/23.png}{图4.23 --- 设置作业并发数和运行器组}

\item 
通过给运行器分配一个唯一且静态的公共IP地址范围,你可以启用网络隔离(参见图4.24)。点击“创建运行器”以完成此过程:

\myGraphic{0.4}{content/chapter4/images/24.png}{图4.24 --- 为运行器启用网络隔离}

\item 
注意,资源配置需要一些时间。一旦运行器准备就绪,你就可以检查与其关联的IP地址范围,在这一点上,你会看到标签与运行器的名称相同(参见图4.25)。现在,你可以在更大的运行器上开始执行工作流了!

\myGraphic{0.4}{content/chapter4/images/25.png}{图4.25 --- 具有网络隔离功能的更大规模运行器}

\end{enumerate}

\mySubsubsection{4.6.3}{How it works…}

GitHub将为你配置这些更大规模的运行器,并在你希望的情况下分配一个静态公共IP地址范围。需要注意的是,你需要使用这个运行器;如果没有使用,GitHub将在一段时间后将其关闭。网络隔离使你能够在不需要公开访问的情况下,给予运行器访问本地资源的权限。