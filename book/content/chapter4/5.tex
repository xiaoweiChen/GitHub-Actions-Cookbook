在组织和企业级别,对运行器的访问是通过运行器组来组织的。工作流与运行器之间的关联是通过标签完成的 --- 但运行器组控制着工作流可以访问哪些运行器。

\mySubsubsection{4.5.1}{Getting ready}

请注意,在免费组织中,只有一个名为“Default”的运行器组,你可以用它来添加自托管运行器。如果要创建多个运行器组或将它们用于 GitHub 托管的运行器,则需要付费的团队(Team)或企业(Enterprise)计划。

\mySubsubsection{4.5.2}{How to do it…}

\begin{enumerate}
\item 
在拥有付费计划的组织中,导航到“设置 | Actions | 运行器组”(/settings/actions/runner-groups),并点击“新建组”。

为该组命名。在“仓库访问权限”下,将选择从“所有仓库”更改为“选定的仓库”,并点击齿轮图标来选择一个或多个可以访问该组的仓库(参见图4.15)。请注意,你可以在此处允许访问公共仓库,但此选项默认是禁用的:

\myGraphic{0.4}{content/chapter4/images/15.png}{图4.15 --- 管理对组的仓库访问权限}

\item 
在“工作流访问权限”下,选择“选定的工作流”并点击齿轮图标(参见图4.16):

\myGraphic{0.4}{content/chapter4/images/16.png}{图4.16 --- 限制对特定工作流的访问权限}

\item 
在出现的对话框中,你可以添加多种模式以识别工作流(参见图4.17)。

\myGraphic{0.4}{content/chapter4/images/17.png}{图4.17 --- 限制工作流版本访问的语法}

工作流通过指向工作流文件的路径和一个有效的git引用来指定。该引用可以是分支、标签或SHA值。

\item 
退出对话框,将值改回“所有工作流”,并点击“创建组”。现在你可以向该组添加自托管运行器和GitHub托管的运行器。我们将在下一个配方中介绍GitHub托管的运行器。点击“新建自托管运行器”(参见图4.18):

\myGraphic{0.4}{content/chapter4/images/18.png}{图4.18 --- 向运行器组添加自托管运行器}

\item 
请注意,添加运行器的对话框与设置自托管运行器的对话框相同。你可以通过覆盖RUNNER\_URL来测试你的容器:

\begin{shell}
$ docker run -d --rm -e RUNNER_NAME=Runner_Group \
> -e TOKEN={TOKEN} \
> -e RUNNER_URL=https://github.com/{org} \
> simple-ubuntu-runner
\end{shell}

这将创建位于“Default”组中的运行器!打开该运行器,并将其分配给你创建的新组(参见图4.19):

\myGraphic{0.4}{content/chapter4/images/19.png}{图4.19 --- 将运行器分配给运行器组}

请注意,一个运行器只能被分配到一个组中。要在设置时直接将运行器分配给特定组,你需要在配置脚本中传递相应的参数,这不在上述指令内:

\begin{shell}
$ ./config.sh --url $org_or_enterprise_url --token $token --runnergroup rg-runnergroup
\end{shell}

\end{enumerate}

运行器组是组织和企业用于管理对运行器访问权限的重要功能。此功能非常直观,不需要过多解释 --- 因此这里保持简明扼要。在下一个配方中,我们将使用运行器组来添加更大规模的GitHub托管运行器。