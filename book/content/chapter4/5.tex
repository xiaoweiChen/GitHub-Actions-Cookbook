在组织和企业级别,可按运行器组组织访问运行器。工作流与运行器的关联通过标签完成 - 但运行器组控制工作流可以访问哪些运行器。

\mySubsubsection{4.5.1}{Getting ready}

请注意,在免费组织中,只有一个名为“默认”的运行器组,可以使用它来添加自托管运行器。要创建多个运行器组或用于 GitHub 托管运行器,需要付费的团队或企业计划。

\mySubsubsection{4.5.2}{How to do it…}

\begin{enumerate}
\item 
在拥有付费计划的组织中,导航到\textbf{Settings | Actions | Runner groups}(设置 | 操作 | 运行器组)(/settings/actions/runner-groups),并点击\textbf{New group}(新建组)。

为该组命名。在\textbf{Repository access}(库访问权限)下,将选择从\textbf{All repositories}(所有库)更改为\textbf{Selected repositories}(选定的库),并点击齿轮图标来选择一个或多个可以访问该组的库(参见图4.15)。请注意,可以在此处允许访问公共库,但此选项默认禁用:

\myGraphic{0.6}{content/chapter4/images/15.png}{图4.15 --- 管理库对组的访问权限}

\item 
在\textbf{Workflow access}(工作流访问权限)下,选择\textbf{Selected workflows }(选定的工作流)并点击齿轮图标(参见图4.16):

\myGraphic{0.6}{content/chapter4/images/16.png}{图4.16 --- 限制对某些工作流的访问}

\item 
请注意,在出现的对话框中,可以添加多个模式来识别工作流(见图 4.17):

\myGraphic{0.6}{content/chapter4/images/17.png}{图4.17 --- 限制对工作流版本访问的语法}

工作流通过指向工作流文件的路径和一个有效的 git 引用来指定,该引用可以是分支、标签或 SHA 值。

\item 
退出对话框,将值设置回\textbf{All workflows}(所有工作流程),然后点击\textbf{Create group}(创建组)。现在可以向该组添加自托管运行器和 GitHub 托管运行器,点击\textbf{New self-hosted runner }(新建自托管运行器)(见图 4.18)

\myGraphic{0.6}{content/chapter4/images/18.png}{图4.18 --- 向运行器组添加自托管运行器}

\item 
请注意,添加运行器的对话框与设置自托管运行器配方中出现的对话框相同,可以通过覆盖 RUNNER\_URL 来使用容器进行测试:

\begin{shell}
$ docker run -d --rm -e RUNNER_NAME=Runner_Group \
> -e TOKEN={TOKEN} \
> -e RUNNER_URL=https://github.com/{org} \
> simple-ubuntu-runner
\end{shell}

这会将在默认组中创建运行器!打开运行器并将其分配给你创建的新组(见图 4.19):

\myGraphic{0.6}{content/chapter4/images/19.png}{图4.19 --- 将运行器分配给运行器组}

请注意,运行器只能分配到一个组。为了在设置过程中直接将运行器分配给一个组,必须将参数传递给配置脚本,这在说明中没有:

\begin{shell}
$ ./config.sh --url $org_or_enterprise_url --token $token --runnergroup rg-runnergroup
\end{shell}

\end{enumerate}

对于组织和企业管理运行器的访问权限来说,运行器组是一个重要的功能。这个功能很简单,不需要太多解释。在下一个示例中,我们将使用运行器组来添加更大的 GitHub 托管运行器。