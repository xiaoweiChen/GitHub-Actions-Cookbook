
到目前为止,我们仅使用了 ubuntu-latest 标签来运行我们的作业。这会在 GitHub 托管的最新版 Ubuntu 镜像上运行工作流。不过,GitHub 也提供了 macOS 和 Windows 的运行器,并且有不同的配置。您可以在任何您喜欢的平台上托管自己的运行器。在这个初步的示例中,我们将设置一个位于 Linux Docker 容器中的自托管运行器。通过这种方式,我们可以轻松地扩展它,并在工作流运行后清理资源。

\mySubsubsection{4.2.1}{Getting ready}

你需要安装 Docker 来完成这个示例,并且需要知道你的处理器架构。如果你不清楚自己的处理器架构,只需运行 docker info 并查找 "Architecture:"

\begin{shell}
$ docker info | grep Architecture
\end{shell}

\mySubsubsection{4.2.2}{How to do it…}

\begin{enumerate}
\item 
前往 GitHub 上的一个代码仓库。你可以创建一个新的仓库,或者使用你在第 1 章中创建的 GitHubActionsCookbook 仓库。进入 Settings(设置) > Actions(操作) > Runners(运行器)(路径:/settings/actions/runners),然后点击 New self-hosted runner(新建自托管运行器)(参见图 4.1)。这将引导你开始设置自己的自托管运行器。

\myGraphic{0.4}{content/chapter4/images/1.png}{图4.1 --- 
可以通过进入仓库的 Settings(设置) 区域来添加自托管运行器}

这将重定向你到 /settings/actions/runners/new 页面。在这里,为运行器镜像选择 Linux 并根据你的 Docker 环境设置 Architecture(架构) 属性。

请注意每个平台和处理器架构的不同脚本。你可以复制整个脚本用于虚拟机上的安装。然而,由于我们是在 Docker 容器中操作,我们将需要一些额外的步骤。请注意,你也可以单独复制某一行命令(参见图 4.2)。这种灵活性允许你根据实际需求调整安装过程,确保与你的 Docker 环境兼容。在进行容器化部署时,这些额外步骤可能包括修改 Dockerfile 或者调整容器的启动参数以正确配置自托管运行器。

\myGraphic{0.4}{content/chapter4/images/2.png}{图4.2 --- 
在不同平台上安装自托管运行器的脚本}

\item 
在最新版本的Ubuntu容器中启动控制台:

\begin{shell}
$ docker run -it ubuntu:latest /bin/bash
\end{shell}

\item 
运行脚本的第一行。这将为运行器创建一个文件夹并切换到该目录:

\begin{shell}
$ mkdir actions-runner && cd actions-runner
\end{shell}

\item 
要下载运行器的二进制文件,我们需要在容器中安装 `curl`,因为它不是普通 Ubuntu 镜像的一部分:

\begin{shell}
$ apt-get -y update; apt-get -y install curl
\end{shell}

\item 
下载运行器二进制文件:从浏览器中复制并粘贴对应的 curl 命令到你的控制台来下载最新的运行器包。命令看起来像这样:

\begin{shell}
$ curl -o actions-runner-{version}.tar.gz -L https://{URL}.tar.gz
\end{shell}

\item 
解压下载的包:使用脚本中的 tar 命令解压缩下载的包:

\begin{shell}
tar xzf ./actions-runner-{version}.tar.gz
\end{shell}

\item 
安装运行器所需的依赖:通过执行以下脚本来安装运行器需要的依赖项。

\begin{shell}
$ ./bin/installdependencies.sh
\end{shell}

\item 
允许以 root 身份运行:由于我们的容器默认以 root 用户运行,我们需要允许运行器作为 root 运行。这可以通过设置 RUNNER\_ALLOW\_RUNASROOT 环境变量为非零值来实现:

\begin{shell}
$ export RUNNER_ALLOW_RUNASROOT="1"
\end{shell}

\item 
配置运行器:如果前面的步骤花费了太长时间,你可能需要刷新浏览器页面以获取新的令牌,因为令牌的有效期很短。然后,复制包含令牌的那一行并执行它来配置运行器。

\begin{shell}
./config.sh --url https://github.com/{OWNER}/{REPO} --token {TOKEN}
\end{shell}

按下回车键,并接受所有默认值。

完成上述步骤后,你可以导航回到 Settings > Actions > Runners 查看新注册的运行器。你会看到运行器仍处于离线状态,因为我们还没有启动运行器进程(参见图 4.3)。接下来,你需要启动运行器服务以使它在线并开始接收任务。

\myGraphic{0.4}{content/chapter4/images/3.png}{图4.3 --- 
一个已配置但未运行的运行器会显示为 离线(Offline) 状态}

\item 
使用以下脚本启动运行器:

\begin{shell}
$ ./run.sh
\end{shell}

运行器启动后,它将进入 空闲(Idle) 状态,这意味着它正在等待工作流任务来执行。

\item 
创建一个简单的使用自托管标签的工作流。由于 bash 在所有平台上都可用,因此我们可以省略为 Linux 添加额外的标签。不过,你也可以通过指定 [self-hosted, Linux] 来明确指定运行器需要满足的条件:

\begin{shell}
name: Self-Hosted

on: [workflow_dispatch]

jobs:
  main:
    runs-on: self-hosted
    steps:
      - name: Output environment
      shell: bash
      run: |-
        echo "Runner Name: '${{ runner.name }}'"
        echo "Runner OS: '${{ runner.os }}'"
        echo "Runner ARCH: '${{ runner.arch }}'"
\end{shell}

\item 
执行工作流并监控你的 Docker 容器以查看它是如何执行工作流的。你可以根据需要重复此步骤多次。只要你的容器正在运行,它就会执行所有带有匹配标签的工作流。

\item 
如果你现在停止(kill)你的容器,运行器将在 GitHub 上显示为离线状态。要移除它,请导航回 Settings > Actions > Runners,然后在对应运行器右侧的菜单中选择 Remove runner(参见图 4.4)。

\myGraphic{0.4}{content/chapter4/images/4.png}{图4.4 --- 
从 GitHub 中移除运行器}

运行对话中提供的脚本来移除运行器:

\begin{shell}
$ ./config.sh remove --token {TOKEN}
\end{shell}

运行此命令后,该运行器将从 GitHub 中被移除。

\item 
我们可以进一步配置运行器,使其在执行完一个任务后自动注销自身。这种方法在容器环境中非常有意义。为此,只需在生成新令牌后的配置步骤中添加 -{}-ephemeral 参数即可:

\begin{shell}
./config.sh --url {URL} --token {TOKEN} --ephemeral
\end{shell}

\item 
再次运行工作流,你会发现运行器在任务执行完成后会自动移除。

\item 
接下来,我们可以将所学内容整合到一个 Dockerfile 中(你可以使用以下文件:\url{https://github.com/wulfland/GitHubActionsCookbook/blob/main/SelfHostedRunner/Dockerfile})。通过这种方式,我们可以创建一个可重复使用的 Docker 镜像,每次运行时,该镜像都会自动注册自身、等待任务、执行任务,并在完成后终止。

我们以 ubuntu:latest 为基础镜像,以便简化操作。当然,你也可以轻松将其替换为包含所有构建工具的基础镜像:

\begin{shell}
FROM ubuntu:latest
\end{shell}

设置连接所需的变量。将 TOKEN 和 RUNNER\_NAME 留空,因为这些值将在容器启动时提供,而不是在镜像创建时。同时设置正确的 URL、平台和版本:

\begin{shell}
ENV TOKEN=
ENV RUNNER_NAME=
ENV RUNNER_URL="https://github.com/{owner}/{repo}"
ENV GH_RUNNER_PLATFORM="linux-arm64"
ENV GH_RUNNER_VERSION="2.311.0"
ENV LABELS="self-hosted,ARM64,Linux"
ENV RUNNER_GROUP="Default"
\end{shell}

为了确保操作系统在构建 Docker 镜像的过程中不会提示用户输入,我们需要将 DEBIAN\_FRONTEND 设置为非交互模式:

\begin{shell}
ARG DEBIAN_FRONTEND=noninteractive
\end{shell}

为了减少 Docker 镜像的层数,最好将整个脚本合并为一个 RUN 命令。该脚本更新包管理器及其所有软件包,安装所有依赖项,添加容器运行时使用的 docker 用户(我们不希望容器以 root 身份运行),下载对应的运行器包,解压它,更改文件所有者为 docker 用户,并执行 installdependencies.sh 脚本:

\begin{shell}
RUN apt-get -y update && \
  apt-get upgrade -y && \
  useradd -m docker && \
  apt-get install -y --no-install-recommends curl ca-certificates && \
  mkdir -p /opt/hostedtoolcache /home/docker/actions-runner && \
  curl -L https://github.com/actions/runner/releases/download/v${GH_RUNNER_VERSION}/actions-runner-${GH_RUNNER_PLATFORM}-${GH_RUNNER_VERSION}.tar.gz -o /home/docker/actions-runner/actionsrunner.tar.gz && \
  tar xzf /home/docker/actions-runner/actions-runner.tar.gz -C /home/docker/actions-runner && \
  chown -R docker /home/docker && \
  /home/docker/actions-runner/bin/installdependencies.sh
\end{shell}

在 Dockerfile 中,添加以下内容以确保容器以非特权用户 docker 身份运行,并将工作目录设置为该用户的 /home/docker/actions-runner 文件夹:

\begin{shell}
USER docker
WORKDIR /home/docker/actions-runner
\end{shell}

接下来,确保在容器启动时检查必要的环境变量(如 TOKEN 和 RUNNER\_NAME),并运行 config.sh 脚本完成运行器的注册。之后,启动运行器服务(run.sh)。以下是完整的 CMD 指令:

\begin{shell}
CMD if [ -z "$TOKEN" ]; then echo 'TOKEN is not set'; exit 1; fi
&& \
  if [ -z "$RUNNER_NAME" ]; then echo 'RUNNER_NAME is not set';
exit 1; fi && \
  ./config.sh --url "${RUNNER_URL}" --token "${TOKEN}"
--name "${RUNNER_NAME}" --work "_work" --labels "${LABELS}"
--runnergroup "${RUNNER_GROUP}" --unattended --ephemeral && \
  ./run.sh
\end{shell}

\item 
进入包含 Dockerfile 的文件夹,并使用以下命令构建 Docker 镜像:

\begin{shell}
$ docker build -t simple-ubuntu-runner .
\end{shell}

\item 
现在,您可以使用 docker run 运行任意数量的镜像实例。-d(-{}-detached)选项将以分离模式在后台运行容器。它不会阻塞您的终端,但也不会在终端中接收输入或显示输出。-{}-rm 选项会在容器退出时自动删除容器。使用 -e 选项传递 RUNNER\_NAME 和 TOKEN 参数。请注意,这些名称是区分大小写的!

\begin{shell}
$ docker run -d --rm -e RUNNER_NAME=Runner1 -e TOKEN={TOKEN} simple-ubuntu-runner
\end{shell}

您将在仓库的“设置”(Settings)区域看到运行器,如图 4.5 所示:

\myGraphic{0.4}{content/chapter4/images/5.png}{图4.5 --- 通过多次运行相同的 Docker 镜像来获得临时运行器,这些运行器将等待传入的任务}

\end{enumerate}

启动工作流的次数与您创建的容器数量相同,每个容器将运行一个工作流并在执行完成后清理该容器

\mySubsubsection{4.2.3}{How it works…}

来了解一下流程的工作原理。

\mySamllsectionNoContent{自托管运行器应用程序}

自托管运行器是通过安装开源的运行器应用程序(\url{https://github.com/actions/runner})创建的。该应用程序基于 .NET Core 运行时,可以在大量操作系统和处理器架构上运行。它可以在 macOS 11(Big Sur)或更高版本、Windows(7 至 10 和 Server 2012 R2 至 2022)、以及许多 Linux 发行版(Red Hat Enterprise 7 或更高版本、Fedora 29 或更高版本、Ubuntu 16.04 或更高版本,以及其他更多系统)上运行。同时,它还支持 x64、ARM64 和 ARM32 架构。有关受支持操作系统的最新列表,请参阅 \url{https://docs.github.com/en/actions/hosting-your-own-runners/managing-self-hosted-runners/about-self-hosted-runners#supported-architectures-and-operating-systems-for-self-hosted-runners}。您可以在运行器应用程序文件夹中使用 bin/installdependencies.sh 脚本来安装 .NET Core 运行时所需的所有库。

如果您想运行基于 Docker 的操作,则必须使用 Linux 镜像。Windows 和 macOS 不支持运行基于 Docker 的操作!

\mySamllsectionNoContent{GitHub的身份验证}

将运行器连接到 GitHub 是通过一个可通过 GitHub 用户界面由用户生成的配置令牌完成的。这个令牌的有效期仅为1小时,且仅能用于安装运行器。您也可以通过发送 POST 请求到 REST API 来按需创建一个安装令牌,对于仓库级别的运行器,请求地址为 https://api.github.com/repos/{OWNER}/{REPO}/actions/runners/registration-token;对于组织级别的运行器,则为 https://api.github.com/orgs/{ORG}/actions/runners/registration-token。

\begin{shell}
$ curl -L \
> -X POST \
> -H "Accept: application/vnd.github+json" \
> -H "Authorization: Bearer <YOUR-PAT>" \
> -H "X-GitHub-Api-Version: 2022-11-28" \
> https://api.github.com/repos/{OWNER}/{REPO}/actions/runners/registration-token
\end{shell}

结果中也包含了过期日期。如果您想将令牌存储在一个变量中,可以将结果通过管道传递给 jq,如下所示:

\begin{shell}
$ TOKEN=$(<curl command> | jq .token --raw-output)
\end{shell}

要使用此端点,您必须使用具有 repo 范围的 PAT 访问令牌进行身份验证。GitHub 应用程序必须对仓库有管理权限,并对组织有 organization\_self\_hosted\_runners 权限。已认证用户需要对仓库或组织拥有管理员访问权限,或者对于企业而言,需要 manage\_runners:enterprise 范围。

该令牌仅用于注册。在注册过程中,服务器会发送一个仅具有监听队列权限的 JWT(OAuth 交换的 JSON Web Token)。当工作流运行开始时,会为构建的生命周期创建另一个具有有限范围(由工作流定义)的预构建令牌。该令牌无法通过临时脚本或不受信任的代码访问——只能由构建代理和任务访问。代理与服务器之间的 OAuth 令牌交换所用的 RSA 私钥将被存储在一个名为 .credentials\_rsaparams 的文件中,而服务器持有公钥。每隔50分钟,服务器会发送一个新的由公钥加密的令牌给代理。OAuth 配置则存储在 .credentials 文件中:

\begin{shell}
{
  "scheme": "OAuth",
  "data": {
    "clientId": "{CLIENT_ID}",
    "authorizationUrl": "https://pipelinesghubeus4.actions.githubusercontent.com/{TOKEN}/_apis/oauth2/token",
    "requireFipsCryptography": "True"
  }
\end{shell}

\mySamllsectionNoContent{运行应用程序作为服务}

在 Windows 上,配置脚本会询问您是否希望将运行器作为服务执行,这样它将在环境启动时一并启动。在 Linux 上,您需要使用 svc.sh 脚本自行配置服务:

\begin{shell}
sudo ./svc.sh install
sudo ./svc.sh start
\end{shell}

\mySamllsectionNoContent{网络通信}

运行器应用程序通过端口 443 上的出站 HTTPS 连接与 GitHub 通信,使用带有 50 秒超时的长轮询。这意味着应用程序会询问 GitHub 是否有排队的工作适合该运行器的标签,然后等待 50 秒以获取响应,之后连接关闭。一旦连接关闭,就会立即开启一个新的连接。不需要任何来自 GitHub 的入站连接或打开任何防火墙端口。只需要确保使用 SSL 通过端口 443 进行安全的出站连接即可。

\mySamllsectionNoContent{更新自托管运行器}

自托管运行器会自动检查是否有可用的新版本的运行器应用程序,并进行更新。GitHub 仅更新运行器本身——机器的其余部分由客户管理。

\mySamllsectionNoContent{清理}

值得注意的是,GitHub 运行器应用程序不会在工作流运行后清理资源。这种行为与 GitHub 托管的运行器不同,因为它们为每个工作流运行提供了一个临时的清新环境。如果您下载了仓库并执行构建,所有文件都将保留在那里。如果您想让一个运行器应用程序用于多个工作流运行,则必须自己清理一切。因此,推荐在容器中使用临时运行器,这样每次都有一个干净的环境。

可以使用工作流逻辑在工作流运行后进行清理——也可以使用预作业或后作业脚本来在运行器上完成此操作。要配置预作业或后作业脚本,需要将脚本文件保存到运行器可访问的位置,然后配置一个环境变量,其名称为以下之一,值为脚本的完整路径:

\begin{itemize}
\item 
ACTIONS\_RUNNER\_HOOK\_JOB\_STARTED

\item 
ACTIONS\_RUNNER\_HOOK\_JOB\_COMPLETED
\end{itemize}

作为替代方案,您可以在运行器应用程序目录内的 .env 文件中存储键值对。这样可以根据需要自动化一些清理任务或其他准备工作。

\mySubsubsection{4.2.4}{There’s more…}

在 macOS 上安装运行器的过程与在 Linux 上相同。不同之处在于 Windows,脚本是 PowerShell 脚本而不是 bash 脚本。例如,它使用 Invoke-WebRequest 而不是 curl,但所有步骤都是相同的。用于配置和启动运行器的脚本扩展名为 .cmd,而不是 .sh:

\begin{shell}
$ ./config.cmd --url <URL> --token <TOKEN>
$ ./run.cmd
\end{shell}

如果您已经成功在 Linux 容器中安装了运行器,那么在 Windows 上安装也不会有任何问题。

