
到目前为止,只使用了 ubuntu-latest 标签来运行我们的作业,可在 GitHub 托管的最新 Ubuntu 镜像上运行工作流。但还有 macOS 和 Windows 上的运行器,以及不同的配置,可以在任何平台上托管自己的运行器。在第一个示例中,将在一个 Linux Docker 容器中设置一个自托管运行器。这样,在工作流运行之后,将很容易扩展并清理资源。

\mySubsubsection{4.2.1}{Getting ready}

需要为这个示例安装 Docker,还需要知道处理器的架构。如果不知道,只需运行 docker info 并查找 "Architecture"即可:

\begin{shell}
$ docker info | grep Architecture
\end{shell}

\mySubsubsection{4.2.2}{How to do it…}

\begin{enumerate}
\item 
可以只用在上一个示例中使用的库,或可以创建一个新库,也可以使用在第 1 章中创建的 GitHubActionsCookbook 库。转到\textbf{Settings | Actions | Runners}(设置|操作|运行器)(/settings/actions/runners)并点击\textbf{New self-hosted}(新自托管运行器)(见图 4.1):

\myGraphic{0.6}{content/chapter4/images/1.png}{图4.1 --- 
可以通过进入库的 \textbf{Settings}(设置) 区域来添加自托管运行器}

这将重定向你到 /settings/actions/runners/new。为运行器镜像选择 Linux,并根据 Docker 环境设置架构属性。

注意,每个平台和处理器架构的不同脚本。可以复制整个脚本来在虚拟机上安装,因为在 Docker 容器中,所以需要增加一些步骤;也可以复制单行(见图 4.2):

\myGraphic{0.6}{content/chapter4/images/2.png}{图4.2 --- 
不同平台上安装自托管运行器的脚本}

\item 
最新版本的 Ubuntu 容器中启动一个控制台:

\begin{shell}
$ docker run -it ubuntu:latest /bin/bash
\end{shell}

\item 
运行脚本的第一行。这将创建一个运行器文件夹并切换到该目录:

\begin{shell}
$ mkdir actions-runner && cd actions-runner
\end{shell}

\item 
为了下载运行器二进制文件,必须在容器中安装 curl:

\begin{shell}
$ apt-get -y update; apt-get -y install curl
\end{shell}

\item 
现在,执行从脚本中下载最新运行器包的那一行。只需从浏览器复制并粘贴到控制台即可:

\begin{shell}
$ curl -o actions-runner-{version}.tar.gz -L https://{URL}.tar.gz
\end{shell}

\item 
使用脚本中的 tar 命令解压包:

\begin{shell}
tar xzf ./actions-runner-{version}.tar.gz
\end{shell}

\item 
通过执行以下脚本来安装运行器所需的所有依赖项:

\begin{shell}
$ ./bin/installdependencies.sh
\end{shell}

\item 
在配置运行器之前,必须允许它以 root 用户身份运行,容器默认会以 root 用户身份运行。我们可以通过将 RUNNER\_ALLOW\_RUNASROOT 环境变量设置为一个非零值来实现:

\begin{shell}
$ export RUNNER_ALLOW_RUNASROOT="1"
\end{shell}

\item 
现在,我们可以运行配置脚本。如果执行其他步骤花费了太长时间,可能需要在浏览器中刷新页面,令牌只在短时间内有效。复制并执行包含令牌的那一行:

\begin{shell}
./config.sh --url https://github.com/{OWNER}/{REPO} --token {TOKEN}
\end{shell}

按回车键并接受所有默认值。

执行脚本后,可以导航回\textbf{Settings | Actions | Runners}(设置|操作|运行器)以查看新注册的运行器。会看到它仍然离线,这是因为还没有启动运行器进程(见图 4.3):

\myGraphic{0.6}{content/chapter4/images/3.png}{图4.3 --- 配置好的但未运行的运行器,会显示状态为离线}

\item 
使用以下脚本来启动运行器:

\begin{shell}
$ ./run.sh
\end{shell}

运行器将变为空闲状态,则意味着正在等待执行工作流。

\item 
创建一个使用自托管标签的简单工作流程。bash 应该在所有平台上都可用,所以可以省略为 Linux 添加的标签,但也可以通过运行 [self-hosted, Linux] 进行实现:

\begin{shell}
name: Self-Hosted

on: [workflow_dispatch]

jobs:
  main:
    runs-on: self-hosted
    steps:
      - name: Output environment
      shell: bash
      run: |-
        echo "Runner Name: '${{ runner.name }}'"
        echo "Runner OS: '${{ runner.os }}'"
        echo "Runner ARCH: '${{ runner.arch }}'"
\end{shell}

\item 
执行工作流程并监控 Docker 容器,以查看它如何执行工作流程,可以根据需要重复此步骤。只要容器正在运行,就会执行所有带有匹配标签的工作流。

\item 
如果现在终止容器,运行器将保持在 GitHub 上离线状态。想要删除,请导航回\textbf{Settings | Actions | Runners}(设置 | 操作 | 运行器),并从运行器右侧的菜单中选择\textbf{Remove runner}(删除运行器)(见图 4.4):

\myGraphic{0.6}{content/chapter4/images/4.png}{图4.4 --- 从 GitHub 中移除运行器}

运行对话框提供的脚本来删除运行器:

\begin{shell}
$ ./config.sh remove --token {TOKEN}
\end{shell}

现在,运行器将从 GitHub 中删除。

\item 
还可以配置运行器,使其仅运行一个作业然后自行注销,这在容器中非常有意义。为此,可以在生成新令牌后,在配置步骤中添加 -{}-ephemeral 开关:

\begin{shell}
./config.sh --url {URL} --token {TOKEN} --ephemeral
\end{shell}

\item 
再次运行工作流程,会看到运行器在执行后将自动删除。

\item 
接下来,可将我们学到的所有内容放入一个 Dockerfile 中(可以使用:\url{https://github.com/wulfland/GitHubActionsCookbook/blob/main/SelfHostedRunner/Dockerfile}提供的文件)。这样,可以创建一个可重用的 Docker 镜像,将在每次运行时自动注册、等待作业、执行作业,然后终止。

我们为了简单起见,从 ubuntu:latest 继承。可以轻松地将其替换为包含所有构建工具的基础镜像:

\begin{shell}
FROM ubuntu:latest
\end{shell}

设置将用于连接的变量。将 TOKEN 和 RUNNER\_NAME 留空,这些值将在容器启动时提供,而不是在镜像创建期间,并设置正确的 URL、平台和版本:

\begin{shell}
ENV TOKEN=
ENV RUNNER_NAME=
ENV RUNNER_URL="https://github.com/{owner}/{repo}"
ENV GH_RUNNER_PLATFORM="linux-arm64"
ENV GH_RUNNER_VERSION="2.311.0"
ENV LABELS="self-hosted,ARM64,Linux"
ENV RUNNER_GROUP="Default"
\end{shell}

安装缺失的软件包之前,必须将 DEBIAN\_FRONTEND 设置为 noninteractive,以确保在 Docker 镜像构建过程中操作系统不会提示用户输入:

\begin{shell}
ARG DEBIAN_FRONTEND=noninteractive
\end{shell}

为了在 Docker 镜像中减少层数,最好将整个脚本合并为一个 RUN 命令。该脚本更新包管理器及其所有包,安装所有依赖项,添加容器将以其下运行的 docker 用户(不希望容器以 root 用户运行),下载相应的包并解压,将所有者更改为 docker 用户并执行 installdependencies.sh 脚本:

\begin{shell}
RUN apt-get -y update && \
  apt-get upgrade -y && \
  useradd -m docker && \
  apt-get install -y --no-install-recommends curl ca-certificates && \
  mkdir -p /opt/hostedtoolcache /home/docker/actions-runner && \
  curl -L \
  https://github.com/actions/runner/releases/download/v${GH_RUNNER_VERSION}/
  actions-runner-${GH_RUNNER_PLATFORM}-${GH_RUNNER_VERSION}.tar.gz \
  -o /home/docker/actions-runner/actionsrunner.tar.gz && \
  tar xzf /home/docker/actions-runner/actions-runner.tar.gz -C /home/docker/actions-runner && \
  chown -R docker /home/docker && \
  /home/docker/actions-runner/bin/installdependencies.sh
\end{shell}

以 docker 用户身份运行容器,并将工作目录设置为该用户主目录下的 actions-runner:

\begin{shell}
USER docker
WORKDIR /home/docker/actions-runner
\end{shell}

如果容器将要运行,必须检查是否提供了 TOKEN 和 RUNNER\_NAME。然后,使用所有参数(包括 -{}-ephemeral)运行配置脚本,然后运行 run.sh 脚本:

\begin{shell}
CMD if [ -z "$TOKEN" ]; then echo 'TOKEN is not set'; exit 1; fi
&& \
  if [ -z "$RUNNER_NAME" ]; then echo 'RUNNER_NAME is not set';
exit 1; fi && \
  ./config.sh --url "${RUNNER_URL}" --token "${TOKEN}"
--name "${RUNNER_NAME}" --work "_work" --labels "${LABELS}"
--runnergroup "${RUNNER_GROUP}" --unattended --ephemeral && \
  ./run.sh
\end{shell}

\item 
进入包含 Dockerfile 的文件夹,并根据该文件创建一个 Docker 镜像:

\begin{shell}
$ docker build -t simple-ubuntu-runner .
\end{shell}

\item 
现在,使用 docker run 运行镜像的任意多个实例。 -d (-{}-detached) 选项将在后台以分离模式运行容器,这不会阻塞控制台,但也不会在终端接收输入或显示输出。 -{}-rm 选项将在容器退出时删除它。使用 -e 选项传递 RUNNER\_NAME 和 TOKEN 的参数。需要注意的时,这些名称区分大小写!

\begin{shell}
$ docker run -d --rm -e RUNNER_NAME=Runner1 -e TOKEN={TOKEN} simple-ubuntu-runner
\end{shell}

能在库\textbf{Settings}(设置)区域看到运行器,如图 4.5 所示:

\myGraphic{0.6}{content/chapter4/images/5.png}{图4.5 --- 通过多次运行相同的 Docker 镜像来获得临时运行器,这些运行器将等待传入的任务}

\end{enumerate}

根据创建的容器数量,多次启动工作流程,并在执行后清理容器。

\mySubsubsection{4.2.3}{How it works…}

来了解一下流程的工作原理。

\mySamllsectionNoContent{自托管运行器应用程序}

自托管运行器是通过安装开源的运行器应用程序(\url{https://github.com/actions/runner})创建,创建的。该应用程序基于 .NET Core 运行时,可以在许多操作系统和处理器架构上运行。它可以在 macOS 11(Big Sur)或更高版本、Windows(7 到 10 和 Server 2012 R2 到 20222)、以及许多 Linux 发行版(Red Hat Enterprise 7 或更高版本、Fedora 29 或更高版本、Ubuntu 16.04 或更高版本,等等)上运行。它还可以在 x64、ARM64 和 ARM32 上运行。同时,它还可以在 x64、ARM64 和 ARM32 上运行。有关支持的操作系统的最新列表,请参阅 \url{https://docs.github.com/en/actions/hosting-your-own-runners/managing-self-hosted-runners/about-self-hosted-runners#supported-architectures-and-operating-systems-for-self-hosted-runners}。可以使用运行器应用程序文件夹中的 bin/installdependencies.sh 脚本来安装 .NET Core 运行时所需的所有库。

如果要运行基于 Docker 的操作,必须使用 Linux 镜像。Windows 和 macOS 不支持运行基于 Docker 的操作!

\mySamllsectionNoContent{GitHub的身份验证}

使用通过 GitHub UI 生成配置令牌来连接运行器到 GitHub。该令牌仅有效 1 小时,并且只能使用该令牌来安装运行器。也可以通过向 https://api.github.com/repos/{OWNER}/{REPO}/actions/runners/registration-token(或对于组织级别的运行器,https://api.github.com/orgs/{ORG}/actions/runners/registration-token)发送 POST 请求,按需通过 REST API 创建安装令牌。

以下是使用个人访问令牌(PAT, personal access token)接收令牌的示例:

\begin{shell}
$ curl -L \
> -X POST \
> -H "Accept: application/vnd.github+json" \
> -H "Authorization: Bearer <YOUR-PAT>" \
> -H "X-GitHub-Api-Version: 2022-11-28" \
> https://api.github.com/repos/{OWNER}/{REPO}/actions/runners/registration-token
\end{shell}

结果还包含到期日期。如果想在变量中使用令牌,可以将结果管道传递给 jq:

\begin{shell}
$ TOKEN=$(<curl command> | jq .token --raw-output)
\end{shell}

必须使用具有 repo 范围的 PAT 访问令牌来使用此端点,GitHub Apps 必须具有库的管理权限和组织的 organization\_self\_hosted\_runners 权限。经过身份验证的用户必须具有对库或组织的管理访问权限,或企业的 manage\_runners:enterprise 范围。该令牌仅用于注册。

在注册过程中,将从服务器接收到一个 JWT(OAuth 交换的 JSON Web 令牌),只有权限监听队列。当工作流程运行开始时,将为构建的生命周期创建另一个预构建的、有限范围(由工作流程定义)的令牌。该令牌无法通过临时脚本或不可靠的代码访问 - 只能由构建代理和任务访问。OAuth 令牌交换之间代理和服务器之间的 RSA 私钥将存储在一个名为 .credentials\_rsaparams 的文件中,服务器持有公钥。每 50 分钟,服务器将向代理发送一个由公钥加密的新令牌。OAuth 配置存储在 .credentials 文件中:

\begin{shell}
{
  "scheme": "OAuth",
  "data": {
    "clientId": "{CLIENT_ID}",
    "authorizationUrl": "https://pipelinesghubeus4.actions.githubusercontent.com/{TOKEN}/_apis/oauth2/token",
    "requireFipsCryptography": "True"
  }
\end{shell}

\mySamllsectionNoContent{运行应用程序作为服务}

在 Windows 上,配置脚本会询问你是否要作为服务执行运行器,以便它与环境一起启动。在 Linux 上,需要使用 svc.sh 脚本自己配置服务:

\begin{shell}
sudo ./svc.sh install
sudo ./svc.sh start
\end{shell}

\mySamllsectionNoContent{网络通信}

运行器应用程序使用出站 HTTPS 连接通过端口 443 与 GitHub 通信,使用具有 50 秒超时的长轮询。所以应用程序询问 GitHub 是否有工作排队等待运行器的标签,然后在连接关闭之前等待 50 秒的响应。连接关闭后,立即启动一个新连接。无需 GitHub 的入站连接或打开防火墙端口,仅在端口 443 上使用 SSL 的安全出站连接。

\mySamllsectionNoContent{更新自托管运行器}

自托管运行器将自动检查是否有可用的新版本运行器应用程序,并更新它。GitHub 只会更新运行器本身 - 机器的其余部分由客户管理。

\mySamllsectionNoContent{清理}

值得注意的是,GitHub 运行器应用程序不会在 workflow 运行后清理资源。这种行为与 GitHub 托管的运行器不同,它们为每个 workflow 运行提供临时干净的环境。如果下载库并执行构建,所有文件将保留在那里。如果想要为多个 workflow 运行使用运行器应用程序,则必须自己清理所有内容。这样,总是有一个干净的环境。

可以使用 workflow 逻辑,在库的 workflow 运行后进行清理 - 也可以使用前工作或后工作脚本来在运行器上执行此操作。要配置前工作或后工作脚本,需要将一个脚本文件保存在运行器可以访问的位置,然后使用以下名称之一和脚本的完整路径作为值配置一个环境变量:

\begin{itemize}
\item 
ACTIONS\_RUNNER\_HOOK\_JOB\_STARTED

\item 
ACTIONS\_RUNNER\_HOOK\_JOB\_COMPLETED
\end{itemize}

或者,你可以将键值对存储在运行器应用程序目录内的 .env 文件中。

\mySubsubsection{4.2.4}{There’s more…}

在 macOS 上安装运行器与在 Linux 上安装相同。Windows 上的不同之处在于,脚本是一个 PowerShell 脚本,而不是 bash 脚本。例如,使用 Invoke-WebRequest 而不是 curl,但所有步骤都是相同的。用于配置和启动运行器的脚本的扩展名为 .cmd,而不是 .sh:

\begin{shell}
$ ./config.cmd --url <URL> --token <TOKEN>
$ ./run.cmd
\end{shell}

如果已经在 Linux 容器中成功安装了运行器,则在 Windows 上安装也不会有问题。

