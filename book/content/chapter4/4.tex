Kubernetes 非常强大,但也相当复杂。本示例中,我将只专注于让你开始使用 Kubernetes 扩展自托管运行器。

ARC 是一个 Kubernetes 运营商,能编排和扩展自托管运行器的工作负载。这是一个开源项目,但现在已得到 GitHub 的完全支持。

\mySubsubsection{4.4.1}{Getting ready}

如果已经有一个 Kubernetes 集群,可以使用它。如果没有,可以通过运行以下命令在 Azure 中创建一个新集群:

\begin{shell}
$ az group create --name AKSCluster -l westeurope
$ az aks create --resource-group AKSCluster \
> --name AKSCluster \
> --node-count 3 \
> --enable-addons monitoring \
> --generate-ssh-keys
$ az aks get-credentials --resource-group AKSCluster --name AKSCluster
\end{shell}

确保集群中安装了 cert-manager(\url{https://cert-manager.io/docs/installation/}),可以通过运行以下命令来完成此操作,确保为最新版本:

\begin{shell}
$ kubectl apply -f https://github.com/cert-manager/cert-manager/releases/download/v1.13.2/cert-manager.yaml
\end{shell}

可以检查 ARC 的快速入门教程中先决条件是否有变化:\url{https://github.com/actions/actions-runner-controller/blob/master/docs/quickstart.md}。

\mySubsubsection{4.4.2}{How to do it…}

\begin{enumerate}
\item 
将 ARC 部署到集群中,确保版本为最新版本:

\begin{shell}
$ kubectl apply -f https://github.com/actions/actions-runnercontroller/releases/download/v0.23.7/
actions-runner-controller.yaml
\end{shell}

\item 
在 GitHub 中创建一个具有 repo 范围的 PAT。转到 \url{https://github.com/settings/tokens/new},选择 repo,设置过期日期,然后点击生成令牌,再复制该令牌。

\item 
现在,将令牌保存为 Kubernetes 中的一个秘钥:

\begin{shell}
$ kubectl create secret generic controller-manager \
> -n actions-runner-system \
> --from-literal=github_token=<YOUR_TOKEN>
\end{shell}

\item 
在库中创建一个名为 runnerdeployment.yml 的文件,并包含以下内容。将库所有者和名称替换为库的值:

\begin{shell}
apiVersion: actions.summerwind.dev/v1alpha1
kind: RunnerDeployment
metadata:
  name: example-runnerdeploy
spec:
  replicas: 1
  template:
    spec:
      repository: wulfland/GitHubActionsCookbook
\end{shell}

\item 
将 RunnerDeployment 应用到集群:

\begin{shell}
$ kubectl apply -f runnerdeployment.yml
\end{shell}

现在,应该有一个运行器和两个正在运行的 pod:

\begin{shell}
$ kubectl get runners
$ kubectl get pods
\end{shell}

验证否可以在 GitHub 中看到运行器(settings/actions/runners)。请注意,每次工作流程运行后,名称都会更改(见图 4.13):

\myGraphic{0.6}{content/chapter4/images/13.png}{图4.13 --- GitHub 中的 ARC 运行器}

从之前的配方中运行 .github/workflows/self-hosted.yml 工作流程并检查输出,在 AKS 集群中执行(见图 4.14):

\myGraphic{0.6}{content/chapter4/images/14.png}{图4.14 --- 自托管工作流程在 Kubernetes 集群中由 ARC 执行}
\end{enumerate}

\mySubsubsection{4.4.3}{How it works…}

ARC 运行器默认设置为临时运行,使用 Kubernetes 副本集,并在执行后自动启动一个新的容器。ARC 提供三种扩展选项:

\begin{itemize}
\item 
定时:根据计划进行扩展和缩减

\item 
基于规模:执行作业的繁忙运行器的百分比

\item 
按需:当新的工作流程作业排队时启动实例
\end{itemize}

ARC 支持在企业、组织和库级别创建运行器。可以使用 GitHub 应用程序或 PAT 在组织和库级别进行身份验证。使用企业库时,必须使用 PAT,因为应用程序不能限定在企业范围内。

可以为不同的团队配置具有不同镜像和命名空间的扩展集,也允许限制它们之间的网络访问。

\mySubsubsection{4.4.4}{There’s more…}

对软件供应链和构建过程的攻击构成了一个巨大的威胁。应该非常小心,特别是对于自托管运行器。永远不要在允许派生的公共库中使用,并采取措施控制在运行器上拉取的所有依赖项,以便没有人可以执行以下操作:

\begin{itemize}
\item 
篡改构建过程中的文件

\item 
逃离容器/沙盒或访问工作流程范围之外的文件

\item 
通过操纵依赖项缓存(缓存中毒)来损害依赖项

\item 
窃取数据或机密,并将数据发送到控制服务器
\end{itemize}

可以在工作流程中使用 StepSecurity 的 Harden-Runner 操作(\url{https://github.com/stepsecurity/harden-runner}),将根据策略(如出站网络出口)监控工作流:

\begin{shell}
steps:
  - uses: step-security/harden-runner@v2.6.1
    with:
      egress-policy: audit
\end{shell}

也可以使用它来阻止流量,只允许特定的端点和端口:

\begin{shell}
egress-policy: block
allowed-endpoints: >
  api.nuget.org:443
  github.com:443
\end{shell}

对于 ARC,不必将 Harden-Runner 操作添加到每个工作流程作业中,可以在 Kubernetes 集群上安装 ARC Harden-Runner DaemonSet。该 DaemonSet 将不断监控每个工作流程的运行,而无需将操作添加到每个工作流程中。

可以在仪表板的运行时安全选项卡下访问安全见解和运行时检测。

请注意,这不是免费软件。对于 \url{https://github.com} 上的公共库,提供免费的社区许可证;但对于私有库或 GitHub Enterprise Server (GHES),需要购买许可证(\url{https://www.stepsecurity.io/pricing})。



