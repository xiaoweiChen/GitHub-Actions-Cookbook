Kubernetes 功能强大,但也相当复杂。请注意,在这个指南中,我只会专注于让你开始如何在 Kubernetes 中扩展自托管运行器。如果你想要运行和维护一个安全的环境,你需要更深入的 Kubernetes 知识,并且必须根据你的需求承担更多的工作。

ARC 是一个 Kubernetes 运算符,它可以编排和扩展你的自托管运行器的工作负载。它是一个开源项目,但现在已得到 GitHub 的全面支持。

\mySubsubsection{4.4.1}{Getting ready}

如果你已经有一个 Kubernetes 集群,你可以直接使用它。如果没有,你可以在 Azure 中通过运行以下命令创建一个新的集群:

\begin{shell}
$ az group create --name AKSCluster -l westeurope
$ az aks create --resource-group AKSCluster \
> --name AKSCluster \
> --node-count 3 \
> --enable-addons monitoring \
> --generate-ssh-keys
$ az aks get-credentials --resource-group AKSCluster --name AKSCluster
\end{shell}

确保你在集群中安装了 cert-manager(\url{https://cert-manager.io/docs/installation/})。你可以通过运行以下命令来完成安装。请确保将版本号替换为最新的版本:

\begin{shell}
$ kubectl apply -f https://github.com/cert-manager/cert-manager/releases/download/v1.13.2/cert-manager.yaml
\end{shell}

你可以检查 ARC 的快速入门教程中先决条件是否有变化:\url{https://github.com/actions/actions-runner-controller/blob/master/docs/quickstart.md}。

\mySubsubsection{4.4.2}{How to do it…}

\begin{enumerate}
\item 
将 ARC 部署到你的集群中。确保你将其版本更改为最新版本:

\begin{shell}
$ kubectl apply -f https://github.com/actions/actions-runnercontroller/releases/download/v0.23.7/actions-runner-controller.yaml
\end{shell}

\item 
在 GitHub 上创建一个具有仓库权限的 PAT(个人访问令牌)。访问 \url{https://github.com/settings/tokens/new},选择“repo”,设置过期日期,然后点击“Generate token”(生成令牌)。接着,复制该令牌。

\item 
现在,将令牌保存为 Kubernetes 中的秘钥:

\begin{shell}
$ kubectl create secret generic controller-manager \
> -n actions-runner-system \
> --from-literal=github_token=<YOUR_TOKEN>
\end{shell}

\item 
在您的仓库中创建一个名为 runnerdeployment.yml 的文件,并包含以下内容。将仓库所有者和名称替换为您的仓库的实际值:

\begin{shell}
apiVersion: actions.summerwind.dev/v1alpha1
kind: RunnerDeployment
metadata:
  name: example-runnerdeploy
spec:
  replicas: 1
  template:
    spec:
      repository: wulfland/GitHubActionsCookbook
\end{shell}

\item 
将 RunnerDeployment 应用到您的集群中:

\begin{shell}
$ kubectl apply -f runnerdeployment.yml
\end{shell}

现在,您应该有一个运行器和两个pods在运行:

\begin{shell}
$ kubectl get runners
$ kubectl get pods
\end{shell}

验证您是否可以在 GitHub 中看到该运行器(设置/操作/运行器)。请注意,每次工作流程运行后名称都会更改(见图 4.13):

\myGraphic{0.4}{content/chapter4/images/13.png}{图4.13 --- GitHub 上的 ARC 运行器}

运行你之前配方中的 .github/workflows/self-hosted.yml 工作流程,并检查输出。注意,它是在你的 AKS 集群中执行的(见图 4.14):

\myGraphic{0.4}{content/chapter4/images/14.png}{图4.14 --- GitHub 上的 ARC 运行器}

自托管的 GitHub Actions 工作流程由部署在 Kubernetes 集群中的 ARC(Actions Runner Controller)执行。
\end{enumerate}

\mySubsubsection{4.4.3}{How it works…}

ARC 运行器默认设置为临时性。它们使用 Kubernetes 副本集,并在执行后自动启动一个新的容器。ARC 提供三种扩展选项:

\begin{itemize}
\item 
定时扩展:根据时间表进行扩展和缩减

\item 
基于负载的扩展:根据正在执行作业的运行器的百分比进行扩展

\item 
按需扩展:当新的工作流程作业排队时启动实例
\end{itemize}

ARC 支持在企业、组织和仓库级别创建运行器。您可以使用 GitHub 应用程序或 PAT 在组织和仓库级别进行身份验证。但是,在企业级别,您必须使用 PAT,因为应用程序不能限定在企业范围内。

您可以为不同的团队配置具有不同镜像和命名空间的扩展集。这也允许您限制它们之间的网络访问。

\mySubsubsection{4.4.4}{There’s more…}

攻击软件供应链和构建流程是一个很大的威胁。你应该非常小心,特别是对于自托管的运行器。永远不要在允许分叉的公共代码仓库中使用它们,并采取措施控制你在运行器上拉取的所有依赖项,以便没有人能够执行以下操作:

\begin{itemize}
\item 
在你的构建过程中篡改文件

\item 
逃逸你的容器/沙盒或访问工作流程范围之外的文件

\item 
通过操纵依赖缓存(缓存投毒)来危害你的依赖项 

\item 
窃取数据或机密并将其发送到控制服务器。
\end{itemize}

你可以在工作流中使用 StepSecurity 提供的 Harden-Runner 动作(\url{https://github.com/stepsecurity/harden-runner})。它将根据诸如出站网络流量等策略监控你的工作流,例如:

\begin{shell}
steps:
  - uses: step-security/harden-runner@v2.6.1
    with:
      egress-policy: audit
\end{shell}

你还可以用它来阻止流量,并仅允许特定的端点和端口:

\begin{shell}
egress-policy: block
allowed-endpoints: >
  api.nuget.org:443
  github.com:443
\end{shell}

对于 ARC,你无需在每个工作流任务中添加 Harden-Runner 动作,而是可以在 Kubernetes 集群上安装 ARC Harden-Runner DaemonSet。该 DaemonSet 将持续监控每个工作流运行,而无需在每个工作流中添加动作。

你可以在仪表板的“运行时安全”选项卡下访问安全洞察和运行时检测信息。

请注意,这不是免费软件。对于 \url{https://github.com} 上的公共存储库,提供免费的社区许可证;但对于私有存储库或 GitHub Enterprise Server (GHES),你需要购买许可证(详情请参见 \url{https://www.stepsecurity.io/pricing})。



