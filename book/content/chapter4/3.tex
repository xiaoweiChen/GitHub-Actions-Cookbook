在这个示例中,我们将在上一个指南的基础上进行构建,以便有一个解决方案,每次触发新工作流程时,都会自动启动一个新的临时 Docker 容器实例。这次,将使用 GitHub Webhook 进行实现。

\mySubsubsection{4.3.1}{Getting ready}

确保机器或 GitHub Codespaces 上仍然有上一个示例创建的 simple-ubuntu-runner Docker 镜像。

\mySubsubsection{4.3.2}{How to do it…}

\begin{enumerate}
\item 
前往 https://github.com/settings/apps。点击 \textbf{New GitHub App}(新建 GitHub 应用)。

\item 
将 \textbf{GitHub App Name} 设置为 auto-scale-runners,并将 \textbf{Homepage URL} 设置为正在使用库的 URL(参见图 4.6):

\myGraphic{0.6}{content/chapter4/images/6.png}{图4.6 --- 设置新应用程序的名称和 URL}

\item 
跳过 \textbf{Identifying and authorizing users}(识别和授权用户)以及 \textbf{Post installation}(安装后)部分,直接进入 Webhook 配置。

\item 
打开另一个浏览器标签页,访问 \url{https://smee.io},然后点击 \textbf{Start new channel}(启动新频道),再复制生成的 \textbf{Webhook Proxy URL} 值。

\item 
返回到创建 GitHub 应用的标签页,将复制的 \textbf{Webhook Proxy URL} 粘贴到 \textbf{Webhook URL} 字段中。在 Webhook secret 字段中设置一个能记住的字符串(参见图 4.7)。

\myGraphic{0.6}{content/chapter4/images/7.png}{图4.7 ---  配置Webhook}

\item 
在 \textbf{Permissions}(权限) 下的 \textbf{Repository permissions}(仓库权限) 中,将 \textbf{Actions}(操作) 设置为 \textbf{Read-only}(只读),并将 \textbf{Administration}(管理) 设置为 \textbf{Read and write}(读写)(参见图 4.8)。

\myGraphic{0.6}{content/chapter4/images/8.png}{图4.8 --- 为应用程序配置库权限}

\item 
在 \textbf{Subscribe to events}(订阅事件) 部分,勾选 W\textbf{orkflow job}(工作流作业)(参见图 4.9)。

\myGraphic{0.6}{content/chapter4/images/9.png}{图4.9 --- 订阅工作流作业 webhook}

\item 
在 \textbf{Where can this GitHub App be installed}(此 GitHub 应用可以安装在哪里) 下,选择 \textbf{Only on this account}(仅在此账户上),然后点击 \textbf{Create GitHub App}(创建 GitHub 应用)。

\item 
在您新创建的应用中,点击 \textbf{Generate a private key}(生成私钥)。私钥将会自动下载,将其移动到库中。

\item 
从应用的 \textbf{General}(通用) 标签页中复制 \textbf{App ID} 值(参见图 4.10)。

\myGraphic{0.6}{content/chapter4/images/10.png}{图4.10 --- 获取应用程序 ID 值}

\item 
在库中,创建一个名为 .env 的新文件,并添加具有相应值的 APP\_ID、WEBHOOK\_SECRET 和 PRIVATE\_KEY\_PATH 变量:

\begin{shell}
APP_ID="653496"
WEBHOOK_SECRET="YOUR_SECRET"
PRIVATE_KEY_PATH="auto-scale-runners.2023-11-26.private-key.pem"
\end{shell}

后将在我们的应用程序中,会使用这些环境变量来对 GitHub 进行身份验证。

\item 
将 .env 文件添加到 .gitignore 文件中,以避免意外提交:

\begin{shell}
$ echo ".env" >> .gitignore
\end{shell}

\item 
在github的应用中,选择“安装应用”,然后单击安装(请参见图4.11):

\myGraphic{0.6}{content/chapter4/images/11.png}{图4.11 --- 安装应用程序}

\item 
在出现的对话框中选择库,然后单击\textbf{Install}(安装)(请参见图4.12):

\myGraphic{0.6}{content/chapter4/images/12.png}{图4.12 --- 在库中安装应用程序}

接下来,将创建一个服务器,当它从 webhook 接收到有效负载时,该服务器将运行,转到库并初始化:

\begin{shell}
$ npm init –yes
\end{shell}

添加代码中使用的依赖项:

\begin{shell}
$ npm install octokit
$ npm install dotenv
$ npm install smee-client --save-dev
\end{shell}

将 node\_modules 文件夹添加到 .gitignore:

\begin{shell}
$ echo "node_modules" >> .gitignore
\end{shell}

\item 
创建一个名为 app.js 的新文件,可以从这里复制内容:\url{https://github.com/wulfland/GitHubActionsCookbook/blob/main/SelfHostedRunner/auto-scale/app.js}。

\item 
添加必要的依赖项

\begin{shell}
import dotenv from "dotenv";
import {App} from "octokit";
import {createNodeMiddleware} from "@octokit/webhooks";
import fs from "fs";
import http from "http";
import { exec } from 'child_process';
\end{shell}

\item 
接下来,使用 dotenv 包从之前创建的 .env 文件中读取环境变量:

\begin{shell}
dotenv.config();
const appId = process.env.APP_ID;
const webhookSecret = process.env.WEBHOOK_SECRET;
const privateKeyPath = process.env.PRIVATE_KEY_PATH;
\end{shell}

\item 
接下来,从相应路径加载私钥:

\begin{shell}
const privateKey = fs.readFileSync(privateKeyPath, "utf8");
\end{shell}

\item 
创建 octokit 的 app 类的一个新实例:

\begin{shell}
const app = new App({
  appId: appId,
  privateKey: privateKey,
  webhooks: {
    secret: webhookSecret
  },
});
\end{shell}

然后,为 workflow\_job.queued 事件注册一个事件处理程序:

\begin{shell}
app.webhooks.on("workflow_job.queued", handleNewQueuedJobsRequestOpened);
\end{shell}

\item 
调用 GitHub API 以接收一个可以注册运行器的新令牌。我们需要这个令牌,以便可以将其传递给容器。必须使用 workflow\_job.id 为运行器创建名称:

\begin{shell}
const response = await octokit.request('POST /repos/{owner}/{repo}/actions/runners/registration-token', { owner: payload.repository.owner.login,
  repo: payload.repository.name,
  headers: {
    'X-GitHub-Api-Version': '2022-11-28'
  }
});

const token = response.data.token;
const runner_name = `Runner_${payload.workflow_job.id}`;
\end{shell}

\item 
然后,创建一个新的 Docker 容器实例,并传入令牌和名称:

\begin{shell}
exec(`docker run -d --rm -e RUNNER_NAME=${runner_name} -e
  TOKEN=${token} simple-ubuntu-runner`, (error, stdout, stderr) =>
  {
  if (error) {
    console.error(`exec error: ${error}`);
    return;
  }
  console.log(`stdout: ${stdout}`);
  console.error(`stderr: ${stderr}`);
});
\end{shell}

我在这里跳过了错误处理部分,因为它与此无关。

\item 
在文件的最后一部分,必须创建一个在端口 3000 上监听的开发服务器:

\begin{shell}
const port = 3000;
const host = 'localhost';
const path = "/api/webhook";
const localWebhookUrl = `http://${host}:${port}${path}`;
const middleware = createNodeMiddleware(app.webhooks, {path});
http.createServer(middleware).listen(port, () => {
console.log(`Server is listening for events at: ${localWebhookUrl}`);
console.log('Press Ctrl + C to quit.')
});
\end{shell}

\item 
在 package.json 文件中,添加一个名为 type 的顶层条目,并将其设置为 module。然后,添加一个名为 server 的脚本来运行应用程序

\begin{shell}
"type": "module",
"scripts": {
  "server": "node app.js"
},
\end{shell}

\item 
我们准备好了!打开一个新的终端,并使用你的频道 URL(第 4 步)启动一个 smee 客户端:

\begin{shell}
$ npx smee -u https://smee.io/{ID} -t http://localhost:3000/api/webhook
\end{shell}

在库的终端中,运行应用程序:

\begin{shell}
$ npm run server
\end{shell}

为自托管工作流启动一个新的工作流:

\begin{shell}
$ gh workflow run Self-Hosted
\end{shell}

注意 smee 客户端如何接收从 GitHub 转发来的 webhook,以及服务器如何处理它,并启动一个新容器来执行工作流。

\end{enumerate}

\mySubsubsection{4.3.3}{How it works…}

GitHub 应用程序提供了一个简单的方式来对 GitHub 进行身份验证并注册 webhook。在这个指南中,我们注册了 workflow\_job webhook,并设置了 queued 动作类型,以便每次有新的工作流程排队时,我们都可以启动一个新的运行器。如果有更大的镜像需要更长时间来加载,也可以有一组现有的运行器,并在作业排队时仍然加载一个新的运行器。

启动的临时运行器不一定是执行你作业的那个运行器。

由于我们需要一个 GitHub 可以访问的端点,所以使用了 smee.io 作为代理,并在事件发生时转发有效载荷。这不适用于生产环境,只是提供了一个方便的方式来在本地或 GitHub Codespace 中开发,而无需公开可用的入站端口。对于生产用途,应该将应用程序托管在 Web 服务器上。

\mySubsubsection{4.3.4}{There’s more…}

这个指南旨在提供创建自托管运行器扩展解决方案的基本构建模块。使用临时运行器和 webhook,很容易自动化这个过程。但如果需要一个更可扩展、更成熟的解决方案,应该考虑使用 Kubernetes 来实现。

